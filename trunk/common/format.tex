%&latex
\documentclass[a4paper]{article}

\frenchspacing

\usepackage[cp1250]{inputenc}
\usepackage[czech]{babel}

\usepackage{a4wide}
\usepackage{amsmath, amsthm, amssymb, amsfonts}
\usepackage[mathcal]{eucal}
\usepackage{graphicx}
\usepackage{url}
\usepackage{color}
\usepackage{wrapfig}




% sirka a vyska textu nastavena jako papir, vsechny okraje vynulovany a pridano 20pt na kazdou stranu
% horizontalni rozmery
\setlength{\textwidth}{\paperwidth}
\addtolength{\textwidth}{-40pt}
\addtolength{\hoffset}{-1in}
\addtolength{\hoffset}{20pt}
\setlength{\oddsidemargin}{0in}
\setlength{\marginparsep}{0in}
% vertikalni rozmery
\setlength{\textheight}{\paperheight}
\addtolength{\textheight}{-40pt}
\addtolength{\voffset}{-1in}
\addtolength{\voffset}{20pt}
\setlength{\topmargin}{0in}
\setlength{\headheight}{0in}
\setlength{\headsep}{0in}




%==========================================
%PEKELNA MAKRA NA ZAROVNANI OBRAZKU DOPRAVA

\makeatletter


%tohle je makro, ktere mi dovoluje obtekani i u kratkych environmentu
%ABSOLUTNE nechapu, jak to funguje, ale funguje to
%viz http://tex.stackexchange.com/questions/26078/ 
\def\odrovnej{\@@par
\ifnum\@@parshape=\z@ \let\WF@pspars\@empty \fi % reset `parshape'
\global\advance\c@WF@wrappedlines-\prevgraf \prevgraf\z@
\ifnum\c@WF@wrappedlines<\tw@ \WF@finale \fi}

\makeatother


%---
%makro, co da obrazek doprava a ostatni text ho obteka
%(bez toho predchazejiciho makra to ale poradne nebeha)
%pouziti:
%\obrazekvpravo{adresa}{popisek}{label}{procento sirky}
\long\def\obrazekvpravo#1#2#3#4{

\setlength\intextsep{-20pt}

    \begin{wrapfigure}{r}{#4\textwidth}
      \begin{center}
          \vspace{-10pt}
          
        \includegraphics[width=#4\textwidth]{#1}
        \vspace{-10pt}
        
      \end{center}
      
      \caption{#2}
      \label{#3}
      
      
    \end{wrapfigure}

\setlength\intextsep{0pt}

    
}


%---
%makro pro pripady, kdy wrapfigure neco mrsi
%je to docela pekelne
%je nutne mu dat jak text vpravo, tak text vlevo
%a nevim, jestli bude 100% fungovat, ale doufam, ze jo

%pouziti:
%\obrazekvpravominipage{adresa}{popisek}{label}{procento sirky}{1 - procento sirky}{text vlevo}
\long\def\obrazekvpravominipage#1#2#3#4#5#6{

\begin{figure}[h]
\begin{minipage}[t]{#4\linewidth}
\vspace{0pt}
#6
\end{minipage}
\hspace{0.5cm}
\begin{minipage}[t]{#5\linewidth}
\vspace{0pt}
\centering
\includegraphics[width=0.9\textwidth]{#1}
\caption{#2}
\label{#3}
\end{minipage}
\end{figure}

}

%KONEC PEKELNYCH MAKER
%=====================


%Vacsina prostredi je dvojjazicne. V pripade, ze znenie napr pozorovania je pisane po slovensky, malo by byt po slovensky aj oznacenie.

\newenvironment{pozadavky}{\pagebreak[2]\noindent\textbf{Po�adavky}\par\noindent\leftskip 10pt}{\odrovnej\par\bigskip}
\newenvironment{poziadavky}{\pagebreak[2]\noindent\textbf{Po�iadavky}\par\noindent\leftskip 10pt}{\odrovnej\par\bigskip}

\newenvironment{definice}{\pagebreak[2]\noindent\textbf{Definice}\par\noindent\leftskip 10pt}{\odrovnej\par\bigskip}
\newenvironment{definiceN}[1]{\pagebreak[2]\noindent\textbf{Definice~}\emph{(#1)}\par\noindent\leftskip 10pt}{\odrovnej\par\bigskip}
\newenvironment{definicia}{\pagebreak[2]\noindent\textbf{Defin�cia}\par \noindent\leftskip 10pt}{\odrovnej\par\bigskip}
\newenvironment{definiciaN}[1]{\pagebreak[2]\noindent\textbf{Defin�cia~}\emph{(#1)}\par\noindent\leftskip 10pt}{\odrovnej\par\bigskip}

\newenvironment{pozorovani}{\pagebreak[2]\noindent\textbf{Pozorov�n�}\par\noindent\leftskip 10pt}{\odrovnej\par\bigskip}
\newenvironment{pozorovanie}{\pagebreak[2]\noindent\textbf{Pozorovanie}\par\noindent\leftskip 10pt}{\odrovnej\par\bigskip}
\newenvironment{poznamka}{\pagebreak[2]\noindent\textbf{Pozn�mka}\par\noindent\leftskip 10pt}{\odrovnej\par\bigskip}
\newenvironment{poznamkaN}[1]{\pagebreak[2]\noindent\textbf{Pozn�mka~}\emph{(#1)}\par\noindent\leftskip 10pt}{\odrovnej\par\bigskip}
\newenvironment{lemma}{\pagebreak[2]\noindent\textbf{Lemma}\par\noindent\leftskip 10pt}{\odrovnej\par\bigskip}
\newenvironment{lemmaN}[1]{\pagebreak[2]\noindent\textbf{Lemma~}\emph{(#1)}\par\noindent\leftskip 10pt}{\odrovnej\par\bigskip}
\newenvironment{veta}{\pagebreak[2]\noindent\textbf{V�ta}\par\noindent\leftskip 10pt}{\odrovnej\par\bigskip}
\newenvironment{vetaN}[1]{\pagebreak[2]\noindent\textbf{V�ta~}\emph{(#1)}\par\noindent\leftskip 10pt}{\odrovnej\par\bigskip}
\newenvironment{vetaSK}{\pagebreak[2]\noindent\textbf{Veta}\par\noindent\leftskip 10pt}{\odrovnej\par\bigskip}
\newenvironment{vetaSKN}[1]{\pagebreak[2]\noindent\textbf{Veta~}\emph{(#1)}\par\noindent\leftskip 10pt}{\odrovnej\par\bigskip}

\newenvironment{dusledek}{\pagebreak[2]\noindent\textbf{D�sledek}\par\noindent\leftskip 10pt}{\odrovnej\par\bigskip}
\newenvironment{dosledok}{\pagebreak[2]\noindent\textbf{D�sledok}\par\noindent\leftskip 10pt}{\odrovnej\par\bigskip}

\newenvironment{dokaz}{\pagebreak[2]\noindent\leftskip 10pt\textbf{D�kaz}\par\noindent\leftskip 10pt}{\odrovnej\par\bigskip}
\newenvironment{dukaz}{\pagebreak[2]\noindent\leftskip 10pt\textbf{D�kaz}\par\noindent\leftskip 10pt}{\odrovnej\par\bigskip}

\newenvironment{ideadukazu}{\pagebreak[2]\noindent\leftskip 10pt\textbf{Idea d�kazu}\par\noindent\leftskip 10pt}{\odrovnej\par\bigskip}


\newenvironment{priklad}{\pagebreak[2]\noindent\textbf{P��klad}\par\noindent\leftskip 10pt}{\odrovnej\par\bigskip}
\newenvironment{prikladSK}{\pagebreak[2]\noindent\textbf{Pr�klad}\par\noindent\leftskip 10pt}{\odrovnej\par\bigskip}
\newenvironment{priklady}{\pagebreak[2]\noindent\textbf{P��klady}\par\noindent\leftskip 10pt}{\odrovnej\par\bigskip}
\newenvironment{prikladySK}{\pagebreak[2]\noindent\textbf{Pr�klady}\par\noindent\leftskip 10pt}{\odrovnej\par\bigskip}

\newenvironment{algoritmusN}[1]{\pagebreak[2]\noindent\textbf{Algoritmus~}\emph{(#1)}\par\noindent\leftskip 10pt}{\odrovnej\par\bigskip}
%obecne prostredie, ktore ma vyuzitie pri specialnych odstavcoch ako (uloha, algoritmus...) aby nevzniklo dalsich x prostredi
\newenvironment{obecne}[1]{\pagebreak[2]\noindent\textbf{#1}\par\noindent\leftskip 10pt}{\odrovnej\par\bigskip}


\newenvironment{penumerate}{
\begin{enumerate}
  \setlength{\itemsep}{1pt}
  \setlength{\parskip}{0pt}
  \setlength{\parsep}{0pt}
  %\setlength{\topsep}{200pt}
  \setlength{\partopsep}{200pt}
}{\end{enumerate}}

\def\pismenka{\numberedlistdepth=2} %pouzit, ked clovek chce opismenkovany zoznam...

\newenvironment{pitemize}{
\begin{itemize}
  \setlength{\itemsep}{1pt}
  \setlength{\parskip}{0pt}
  \setlength{\parsep}{0pt}
}{\end{itemize}}

%\definecolor{gris}{gray}{0.95}
\newcommand{\ramcek}[2]{\begin{center}\fcolorbox{white}{gris}{\parbox{#1}{#2}}\end{center}\par}
