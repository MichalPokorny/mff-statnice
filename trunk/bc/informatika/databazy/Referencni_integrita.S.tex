\subsection{Referenční integrita}

Jedná se o nástroj databázového stroje, který pomáhá udržovat
vztahy v relačně propojených databázových tabulkách.

Referenční integrita se definuje \emph{cizím klíčem}, a to pro dvojici tabulek,
nebo nad jednou tabulkou, která obsahuje na sobě závislá data (například
stromové struktury). Tabulka, v níž je pravidlo uvedeno, se nazývá podřízená
tabulka (používá se také anglický termín slave). Tabulka, jejíž jméno je v
omezení uvedeno je nadřízená tabulka (master). Pravidlo referenční integrity
vyžaduje, aby každý záznam použitý v podřízené tabulce existoval v nadřízené
tabulce. To znamená, že každý záznam musí v cizím klíči obsahovat hodnoty
odpovídající primárními klíči v nadřízené tabulce, nebo null.

To sebou nese dva důsledky:
\begin{pitemize}
\item při přidání záznamu do podřízené tabulky se kontroluje, zda stejná hodnota
klíče existuje v nadřízené tabulce – porušení pravidla vyvolá chybu,
\item při mazání nebo úpravě záznamů v nadřízené tabulce se kontroluje, zda
v podřízené tabulce není záznam se stejnou hodnotou klíče – porušení pravidla
může vyvolat chybu nebo úpravu dat v podřízené tabulky v souladu s definovanými
akcemi.
\end{pitemize}

\begin{obecne}{Příklad}
V databázi spolku přátel psů máme následující tabulky:
\begin{pitemize}
 \item \emph{osoby} se sloupci \emph{\underline{osoba-id}} a \emph{jméno},
 \item \emph{psi} se sloupci \emph{\underline{pes-id}}, \emph{majitel} a \emph{rasa}.
\end{pitemize}

Aby byla data v databázi korektní, je třeba, aby každý záznam psa měl uvedeného
platného majitele. Proto označíme v tabulce psi sloupec \emph{majitel} jako cizí
klíč, vztažený k sloupci (klíči) \emph{osoba-id} v tabulce osoby. Když je poté
přidán záznam pro psa, databáze bude vyžadovat, aby číslo v poli \emph{majitel}
nabývalo některé z existujících hodnot \emph{osoba-id} tabulky osoby. Zároveň
můžeme určit, zda se při smazání osoby smažou i záznamy všech psů, kterým je
osoba majitelem, nebo zda má pokus o smazání osoby vlastnící alespoň jednoho psa
selhat.
\end{obecne}

Existuje tedy několik \emph{referenčních akcí}, které mohou být vyvolány, když
dochází ke změně nebo mazání v závislých tabulkách:
\begin{description}
  \item[CASCADE] -- Pokud je smazán řádek v nadřízené tabulce, řádky z podřízené
  tabulky obsahující mazaný cizí klíč budou smazány také.

  \item[RESTRICT] -- Řádek v nadřízené tabulce nesmí být smazán ani změněn, pokud
  v podřízené tabulce existují závisející řádky. Nedojde ani k pokusu o změnu dat.
  
  \item[NO ACTION] -- Operace Update nebo Delete je spuštěna na nadřízenou tabulku
  a na konci je teprve vyhodnoceno, jestli nedošlo k porušení integrity. Rozdíl
  oproti akci RESTRICT je ten, že samotným dotazem, nebo například triggerem může
  být zařízeno, aby k porušení integrity nedošlo. Potom je operace normálně
  provedena.
  
  \item[SET NULL] -- Cizí klíč v podřízené tabulce je nastaven na \emph{null},
  pokud dojde ke změně či smazání v odpovídajícím řádku nadřízené tabulky.

  \item[SET DEFAULT] -- Téměř to samé jako SET NULL, hodnoty cizího klíče jsou
  nastaveny na defaultní hodnotu sloupce, pokud dojde ke změně či smazání
  odpovídajícího řádku v nadřízené tabulce.
\end{description}
