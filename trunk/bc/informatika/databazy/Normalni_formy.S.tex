\subsection{Normální formy}

Je zřejmé, že u složitějších aplikací není každé schéma vhodné. Například pokud
by ve schématu docházelo k redundanci dat. Mohla by potom nastat situace, že při
nějaké aktualizaci, bychom změnili pouze některé výskyty a tím bychom databázi
přivedli do nekonzistentního stavu. Jedním ze způsobů jak se těchto problému
vyvarovat je mít databázi v (první, druhé \dots) \textbf{normální formě}.

Normálních forem je několik. Čím větší normální formu schéma splňuje tím je
snadnější ho udržovat v konzistentním stavu a později rozšiřovat. Vyšší stupeň
normalizovanosti vede obvykle k většímu počtu tabulek a tedy k nutnosti
používání většího množství operací typu \emph{join}. To může zapříčinit snížení
výkonu. Díky tomu jsou vysoce normalizovaná schémata typicky používána v
databázových aplikací, které používají spoustu izolovaných transakcí (např.
automatický bankovní systém). Naproti tomu méně normalizovaná schémata jsou
nasazována v aplikacích, které obsahují data převážně pro čtení (např.
zpravodajské systémy).

\begin{definice}
Řekneme, že atribut \emph{B} je \textbf{funkčně závislý} na atributu \emph{A}
(značíme $A\rightarrow B$), jestliže pro každou hodnotu atributu \emph{A}
existuje právě jedna hodnota atributu \emph{B}.\\[5mm]
\textbf{Nadklíčem}, někdy též \textbf{superklíčem}, schématu $A$ rozumíme každou
podmnožinu množiny $A$ na níž $A$ funkčně závisí. Jinak řečeno nadklíč je množina
atributů, která jednoznačně určuje řádek tabulky.\\[5mm]
\textbf{Klíč}, nebo také \textbf{potenciální klíč}(candidate key), schématu $A$
je takový nadklíč schématu $A$, jehož žádná vlastní podmnožina není nadklíčem
$A$. Čili minimální nadklíč.\\[5mm]
Každý atribut, který je obsažen alespoň v jednom potenciálním klíči se nazývá
\textbf{klíčový}, ostatní atributy jsou \textbf{neklíčové}.
\end{definice}

\subsubsection*{První normální forma (1NF)}

\begin{definiceN}{1NF} Schéma je v \emph{první normální formě}, jestliže každý atribut
schématu je elementárního (jednoduchého) typu, je nestrukturovaný.
\end{definiceN}

Jiná definice říká, že by schéma mělo být reprezentací nějaké \emph{relace} a
neobsahovat \emph{opakující se skupiny (repeating groups)}. Jelikož ale význam
\emph{opakujících se skupin} není přesně stanoven, existují jisté spory ohledně
toho které schéma 1NF splňuje a které ne.

\subsubsection*{Druhá normální forma (2NF)}

\begin{definiceN}{2NF}
Schéma je v \emph{druhé normální formě}, jestliže je v první normální formě a
žádný neklíčový atribut není funkčně závislý na žádné podmnožině klíče.
\end{definiceN}

To znamená, že neklíčový atribut může závist pouze na celém klíči. Pokud by
závisel jen na jeho části, měli bychom tabulku rozdělit na dvě. Mějme například
tabulku zaměstnanců s atributy: jméno, schopnosti, adresa. Ve které dvojice
\{jméno, schopnosti\} je klíč, čili jednoznačně určuje záznam. Nechť adresa
závisí pouze na jméně. Potom tabulka není v 2NF.

\begin{poznamka} Všimněme si pokud je schéma v 1NF a zároveň všechny její
potenciální klíče sestávají pouze z jednoho atributu, můžeme rovnou říct, že
schéma splňuje 2NF. \end{poznamka}

\subsubsection*{Třetí normální forma (3NF)}

\begin{definiceN}{3NF}
Schéma je ve \emph{třetí normální formě}, jestliže je v 2NF a žádný neklíčový
atribut není tranzitivně závislý na žádném klíči.
\end{definiceN}

Tranzitivní závislost, je taková funkční závislost $X \rightarrow Y$, že $Y$
nezávisí na $X$ přímo, ale existuje nějaké $Z$ takové, že $X \rightarrow Z$ a $Z
\rightarrow Y$. Jinak řečeno neklíčové atributy musí na klíči záviset přímo a ne
přes nějaký jiný atribut.

Alternativní definice říká, že schéma je v 3NF právě tehdy, když pro každou
funkční závislost $X \rightarrow Y$ platí alespoň jedna z následujících podmínek:
\begin{pitemize}
\item závislost je triviální, tj. $X$ obsahuje $Y$, 
\item $X$ je nadklíč schématu,
\item $Y$ je klíčový atribut, tj. $Y$ je obsažen v nějakém potenciálním klíči.
\end{pitemize}

\subsubsection*{Boyce-Coddova normální forma(BCNF)}

\begin{definiceN}{BCNF}
Schéma je v Boyce-Coddově normální formě právě tehdy, když pro každou
netriviální funkční závislost $X \rightarrow Y$ platí, že $X$ je nadklíč
schématu.
\end{definiceN}

BCNF je o něco silnější než 3NF. Pokud se podíváme na alternativní definici 3NF, je
dobře vidět rozdíl. Vynecháme-li třetí podmínku, dostaneme definici BCNF. Jinak
řečeno žádný samostatný atribut nesmí záviset na ničem jiném než na nadklíči.

