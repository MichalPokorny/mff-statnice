\subsection{Skriptování a shelly}

Uplně přesně nevím co by tady mělo být, tak aspoň takovej přehled.

\subsubsection*{Skriptování}

Skriptovací jazyk je programovací jazyk, který je interpretován přímo jak je
zadáván z klávesnice. Nedochází tedy k tomu, jako u klasických programovacích
jazyků, že by byl program nejdříve přeložen do binární podoby a potom spuštěn.
Skript zůstává ve své původní podobě a je vyhodnocován příkaz po příkazu, tak
jak je zadáván. Skriptovací jazyky byly vytovořeny pro urychlení standardního
vývojového cyklu editace~-- kompilace~-- linkování~-- spuštění a také pro
možnost automatizovat některé úlohy.

Skripty mohou být také zkompilovány, ale protože napsat interpretr je jednodušší
než napsat kompilátor, tak jsou mnohem častěji interpretovány. Skriptovacích
jazyků je obrovské množství.

\paragraph{Skriptovací jazyky aplikací} -- Velká většina rozsáhlých aplikací
obsahuje svuj vlastní skriptovací jazyk, ušitý přesně na míru požadavkům
konkrétní aplikace. Například některé počítačové hry používají skripty pro popis
chování postav, které nehraje člověk.
\begin{pitemize}
\item Emacs Lisp,
\item Matlab,
\item QuakeC,
\item UnrealScript,
\item Vim scripting language.
\end{pitemize}

\paragraph{Skriptovací jazyky pro WEB} -- Důležitá součást rodiny skriptovacích
jazyků jsou jazyky používáné k tvorbě interaktivních webových aplikací.
\begin{pitemize}
\item ASP,
\item PHP,
\item JavaScript,
\item VBScript.
\end{pitemize}

\paragraph{Jazyky pro zpracování textu} -- Jedno z nejstarších použití
skriptovacích jazyků bylo automatické zpracování textových dat. Spoustu jich
bylo původně navrženo jako pomoc administrátorům při zpracování textových
konfiguračních souborů a později až dorostly do právoplatných skriptovacích
jazyků.
\begin{pitemize}
\item awk,
\item Sed,
\item XSLT.
\end{pitemize}

\paragraph{Obecné skriptovací jazyky} -- Některé jazyky jsou přímo určeny pro
nejširší použití a nejsou vázány na nějaké konkrétní použití. 
\begin{pitemize}
\item Lisp,
\item Perl,
\item Python,
\item Ruby,
\item Tcl.
\end{pitemize}

\paragraph{Job control languages} -- Hlavní skupina skriptovacích jazyků vznikla
za účelem spouštění a kontrolování běhu programů. Většinou bývají navázány na
nějaký operační systém, ale mohou fungovat i na různých architekturách.
\begin{pitemize}
\item AppleScript,
\item bash, csh, ksh \dots,
\item cmd.exe, command.com.
\end{pitemize}


\subsubsection*{Shelly}

Unix shell je tradiční uživatelský interface pro operační systém Unix, nebo
systémy na unixu založené. Uživatelé řídí práci počítače přímo psaním textových
příkazů pro shell. Pro OS Windows existuje obdoba zvaná \emph{command.com}, nebo
\emph{cmd.exe}.

V nejobecnějším významu termín \emph{shell} znamená jakýkoliv program, který
uživatel používá k zadávání příkazů. V OS Unix si uživatel může vybrat jaký
shell bude používat, proto jich bylo vyvinuto nepřeberné množství. Název shell
(skořápka, ulita, plášť \dots) proto, že "schovává" detaily pod ním ležícího
operačního systému za svůj interface.

Výraz shell také znamená nějaký konkrétní program, jako třeba \emph{Bourne
shell}, nebo \emph{Korn shell}. Bourne shell byl použit u prvopočátků operačního
systému Unix a stal se de facto standardem mezi shelly. Každý Unixový systém má
alespoň jeden shell s ním kompatibilní. Program Bourne shell je v Unixove
hierarchii uložen v \texttt{/bin/sh}. Na některých systémech, jako třeba BSD, je
\texttt{/bin/sh} přímo Bourne shell, nebo jeho ekvivalent, na linuxu je to
většinou link na kompatibilní, ale rozšířený a mnohem mocnější shell.

\paragraph{Bourne shell (sh)} -- původní Unixový shell, který napsal Steve
Bourne v Bell Labs. Chybí mu některé věci pro interaktivní práci (doplňování
příkazů, manipulace s historií, editace příkazové řádky \dots), ale již obsahuje
jednoduše použitelný jazyk pro psaní shell skriptů. Dnes se používají spíše
modernější shelly pro svou větší uživatelskou přátelskost.

\paragraph{C shell (csh)} -- shell používající syntaxi podobnou jazyku C. Je o
trochu šikovnější pro interaktivní používání (přidává aliasy a příkazovou
historii), ale zase o něco nešikovnější pro psaní skriptů.

\paragraph{TC shell (tcsh)} -- csh obohacený o doplňování příkazů, editaci
příkazové řádky a další vylepšení.

\paragraph{Bourne again shell (bash)} -- shell nově napsaný ve Free Software
Foundation v rámci GNU iniciativy. Obsahuje jazyk pro psaní skriptů použitý v
\emph{sh}, ale přidává spoustu užitečných funkcí pro interaktivní používání.

\paragraph{Korn shell (ksh)} -- rozšíření \emph{sh} a tedy s ním zpětně
kompatibilní. Velice mocný nástroj i pro interaktivní používání i pro
skriptování. Hlavní výhoda proti ostatním shellům je propracovanost jeho použití
jako programovacího jazyka.

\paragraph{Z shell (zsh)} -- moderní shell vzniklý rozšířením \emph{sh} a
přidáním velikého počtu vylepšení a užitečných věcí z \emph{bash, ksh, tcsh}.
Obsahuje například: programovatelné doplňování příkazů, sdílení historie příkazů
mezi všemi běžícími shelly, kontrola syntaxe, široce nastavitelné možnosti pro
vzhled a chování promptu\dots

\begin{obecne}{Základní shellové nástroje}
  \begin{pitemize}
    \item  cat, grep, head, tail, wc, tee
    \item cp, rm, mv
    \item ls
    \item cd, pwd, mkdir, rmdir
    \item echo
    \item more, less
    \item read
    \item sort, cut, tr
    \item find
    \item xargs
    \item sed - stream editor
    \item awk
    \item a desítky dalších \dots
  \end{pitemize}
\end{obecne}

\begin{obecne}{Skriptování v shellu}
  \begin{pitemize}
    \item  roury (pipe)
    \item standardní vstup, standardní výstup, chybový výstup
    \item proměnné, speciální proměnné \$X, uvozování, parametry skriptu
    \item proměnné prostředí
    \item subshelly
    \item podmínky, cykly
    \item funkce
    \item přesměrování~-- <, >, >>
    \item signály
    \item expanzní znaky, regulární výrazy 
  \end{pitemize}
\end{obecne}

