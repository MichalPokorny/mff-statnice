\newcommand{\nadpis}[1]{\pagebreak[2]\ramcek{\subsubsection*{#1}}}

\subsection{Hašování}

\begin{definiceN}{slovníkový problém} Dáno univerzum $U$, máme reprezentovat
$S \subseteq U$ a navrhnout algoritmy pro operace
\begin{pitemize}
\item MEMBER(x) - zjistí zda $x \in S$ a pokud ano nalezne kde,
\item INSERT(x) - pokud $x \notin S$, vloží $x$ do struktury reprezentující $S$,
\item DELETE(x) - když $x \in S$, smaže $x$ ze struktury reprezentující $S$.
\end{pitemize}
\end{definiceN}

Například pomocí pole můžeme tyto operace implementovat rychle, ale nevýhodou
je prostorová náročnost. Pro velké množiny je to někdy dokonce nemožné.
\emph{Hašování} se snaží zachovat rychlost operací a odstranit prostorovou
náročnost.

Podívejme se nyní na \emph{základní ideu} hašování. Mějme univerzum $U$ a
množinu $S \subseteq U$ takovou, že $|S| << |U|$. Dále mějme funkci
$h:U\rightarrow\{0,1,\dots,m-1\}$. Množinu $S$ potom reprezentujeme tabulkou
(polem) o velikosti $m$ tak, že prvek $x \in S$ je uložen na řádku $h(x)$.

\begin{definiceN}{Hašovací funkce, kolize} Funkci
$h:U\rightarrow\{0,1,\dots,m-1\}$ potom nazýváme \textbf{hašovací funkcí}.
Situaci $h(s)=h(t)$, pro $s \neq t; s,t \in S$  nazveme \textbf{kolize}.
\end{definiceN}

Jelikož mohutnost univerza $U$ je větší než velikost hašovací tabulky, nelze se
kolizím úplně vyhnout. Existuje spousta různých metod, jak kolize řešit.
Podívejme se tedy na některé podrobněji.

\begin{definice}
Ještě si zavedeme některé značení. Velikost $S$ (hašované množiny) označme
\textbf{n}, velikost tabulky (pole) označme \textbf{m}, a faktor naplnění
\textbf{$\alpha=\frac{n}{m}$}.
\end{definice}

\subsubsection*{Hašování se separovanými řetězci}

Použijeme pole velikosti $m$, jehož $i$-tá položka bude spojový seznam $S_i$
takový, že $s\in S_i \Leftrightarrow h(s)=i$, pro $s\in S$. Čili každý řádek
pole obsahuje spojový seznam všech (kolidujících) prvků, které jsou hašovány na
tento řádek. Seznamy nemusí být uspořádané, vznikají tak, jak jsou vkládány
jednotlivé prvky do struktury.

\begin{algoritmusN}{Hašování se separovanými řetězci}
\begin{pitemize}
\item \textbf{MEMBER} -- Spočteme hodnotu hašovací funkce $h(x)$, prohledáme
řetězec začínající na pozici $h(x)$ a zjistíme zda se prvek nachází, či
nenachází ve struktuře. Pokud se prvek v databázi nachází, tak musí nutně ležet
v tomto řetězci.
\item \textbf{INSERT} -- Zjistíme zda $x$ je v řetězci $h(x)$, pokud ne, přidáme
ho nakonec, v opačném případě neděláme nic.
\item \textbf{DELETE} -- Vyhledá $x$ v řetězci $h(x)$ a smaže ho. Pokud se tam
$x$ nenachází, neudělá nic.
\end{pitemize}
\end{algoritmusN}

\paragraph{Očekávaný počet testů} v neúspěšném případě je přibližně
$e^{-\alpha} + \alpha$ a při úspěšném vyhledávání přibližně
$1+\frac{\alpha}{2}$.

\subsubsection*{Hašování s uspořádanými řetězci}

Jak již je zřejmé z názvu je tato metoda téměř stejná jako předchozí. Jediný
rozdíl je, že jednotlivé seznamy jsou uspořádány vzestupně dle velikosti
prvků.

\begin{algoritmusN}{Hašování s uspořádanými řetězci}
Rozdíly jsou pouze pro operaci \textbf{MEMBER}, kde skončíme prohledávání, když
dojdeme na konec, nebo když nalezneme prvek, který je větší než hledaný a
operaci \textbf{INSERT}, které vkládá prvek na místo kde jsme ukončili
vyhledávání (před prvek, který ho ukončil).
\end{algoritmusN}

\paragraph{Očekávaný počet testů} v neúspěšném případě je přibližně roven
$e^{-\alpha}+1+\frac{\alpha}{2}-\frac{1}{\alpha}(1-e^{-\alpha})$ a v úspěšném
případě je přibližně $1+\frac{\alpha}{2}$.

Nevýhodou předchozích dvou metod je nerovnoměrné využití paměti. Zatímco
některé seznamy mohou být dlouhé, v některých není prvek žádný. Řešením je najít
způsob, jak kolidující prvky ukládat na jiné (prázdné) řádky tabulky. Potom je
ale nutné každý prvek tabulky rozšířit a položky pro práci s tabulkou. 

Čím použijeme sofistikovanější metodu ukládání dat do tabulky, tím více budeme
potřebovat položek pro práci s tabulkou a tedy vzroste paměťová náročnost. Naším
cílem je tedy najít rozumný kompromis mezi sofistikovaností (rychlostí)
strategie a její paměťovou náročností. Podívejme se na další algoritmy,
které se o to pokoušejí.

\subsubsection*{Hašování s přemísťováním}

Seznamy jsou tentokrát ukládány do tabulky a implementovány jako dvousměrné.
Potřebujeme tedy dvě položky pro práci s tabulkou: \emph{next} -- číslo řádku
obsahující další prvek seznamu a \emph{previous} -- číslo řádku obsahující
předchozí prvek seznamu. Když dojde ke kolizi, tj. chceme vložit prvek a jeho
místo je obsazené prvkem z jiného řetězce, pak tento prvek z jiného řetězce
přemístíme na jiný prázdný řádek v tabulce (proto hašování s přemísťováním).

\begin{algoritmusN}{Hašování s přemísťováním}
Algoritmus \textbf{MEMBER} funguje stejně jako u hašování se separovanými
řetězci, jen místo ukazatele na další prvek použije hodnotu \emph{next} z
tabulky. Při operaci \textbf{INSERT} vložíme prvek kam patří pokud je tam místo,
pokud již je místo obsazeno prvkem který tam patří, čili zde začíná seznam
kolidujících prvků (\emph{previous} = prázdné), pak postupujeme po položkách
\emph{next} až na konec seznamu, vložíme prvek na některý volný řádek
tabulky a vyplníme správně hodnoty \emph{next} a \emph{previous}. Pokud je místo
obsazeno prvkem z jiného seznamu (\emph{previous} $\neq$ prázdné), tak tento
prvek přemístíme na některý volný řádek, správně přepíšeme položky \emph{next} a
\emph{previous} v měněném seznamu a vkládaný prvek uložíme na jeho místo.
Operace \textbf{DELETE} je vcelku přímočará, jenom je třeba, pokud mažeme první
prvek seznamu na jeho místo přesunout ten druhý v pořadí (pokud existuje).
\end{algoritmusN}

\paragraph{Očekávaný počet testů} je v neúspěšném případě roven přibližně
$(1-\frac{1}{m})^n+\frac{n}{m}\approx e^{-\alpha}+\alpha$ a v úspěšném je stejný
jako pro hašování se separovanými řetězci a tedy $\frac{n-1}{2m}+1 \approx 1 +
\frac{1}{\alpha}$.

\subsubsection*{Hašování se dvěma ukazateli}

Hašování s přemísťováním má tu nevýhodu, že díky přemisťování prvků jsou operace
INSERT a DELETE časově náročné. Tato metoda tedy implementuje řetězce jako
jednosměrné seznamy, ale takové které nemusejí začínat na svém místě, tj.
řetězec $S_j$ obsahující prvky $s \in S$ takové, že $h(s)=j$, nemusí začínat na
$j$-tém řádku. Místo ukazatele na předchozí prvek tak do položek pro práci s
tabulkou přidáme ukazatel na místo, kde začíná řetězec příslušný danému řádku.
Položky pro práci s tabulkou tedy budou: \emph{next} -- číslo řádku tabulky kde
je další prvek seznamu, \emph{begin} -- číslo řádku tabulky obsahující první
prvek seznamu příslušného tomuto místu.

\begin{algoritmusN}{Hašování se dvěma ukazateli}
Položka \emph{begin} v $j$-tém řádku je vyplněna právě tehdy, když
reprezentovaná množina $S$ obsahuje prvek $s \in S$ takový, že $h(s)=j$.
Algoritmy jsou potom podobné těm u hašování s přemísťováním, ale přemísťování
prvků je nahrazeno odpovídajícími změnami v položce \emph{begin} daných řádků.
\end{algoritmusN}

Díky práci s položkami jsou operace INSERT a DELETE rychlejší než při hašování s
přemísťováním, ale začátek řetězce v jiném řádku tabulky přidá navíc jeden test,
což změní složitost operace MEMBER.

\paragraph{Očekávaný počet testů} v neúspěšném případě je přibližně
$1+\frac{\alpha^2}{2}+\alpha+e^{-\alpha}(2+\alpha)-2$ a při úspěšném vyhledávání
je roven $1+\frac{(n-1)(n-2)}{6m^2}+\frac{n-1}{2m} \approx 1 +
\frac{\alpha^2}{6}+\frac{\alpha}{2}$

\subsubsection*{Srůstající hašování}

Nyní se podíváme na několik verzí metody, která se nazývá srůstající hašování.
Budeme potřebovat jedinou položku pro práci s tabulkou a to ukazatel
jednosměrného spojového seznamu. Na rozdíl od předchozích metod zde nejsou
řetězce separované, v jednom řetězci mohou být prvky s různou hodnotou hašovací
funkce. Když máme přidat prvek $s$, tak ho zařadíme do řetězce, který se nachází
na $h(s)$-tém řádku tabulky. Řetězce tedy v této metodě \emph{srůstají}. Různé
verze této metody se liší tím, kam přidáváme nový prvek a podle práce s pamětí.
Dělí se na \emph{standardní srůstající hašování} bez pomocné paměti a na hašování
používající pomocnou paměť, kterému se říká jen \emph{srůstající hašování}.

Nejdříve se budeme věnovat metodám standardního srůstajícího hašování (bez
pomocné paměti):
\begin{pitemize}
\item \textbf{LISCH} -- late-insertion standard coalesced hashing -- vkládá se
za poslední prvek řetězce,
\item \textbf{EISCH} -- early-insertion standard coalesced hashing -- vkládá se
za první prvek řetězce.
\end{pitemize}

Přirozená efektivní operace DELETE pro standardní srůstající hašování není
známa. Na druhou stranu i primitivní algoritmy mají rozumnou očekávanou časovou
složitost.

Další otázka zní, proč používat metodu EISCH, když programy pro metodu LISCH
jsou jednodušší. Odpověď je na první pohled dost překvapující. Při úspěšném
vyhledávání je metoda EISCH rychlejší než metoda LISCH. Je to proto, že je o
něco pravděpodobnější, že se bude pracovat s novým prvkem. V neúspěšném případě
jsou samozřejmě obě metody stejné, neboť řetězce jsou u obou stejně dlouhé.

Metody srůstajícího hašování (s pomocnou pamětí) mají použitou paměť rozdělenou
na dvě části. Na tu přímo adresovatelnou hašovací funkcí a na pomocnou část.
Adresovací část má $m$ řádků, pokud hašovací funkce má hodnoty z oboru
$\{0,1,\dots,m-1\}$, v pomocné části jsou řádky ke kterým nemáme přístup přes
hašovací funkci. Když při přidávání nového prvku vznikne kolize, tak se nejprve
vybere volný řádek z pomocné části a teprve když je pomocné část zaplněna
použijí se k ukládání kolidujících prvků řádky z adresovatelné části tabulky.
Tato strategie oddaluje srůstání řetězců. Srůstající hašování se tedy, aspoň
dokud není zaplněna pomocná část tabulky, podobá hašování se separovanými
řetězci. Existují základní tři varianty:
\begin{pitemize}
\item \textbf{LICH} -- late-insertion coalesced hashing -- vkládá prvek na konec
řetězce,
\item \textbf{VICH} -- early-insertion coalesced hashing -- vkládá prvek na
řádek $h(x)$ pokud je prázdný a nebo hned za prvek na řádku $h(x)$,
\item \textbf{EICH} -- varied-insertion coalesced hashing -- vkládá se za
poslední prvek řetězce, který je ještě v pomocné části. Pokud v pomocné části
žádný není, vkládá se hned za prvek na pozici $h(x)$.
\end{pitemize}

Tyto metody nepodporují přirozené efektivní algoritmy pro operaci DELETE.

\subsubsection*{Hašování s lineárním přidáváním}

Následující metoda nepoužívá žádné položky pro práci s tabulkou to znamená, že
způsob nalezení dalšího řádku řetězce je zabudován přímo do metody. Metoda
funguje tak, že pokud chceme vložit prvek do tabulky a nastane kolize, najdeme
první následující volný řádek a tam prvek vložíme. Předpokládáme, že řádky jsou
číslovány modulo \emph{m}, čili vytvářejí cyklus délky \emph{m}.

Tato metoda sice využívá minimální velikost paměti, ale v tabulce vznikají
shluky obsazených řádků a proto je při velkém zaplnění pomalá. Navíc metoda
nepodporuje efektivní operaci DELETE.

\subsubsection*{Shrnutí}

Zde uvedeme pořadí metod hašování podle očekávaného počtu testů.

\begin{obecne}{Neúspěšné vyhledávání:}
\begin{penumerate}
\item Hašování s uspořádanými separovanými řetězci,
\item Hašování se separovanými řetězci = Hašování s přemísťováním,
\item Hašování se dvěma ukazateli,
\item VICH = LICH
\item EICH,
\item LISCH = EISCH,
\item Hašování s lineárním přidáváním.
\end{penumerate}
\end{obecne}

\begin{obecne}{Úspěšné vyhledávání}
\begin{penumerate}
\item H. s uspořádanými řetězci = H. se separovanými řetězci = H. s přemísťováním,
\item Hašování se dvěma ukazateli,
\item VICH,
\item LICH,
\item EICH,
\item EISCH,
\item LISCH,
\item Hašování s lineárním přidáváním.
\end{penumerate}
\end{obecne}

\begin{poznamka} Metody se separovanými řetězci a srůstající hašování používají
více paměti. Metoda s přemísťováním vyžaduje více času -- na přemístění prvku.
Otázka která z metod je nejlepší není proto jednoznačně rozhodnutelná a je nutné
pečlivě zvážit všechny okolnosti nasazení metody a všechny naše požadavky na ní,
než se rozhodneme, kterou použijeme.
\end{poznamka}

\subsubsection*{Univerzální hašování}

Pro dobré fungování hašování potřebujeme mimo jiné, aby vstupní data byla
rovnoměrně rozdělena a toho někdy není možné dosáhnout. Odstranit tento
nedostatek se pokouší metoda \emph{univerzální hašování}. Základní idea této
metody je taková, že máme množinu \emph{H} hašovacích funkcí z univerza do
tabulky velikosti \emph{m} takových, že pro $S \subseteq U$, $|S| \leq m$ se
většina funkcí chová dobře v tom smyslu, že má malý počet kolizí. Hašovací
funkci potom zvolíme z množiny \emph{H} (takovou s rovnoměrným rozdělením).
Jelikož funkci volíme my, můžeme požadavek rovnoměrného rozdělení zajistit.

\subsubsection*{Perfektní hašování}

Jiná možnost jak vyřešit kolize, je najít takzvanou \emph{perfektní hašovací
funkci}, tj. takovou které nepřipouští kolize. Nevýhoda této metody je, že nelze
dost dobře implementovat operaci INSERT, proto se dá prakticky použít pouze tam,
kde předpokládáme hodně operací MEMBER a jen velmi málo operací INSERT. Kolize
se potom dají řešit třeba malou pomocnou tabulkou, kam se ukládají kolidující
data. 

Pro rozumné fungování metody je nutné, aby hašovací funkce byla rychle
spočitatelná a aby její zadání nevyžadovat mnoho paměti, nejvýhodnější je
analytické zadání. 

Naopak jedna z výhod je, že nalezení perfektní hašovací funkce, může trvat
dlouho, neboť ho provádíme pouze jednou na začátku algoritmu. 

\subsubsection*{Externí hašování}

Externí hašování řeší trochu jiný problém, než výše popsané metody. Chceme
uložit data na externí médium a protože přístup k externím médiím je o několik
řádů pomalejší, než práce v interní paměti, bude naším cílem minimalizovat počet
přístupů do ní. Externí paměť bývá rozdělena na stránky a ty většinou načítáme
do interní paměti celé. Tato operace je však velice pomalá. Problémem externího
hašování je tedy nalézt datovou strukturu pro uložení dat na vnější paměti a
algoritmy pro operace INSERT, DELETE a MEMBER, tak abychom použili co nejmenší
počet komunikací mezi vnější a vnitřní pamětí.

Metod externího hašování je opět mnoho. Některé používají pomocnou datovou
strukturu v interní paměti, kterou často nazýváme adresář. Pokud metody nemají
žádnou takovou pomocnou strukturu neobejdou se obvykle bez oblasti přetečení.
Některé známější metody vnějšího hašování jsou například: \uv{Litwinovo lineární
hašování}, \uv{Faginovo rozšiřitelné hašování}, \uv{Cormackovo perfektní
hašování} nebo \uv{Perfektní hašování Larsona a Kajli}. 
% to neni preklep, hasovani je opravdu od pana KAJLI z nakyho duvodu se to uci
% blbe. viz http://portal.acm.org/citation.cfm?id=358193&coll=portal&dl=ACM
% ajs
