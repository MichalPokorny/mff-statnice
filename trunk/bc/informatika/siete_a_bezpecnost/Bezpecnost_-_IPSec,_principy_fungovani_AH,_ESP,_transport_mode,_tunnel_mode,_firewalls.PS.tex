\subsection{Bezpečnost -- IPSec, principy fungování AH, ESP, transport mode, tunnel mode, firewalls}

\subsubsection*{IPSec}

\begin{pitemize}
	\item Není to pouze jeden protokol ale soustava vzájemně provázaných opatření a dílčích protokolů pro zabezpečení komunikace pomocí IP protokolu, funguje na síťové vrstvě -- není závislý na protokolech vyšších vrstev jako je TCP a UDP (např. SSL protokol pracuje na transportní vrstvě)
	\item Podporováno jak v IPv4 (podpora nepovinná) i v IPv6 (podpora povinná)
	\item Zajišťuje důvěrnost (šifruje přenášená data) a integritu (data nejsou při přenosu změněna)
	\item několik desítek RFC dokumentů
	\item autentifikace -- ověření původu dat (odesílatele)
	\item kryptování -- šifrování komunikace (mimo IP hlavičky)
	\item může být implementováno na bráně (security gateway, lokální síť je považována za bezpečnou) nebo na koncových zařízení 

	\item \textbf{SA (Security Association)}
	\begin{pitemize}
		\item point-to-point bezpečnostní spoj (návrh uvažuje i o jiných variantách)
		\item pro každý směr a každý prototokol nutné mít vlastní SA spoj
	\end{pitemize}
\end{pitemize}

IPsec módy:
\begin{pitemize}
	\item \textbf{transport mode}
	\begin{pitemize}
		\item IP hlavička nechráněná (jeden z důvodů je užívání systému NAT), tělo paketu šifrováno (data vyšších protokolů)
		\item použitelné jen na koncových stanicích 
	\end{pitemize}
	\item \textbf{tunnel mode}
	\begin{pitemize}
		\item pakety jsou celé (včetně hlavičky) zašifrovány a vloženy do dalšího paketu, na druhé straně rozbaleny
		\item povinné pro security gateways, volitelné pro koncové stanice
		\item ve vnější IP hlavičce se jako příjemce uvádí security gateway na hranici cílové sítě 
	\end{pitemize}
\end{pitemize}

IPsec protokoly:
\begin{pitemize}
	\item \textbf{AH (Authentication Header)}
	\begin{pitemize}
		\item komunikující strany se dohodnou na klíči
		\item k datům se připojuje hash
		\item chrání také před replay attack
		\item provádí autentizaci a kontrolu změny dat, neprovádí šifrování 
	\end{pitemize}
	\item \textbf{ESP (Encapsulating Security Payload)}
	\begin{pitemize}
		\item provádí autentizaci a také šifruje obsah
		\item pro šifrování používá 3DES, Blowfish aj. (původně DES, již není považováno za bezpečné) 
	\end{pitemize}
\end{pitemize}

Dohoda klíčů:
\begin{pitemize}
	\item před použitím protokolu AH či ESP si musí strany dohodnout klíče
	\item manuální konfigurace
	\item automatická konfigurace -- IKE (Internet Key Exchange) protokol 
\end{pitemize}

\subsubsection*{Firewally}
\begin{pitemize}
	\item sledování a filtrování komunikace na síti
	\begin{pitemize}
		\item blokování -- zabraňuje neoprávněnému přístupu
		\item prostupnost -- propouštění povoleného toku 
	\end{pitemize}
	\item paketové filtry -- např. na routeru
	\item stavový firewall (stateful) -- sleduje vztahy mezi pakety, ohlíží se na historii
	\item na různých vrstvách
	\begin{pitemize}
		\item síťová -- pouze dle zdrojových a cílových adres a protokolu
		\item transportní -- také podle portů
		\item aplikační -- dle obsahu (dat) 
	\end{pitemize}
	\item demilitarizovaná zóna (DMZ):
	\begin{pitemize}
		\item jiné řešení bezpečnosti
		\item přístup ven pouze přes specializovaná zařízení (proxy, brány), nelze přímo -- platí pro oba směry 
	\end{pitemize}
\end{pitemize}
