\clearpage %&latex
\documentclass[a4paper]{article}

\frenchspacing

\usepackage[cp1250]{inputenc}
\usepackage[czech]{babel}

\usepackage{a4wide}
\usepackage{amsmath, amsthm, amssymb, amsfonts}
\usepackage[mathcal]{eucal}
\usepackage{graphicx}
\usepackage{url}
\usepackage{color}
\usepackage{wrapfig}
\usepackage{capt-of}
\usepackage{float}



% sirka a vyska textu nastavena jako papir, vsechny okraje vynulovany a pridano 20pt na kazdou stranu
% horizontalni rozmery
\setlength{\textwidth}{\paperwidth}
\addtolength{\textwidth}{-40pt}
\addtolength{\hoffset}{-1in}
\addtolength{\hoffset}{20pt}
\setlength{\oddsidemargin}{0in}
\setlength{\marginparsep}{0in}
% vertikalni rozmery
\setlength{\textheight}{\paperheight}
\addtolength{\textheight}{-60pt}
\addtolength{\voffset}{-1in}
\addtolength{\voffset}{20pt}
\setlength{\topmargin}{0in}
\setlength{\headheight}{0in}
\setlength{\headsep}{0in}


%Obrazek na miste
%pouziti
%%\obrazeknahore{adresa}{popisek}{label}
\long\def\obrazeknahore#1#2#3 {

\begin{figure}[t]
    \centering
    \includegraphics[width=0.8\textwidth]{#1}
    
    \caption{#2}
    \label{#3}
    
\end{figure}

}


%==========================================
%PEKELNA MAKRA NA ZAROVNANI OBRAZKU DOPRAVA

\makeatletter


%tohle je makro, ktere mi dovoluje obtekani i u kratkych environmentu
%ABSOLUTNE nechapu, jak to funguje, ale funguje to
%viz http://tex.stackexchange.com/questions/26078/ 
\def\odrovnej{\@@par
\ifnum\@@parshape=\z@ \let\WF@pspars\@empty \fi % reset `parshape'
\global\advance\c@WF@wrappedlines-\prevgraf \prevgraf\z@
\ifnum\c@WF@wrappedlines<\tw@ \WF@finale \fi}

\makeatother



%---
%makro, co da obrazek doprava a ostatni text ho obteka
%(bez toho predchazejiciho makra to ale poradne nebeha)
%pouziti:
%\obrazekvpravo{adresa}{popisek}{label}{procento sirky}
\long\def\obrazekvpravo#1#2#3#4{

\setlength\intextsep{-20pt}

    \begin{wrapfigure}{r}{#4\textwidth}
      \begin{center}
          \vspace{-10pt}
          
        \includegraphics[width=#4\textwidth]{#1}
        \vspace{-10pt}
        
      \end{center}
      
      \caption{#2}
      \label{#3}
      
      
    \end{wrapfigure}

\setlength\intextsep{0pt}

    
}




%---
%makro pro pripady, kdy wrapfigure neco mrsi
%je to docela pekelne
%je nutne mu dat jak text vpravo, tak text vlevo
%a nevim, jestli bude 100% fungovat, ale doufam, ze jo

%pouziti:
%\obrazekvpravominipage{adresa}{popisek}{label}{procento sirky}{1 - procento sirky}{text vlevo}
\long\def\obrazekvpravominipage#1#2#3#4#5#6{

\noindent\begin{minipage}{#5\linewidth}
\vspace{0pt}
#6
\end{minipage}
\hspace{0.5cm}
\noindent\begin{minipage}{#4\linewidth}
\vspace{0pt}
\centering
\includegraphics[width=0.9\textwidth]{#1}
\captionof{figure}{#2}
\label{#3}
\end{minipage}

}

%KONEC PEKELNYCH MAKER
%=====================

% makra pro poznamku u vyrokove a predikatove logiky
\def\vl{ -- ve v�rokov� logice}
\def\pl{ -- v predik�tov� logice}


%Vacsina prostredi je dvojjazicne. V pripade, ze znenie napr pozorovania je pisane po slovensky, malo by byt po slovensky aj oznacenie.

\newenvironment{pozadavky}{\pagebreak[2]\noindent\textbf{Po�adavky}\par\noindent\leftskip 10pt}{\odrovnej\par\bigskip}
\newenvironment{poziadavky}{\pagebreak[2]\noindent\textbf{Po�iadavky}\par\noindent\leftskip 10pt}{\odrovnej\par\bigskip}


\newenvironment{definiceSkull}{\pagebreak[2]\noindent\textbf{$\bigstar$ Definice}\par\noindent\leftskip 10pt}{\odrovnej\par\bigskip}
\newenvironment{definiceNSkull}[1]{\pagebreak[2]\noindent\textbf{$\bigstar$ Definice~}\emph{(#1)}\par\noindent\leftskip 10pt}{\odrovnej\par\bigskip}

\newenvironment{definice}{\pagebreak[2]\noindent\textbf{Definice}\par\noindent\leftskip 10pt}{\odrovnej\par\bigskip}
\newenvironment{definiceN}[1]{\pagebreak[2]\noindent\textbf{Definice~}\emph{(#1)}\par\noindent\leftskip 10pt}{\odrovnej\par\bigskip}
\newenvironment{definicia}{\pagebreak[2]\noindent\textbf{Defin�cia}\par \noindent\leftskip 10pt}{\odrovnej\par\bigskip}
\newenvironment{definiciaN}[1]{\pagebreak[2]\noindent\textbf{Defin�cia~}\emph{(#1)}\par\noindent\leftskip 10pt}{\odrovnej\par\bigskip}

\newenvironment{vetaSkull}{\pagebreak[2]\noindent\textbf{$\bigstar$ V�ta}\par\noindent\leftskip 10pt}{\odrovnej\par\bigskip}
\newenvironment{vetaNSkull}[1]{\pagebreak[2]\noindent\textbf{$\bigstar$ V�ta~}\emph{(#1)}\par\noindent\leftskip 10pt}{\odrovnej\par\bigskip}

\newenvironment{pozorovani}{\pagebreak[2]\noindent\textbf{Pozorov�n�}\par\noindent\leftskip 10pt}{\odrovnej\par\bigskip}
\newenvironment{pozorovanie}{\pagebreak[2]\noindent\textbf{Pozorovanie}\par\noindent\leftskip 10pt}{\odrovnej\par\bigskip}
\newenvironment{poznamka}{\pagebreak[2]\noindent\textbf{Pozn�mka}\par\noindent\leftskip 10pt}{\odrovnej\par\bigskip}
\newenvironment{poznamkaN}[1]{\pagebreak[2]\noindent\textbf{Pozn�mka~}\emph{(#1)}\par\noindent\leftskip 10pt}{\odrovnej\par\bigskip}
\newenvironment{lemma}{\pagebreak[2]\noindent\textbf{Lemma}\par\noindent\leftskip 10pt}{\odrovnej\par\bigskip}
\newenvironment{lemmaN}[1]{\pagebreak[2]\noindent\textbf{Lemma~}\emph{(#1)}\par\noindent\leftskip 10pt}{\odrovnej\par\bigskip}
\newenvironment{veta}{\pagebreak[2]\noindent\textbf{V�ta}\par\noindent\leftskip 10pt}{\odrovnej\par\bigskip}
\newenvironment{vetaN}[1]{\pagebreak[2]\noindent\textbf{V�ta~}\emph{(#1)}\par\noindent\leftskip 10pt}{\odrovnej\par\bigskip}
\newenvironment{vetaSK}{\pagebreak[2]\noindent\textbf{Veta}\par\noindent\leftskip 10pt}{\odrovnej\par\bigskip}
\newenvironment{vetaSKN}[1]{\pagebreak[2]\noindent\textbf{Veta~}\emph{(#1)}\par\noindent\leftskip 10pt}{\odrovnej\par\bigskip}

\newenvironment{dusledek}{\pagebreak[2]\noindent\textbf{D�sledek}\par\noindent\leftskip 10pt}{\odrovnej\par\bigskip}
\newenvironment{dosledok}{\pagebreak[2]\noindent\textbf{D�sledok}\par\noindent\leftskip 10pt}{\odrovnej\par\bigskip}

\newenvironment{dokaz}{\pagebreak[2]\noindent\leftskip 10pt\textbf{D�kaz}\par\noindent\leftskip 10pt}{\odrovnej\par\bigskip}
\newenvironment{dukaz}{\pagebreak[2]\noindent\leftskip 10pt\textbf{D�kaz}\par\noindent\leftskip 10pt}{\odrovnej\par\bigskip}

\newenvironment{ideadukazu}{\pagebreak[2]\noindent\leftskip 10pt\textbf{Idea d�kazu}\par\noindent\leftskip 10pt}{\odrovnej\par\bigskip}


\newenvironment{priklad}{\pagebreak[2]\noindent\textbf{P��klad}\par\noindent\leftskip 10pt}{\odrovnej\par\bigskip}
\newenvironment{prikladN}[1]{\pagebreak[2]\noindent\textbf{P��klad~}\emph{(#1)}\par\noindent\leftskip 10pt}{\odrovnej\par\bigskip}

\newenvironment{prikladSK}{\pagebreak[2]\noindent\textbf{Pr�klad}\par\noindent\leftskip 10pt}{\odrovnej\par\bigskip}
\newenvironment{priklady}{\pagebreak[2]\noindent\textbf{P��klady}\par\noindent\leftskip 10pt}{\odrovnej\par\bigskip}
\newenvironment{prikladySK}{\pagebreak[2]\noindent\textbf{Pr�klady}\par\noindent\leftskip 10pt}{\odrovnej\par\bigskip}

\newenvironment{algoritmusN}[1]{\pagebreak[2]\noindent\textbf{Algoritmus~}\emph{(#1)}\par\noindent\leftskip 10pt}{\odrovnej\par\bigskip}
%obecne prostredie, ktore ma vyuzitie pri specialnych odstavcoch ako (uloha, algoritmus...) aby nevzniklo dalsich x prostredi
\newenvironment{obecne}[1]{\pagebreak[2]\noindent\textbf{#1}\par\noindent\leftskip 10pt}{\odrovnej\par\bigskip}

\newenvironment{report}{\pagebreak[2]\noindent\textbf{Report}\em\par\noindent\leftskip 10pt}{\par\bigskip}

%\newenvironment{reportN}[1]{\pagebreak[2]\noindent\textbf{Report~}\emph{(#1)}\emph\par\noindent\leftskip 10pt}{\odrovnej\par\bigskip}
\newenvironment{reportN}[1]{\pagebreak[2]\noindent\textbf{Report~}\emph{(#1)}\em\par\noindent\leftskip 10pt}{\odrovnej\par\bigskip}

\newenvironment{penumerate}{
\begin{enumerate}
  \setlength{\itemsep}{1pt}
  \setlength{\parskip}{0pt}
  \setlength{\parsep}{0pt}
  %\setlength{\topsep}{200pt}
  \setlength{\partopsep}{200pt}
}{\end{enumerate}}

\def\pismenka{\numberedlistdepth=2} %pouzit, ked clovek chce opismenkovany zoznam...

\newenvironment{pitemize}{
\begin{itemize}
  \setlength{\itemsep}{1pt}
  \setlength{\parskip}{0pt}
  \setlength{\parsep}{0pt}
}{\end{itemize}}

%\definecolor{gris}{gray}{0.95}
\newcommand{\ramcek}[2]{\begin{center}\fcolorbox{white}{gris}{\parbox{#1}{#2}}\end{center}\par}
 \clearpage
\title{\LARGE Ucební texty k státní bakalárské zkoušce \\ Obecná informatika \\ Logika}
\begin{document}
\maketitle
\newpage
\setcounter{section}{0}
\section{Logika}
\begin{pozadavky}
\begin{pitemize}
\item Jazyk, formule, sémantika, tautologie.
\item Rozhodnutelnost, splnitelnost, pravdivost, dokazatelnost.
\item Vety o kompaktnosti a úplnosti výrokové a predikátové logiky.
\item Normální tvary výrokových formulí, prenexní tvary formulí predikátové logiky.
\end{pitemize}
\end{pozadavky}
\def\c#1{\mathcal{#1}}


\subsection{Jazyk, formule, s�mantika, tautologie}

\begin{obecne}{logika prvn�ho ��du}
V logice pracujeme s formulemi a termy, co� jsou slova (�et�zce znak� dan� abecedy) form�ln�ho jazyka. Jazyk prvn�ho ��du m��e zahrnovat:
\begin{pitemize}
    \item neomezen� mnoho symbol� pro prom�nn� $x_1,x_2,\dots $
    \item symboly pro logick� spojky ($\neg,\vee,\&,\rightarrow,\leftrightarrow$)
    \item symboly pro kvantifik�tory ($\forall$ obecn�, $\exists$ existen�n�)
    \item funk�n� symboly $f_1,f_2\dots $ s aritou $n \geq 0$ (nap�. $+, -, 1, *$)
    \item predik�tov� symboly $p_1,p_2\dots $ s aritou $n \geq 1$ (nap�. $\geq, =, \approx, \in, \subset $)
    \item m��e (ale nemus�) obsahovat bin�rn� predik�t \uv{$=$}, kter� pak se pak ale mus� chovat jako rovnost, tj. spl�ovat ur�it� axiomy.
\end{pitemize}
Prom�nn�, logick� spojky, kvantifik�tory a \uv{$=$} jsou \emph{logick� symboly}, ostatn� symboly se naz�vaj� \emph{speci�ln�}, jeliko� ur�uj� specifika jazyka a t�m vymezuj� oblast, kterou jazyk popisuje. V�rokov� a predik�tov� logika se �ad� mezi logiky prvn�ho ��du. Ty pracuj� jen s jazyky prvn�ho ��du, kter� maj� pouze jeden typ prom�nn�ch (pro prvky zvan� \emph{individua}). Jazyky vy���ch ��d� maj� krom� prom�nn�ch pro individua tak� dal�� typy prom�nn�ch (pro p�irozen� ��sla, funkce, relace, mno�iny a dal�� typy objekt�).

Ka�d� \emph{form�ln� syst�m} logiky prvn�ho ��du obsahuje:
\begin{pitemize}
    \item jazyk
    \item axiomy
    \item odvozovac� pravidla (pomoc� nich� tvo��me d�kazy a odvozujeme v�ty).
\end{pitemize}
Ze symbol� jazyka tvo��me dva druhy slov:
\begin{pitemize}
    \item \emph{termy} popisuj� individua (objekty) vznikl� z uveden�ch operac�
    \item \emph{formule} vyjad�uj� tvrzen� o objektech.
\end{pitemize}
\end{obecne}

\begin{definiceN}{jazyk v�rokov� logiky}
Jazyk $L_P$ v�rokov� logiky nad mno�inou $P$ obsahuje prvky mno�iny $P$ zvan� \emph{prvotn� formule} nebo \emph{v�rokov� prom�nn�} (typicky jde o slova n�jak�ho form�ln�ho jazyka nap�. $x, y, ABC, nula$), symboly pro \emph{logick� spojky} ($\neg,\vee,\&,\rightarrow,\leftrightarrow$) a pomocn� symboly (z�vorky).
\end{definiceN}

\begin{definiceN}{jazyk predik�tov� logiky}
Jazyk predik�tov� logiky obsahuje symboly pro prom�nn�, \emph{predik�tov�} a \emph{funk�n�} symboly, symboly pro \emph{logick� spojky}, symboly pro \emph{kvantifik�tory} a \emph{pomocn�} symboly (z�vorky).
\end{definiceN}

\begin{definiceN}{redukce jazyka}
Pro zmen�en� mno�iny axiom� je vhodn� redukovat po�et logick�ch spojek, se kter�mi pracujeme, na n�kolik z�kladn�ch a ostatn� vn�mat jako odvozen�. Je mo�no zvolit negaci a implikaci jako spojky z�kladn� a v predik�tov� logice obecn� kvantifik�tor. Zkratky pak vypadaj� jako $(A \& B)$ za formuli $\neg(A \to \neg B)$, $(A \vee B)$ odpov�d� $(\neg A \to B)$ a nakonec $(A \leftrightarrow B)$ redukujeme na $((A \to B)\&(B \to A))$.
\end{definiceN}

\begin{definiceN}{formule v�rokov� logiky}
Pro jazyk v�rokov� logiky jsou n�sleduj�c� v�razy formule:
\begin{penumerate}
    \item ka�d� v�rokov� prom�nn� $p \in P$
    \item pro formule $A,B$ i v�razy $\neg A$, $(A\vee B)$, $(A\& B)$, $(A\rightarrow B)$,
	$A\leftrightarrow B$
    \item ka�d� v�raz vzniknuv�� kone�n�m u�it�m pravidel 1. a 2.
\end{penumerate}
Tedy v�echny formule jsou kone�n� slova.
\end{definiceN}

\begin{definiceN}{term\pl}
V predik�tov� logice je \emph{term}:
\begin{penumerate}
    \item ka�d� prom�nn�
    \item v�raz $f(t_1,\dots,t_n)$ pro $n$-�rn� funk�n� symbol $f$ a termy $t_1,\dots,t_n$
    \item ka�d� v�raz vzniknuv�� kone�n�m u�it�m pravidel 1. a 2.
\end{penumerate}
Podslovo termu, kter� je samo o sob� term, se naz�v� \emph{podterm}.
\end{definiceN}

\begin{definiceN}{formule predik�tov� logiky}
V predik�tov� logice je formule ka�d� v�raz tvaru $p(t_1,\dots,t_n)$ pro $p$ predik�tov� symbol a $t_1,\dots,t_n$ termy. Stejn� jako ve v�rokov� logice je formule i (kone�n�) spojen� jednodu���ch formul� log. spojkami.
Formule jsou nav�c i v�razy $(\exists x)A$ a $(\forall x)A$ pro formuli $A$ a samoz�ejm� cokoliv, co vznikne kone�n�m u�it�m t�chto pravidel. Podslovo formule, kter� je samo o sob� formule, se naz�v� \emph{podformule}.
\end{definiceN}

\begin{definiceN}{voln� a v�zan� prom�nn�}
V�skyt prom�nn� $x$ ve formuli je \emph{v�zan�}, je-li tato sou��st� n�jak� podformule tvaru $(\exists x)A$ nebo $(\forall x)A$. V opa�n�m p��pad� je \emph{voln�}. Formule je \emph{otev�en�}, pokud neobsahuje v�zanou prom�nnou, je \emph{uzav�en�}, kdy� neobsahuje volnou prom�nnou. Prom�nn� m��e b�t v t�e formuli voln� i v�zan� (nap�. $(x=z)\rightarrow(\exists x)(x=z)$).
\end{definiceN}

\begin{definiceN}{pravdivostn� ohodnocen�\vl}
\emph{S�mantika} zkoum� pravdivost formul�. V�rokov� prom�nn� samotn� neanalyzujeme -- jejich hodnoty m�me d�ny u� z vn�j�ku, m�me pro n� \emph{mno�inu pravdivostn�ch hodnot} ($\lbrace 0,1\rbrace$).

\emph{Pravdivostn� ohodnocen�} $e: P \to \lbrace 0,1\rbrace$ je zobrazen�, kter� ka�d� v�rokov� prom�nn� p�i�ad� pr�v� jednu hodnotu z mno�iny pravdivostn�ch hodnot. Je-li zn�mo ohodnocen� prom�nn�ch, lze ur�it \emph{pravdivostn� hodnotu} $\overline{v}$ pro ka�dou formuli (p�i dan�m ohodnocen�) -- indukc� podle jej� slo�itosti, podle tabulek pro logick� spojky. 
\end{definiceN}


\begin{definiceN}{realizace jazyka, term� a ohodnocen� prom�nn�ch\pl}
\emph{Realizace jazyka} nebo t� \emph{interpretace jazyka} je definov�na mno�inovou strukturou $\c{M}$, kter� ke ka�d�mu symbolu jazyka a mno�in� prom�nn�ch p�i�ad� n�jakou mno�inu individu�. Popisuje \uv{hodnoty} v�ech funk�n�ch a predik�tov�ch symbol�. $\c{M}$ obsahuje:
\begin{pitemize}
    \item nepr�zdnou mno�inu individu� $M$.
    \item zobrazen� $f_M:M^n\to M$ pro ka�d� $n$-�rn� funk�n� symbol $f$
    \item relaci $p_M\subset M^n$ pro ka�d� $n$-�rn� predik�t $p$
\end{pitemize}

Realizace term� se uva�uje pro dan� jazyk $L$ a jeho realizaci $\c{M}$. \emph{Ohodnocen� prom�nn�ch} je zobrazen� $e:X\to M$ (kde $X$ je mno�ina prom�nn�ch). \emph{Realizace termu} $t$ p�i ohodnocen� $e$ (zna��me $t[e]$) se definuje n�sledovn�:
\begin{pitemize}
    \item $t[e]=e(x)$ je-li $t$ prom�nn� $x$
    \item $t[e]=f_M(t_1[e],\dots,t_n[e])$ pro term $t$ tvaru $f(t_1,\dots,t_n)$.
\end{pitemize}
Ohodnocen� z�vis� na zvolen�m $\c{M}$, realizace term� p�i dan�m ohodnocen� pak jen na kone�n� mnoha hodnot�ch z n�j. Pokud jsou $x_1,\dots,x_n$ v�echny prom�nn� termu $t$ a $e,e'$ dv� ohodnocen� tak, �e $\forall i\in\lbrace 1,\dots,n\rbrace$ plat� $e(x_i)=e'(x_i)$, pak $t[e]=t[e']$.
\end{definiceN}

\begin{definiceN}{pozm�n�n� ohodnocen�\pl}
\emph{Pozm�n�n� ohodnocen�} $e(x/m)$ je ohodnocen�, ve kter�m jsme zm�nili hodnotu jedn� prom�nn�. Form�ln� pro ohodnocen� $e$, prom�nnou $x$ a individuum $m \in M$ je definov�no: $$e(x/m)(y)=
    \begin{cases}
	m &\text{(je-li $y$ prom�nn� $x$, $y \equiv x$)} \\
	e(y) &\text{(jinak, )}
    \end{cases}$$
\end{definiceN}

\begin{definiceN}{tautologie $\models$\vl}
Formule je \emph{tautologie}, jestli�e je pravdiv� p�i libovoln�m ohodnocen� prom�nn�ch ($\models A$).
\end{definiceN}

\begin{definiceN}{teorie}
Mno�in� formul� ��k�me \emph{teorie}.
\end{definiceN}

\begin{definiceN}{pravdiv� formule\vl}
Formule v�rokov� logiky $A$ je \emph{pravdiv� p�i ohodnocen� $e$}, je-li $\bar{e}(A) = 1$ (kde $\bar{e}$ je definov�no z ohodnocen� prvotn�ch formul� $e$ induktivn� podle tabulek pro logick� spojky). V opa�n�m p��pad� je formule \emph{nepravdiv�}.
\end{definiceN}

\begin{definiceN}{model, tautologick� d�sledek $T\models$ \vl}
\emph{Model} n�jak� teorie ve v�rokov� logice je takov� ohodnocen� prom�nn�ch, �e ka�d� formule z t�to teorie je pravdiv�. Teorie $U$ je \emph{tautologick� d�sledek} teorie $T$, jestli�e ka�d� model $T$ je tak� modelem $U$ ($T\models U$).
\end{definiceN}
\def\c#1{\mathcal{#1}}
\def\Nat{\mathbb{N}}


\subsection{Rozhodnutelnost, splnitelnost, pravdivost a dokazatelnost}

\begin{obecne}{}
Z t�chto t�mat se rozhodnutelnosti budeme v�novat a� jako posledn�, proto�e k vysloven� n�kter�ch v�t budeme pot�ebovat pojmy definovan� v ��stech o splnitelnosti, pravdivosti a dokazatelnosti.
\end{obecne}

\begin{definiceN}{Form�ln� syst�m v�rokov� logiky}
Pracujeme s redukovan�m jazykem (jen s log. spojkami $\neg,\rightarrow$). Form�ln� syst�m v�rokob� logiky obsahuje:
\begin{penumerate}
    \item \emph{jazyk $L_P$} v�rokov� logiky nad mno�inou prvotn�ch formul� $P$,
    \item \emph{sch�mata axiom� v�rokov� logiky}, ze kter�ch pro libovoln� formule $A, B, C$ jazyka $L_P$ vznikaj� axiomy tvaru
    \begin{penumerate}
        \item $A \to (B \to A)$ \hfill (A1 - \uv{implikace sebe sama}),
        \item $\left(A \to \left(B \to C\right)\right) \to \left[\left(A \to B\right) \to \left(A \to C\right) \right]$ \hfill (A2 - \uv{rozn�soben�}),
        \item $\left(\neg B \to \neg A\right) \to \left(A \to B\right)$ \hfill (A3 - \uv{obr�cen� negace implikace}),
    \end{penumerate}
    \item odvozovac� pravidlo ($modus\ ponens$) -- z formul� $A$ a $A \to B$ odvo� formuli $B$.
\end{penumerate}
\end{definiceN}

\begin{definiceN}{Form�ln� syst�m predik�tov� logiky}
Pracujeme s redukovan�m jazykem (jen s log. spojkami $\neg,\rightarrow$ a jen s kvantifik�torem $\forall$). \emph{Sch�mata axiom� predik�tov� logiky} vzniknou z t�ch ve v�rokov� logice prost�m dosazen�m libovoln�ch formul� predik�tov� logiky za v�rokov� prom�nn�. \emph{Modus ponens} plat� i v pred. logice. PL obsahuje nav�c dal�� dva axiomy a odvozovac� pravidlo:
\begin{pitemize}
    \item \emph{sch�ma specifikace}: $(\forall x)A\rightarrow A_x[t]$
    \item \emph{sch�ma p�eskoku}: $(\forall x)(A\rightarrow B)\rightarrow (A\rightarrow(\forall x)B)$, pokud 
	prom�nn� $x$ nem� voln� v�skyt v $A$.
    \item \emph{pravidlo generalizace}: $\frac{A}{(\forall x)A}$
\end{pitemize}
Toto je form�ln� syst�m pred. logiky \emph{bez rovnosti}. S rovnost� p�ib�v� symbol $=$ a dal�� t�i axiomy.
\end{definiceN}

\begin{definiceN}{splnitelnost\vl}
Formule $A$ ve v�rokov� logice je \emph{splniteln�}, jestli�e existuje ohodnocen� $e$ takov�, �e $A$ je pravdiv� p�i $e$. Mno�ina formul� $T$ je splniteln�, pokud existuje ohodnocen� $e$ takov�, �e ka�d� formule $A\in T$ je pravdiv� p�i $e$. Potom $e$ naz�v�me \emph{modelem teorie} $T$ (zna��me $e\models T$).
\end{definiceN}

\begin{definiceN}{d�kaz\vl}
D�kaz $A$ je kone�n� posloupnost formul� $A_1,\dots A_n$, jestli�e $A_n = A$ a pro ka�d� $i=1,..n$ je $A_i$ bu� axiom, nebo je odvozen� z p�edchoz�ch pravidlem modus ponens (v predik�tov� logice nav�c mo�nost pou�it� pravidla generalizace). Existuje-li d�kaz formule $A$, pak je tato \emph{dokazateln�} ve v�rokov� logice (je v�tou v�rokov� logiky, zna��me $\vdash A$).
\end{definiceN}

\begin{definiceN}{d�kaz z p�edpoklad�}
D�kaz formule $A$ z p�edpoklad� je posloupnost formul� $A_1,\dots A_n$ takov�, �e $A_n = A$ a $\forall i\in\{1,..n\}$ je $A_i$ axiom, nebo prvek mno�iny p�edpoklad� $T$, nebo je odvozena z p�echoz�ch pravidlem modus ponens. Existuje-li d�kaz $A$ z $T$, pak $A$ \emph{je dokazateln�} z $T$, zna��me $T\vdash A$.
\end{definiceN}

\begin{vetaN}{o dedukci\vl}
Pro $T$ mno�inu formul� a formule $A,B$ plat� $T\vdash A\rightarrow B \mbox{ pr�v� kdy� } T,A\vdash B$.
\end{vetaN}
\begin{ideadukazu}
\begin{pitemize}
    \item[$\to$] M�me d�kaz formule $A \to B$, k n�mu m��eme z p�edpokladu p�ipojit $A$ a pomoc� MP odvodit $B$.
    \item[$\leftarrow$] $A_1, \ldots, A_n = B$ je d�kaz formule $B$ z p�edpoklad� $T, A$. Indukc� dok�eme, �e $T \vdash A \to A_i$, tedy pro $i = n$ jsme hotovi.
\end{pitemize}
\end{ideadukazu}

\begin{vetaN}{o dedukci\pl}
Nech� $T$ je mno�ina formul� pred. logiky, $A$ je uzav�en� formule a $B$ lib. formule, potom $T\vdash A\rightarrow B$ pr�v� kdy� $T,A\vdash B$.
\end{vetaN}
\begin{ideadukazu}
Podobn� jako ve VL, pouze v induk�n�m kroku mohlo b�t pou�ito pravidlo generalizace, proto po�adujeme, aby $A$ byla uzav�en�.
\end{ideadukazu}

\begin{dusledek}
Pro libovolnou mno�inu formul� $T$ a formule $A, B, C$ plat�:
$$T \vdash A \to \left(B \to C\right) \text{ pr�v� kdy� } T, A, B \vdash C \text{ pr�v� kdy� } T \vdash B \to \left(A \to C\right),$$
$$T \vdash \left(A \to \left(B \to C\right)\right) \to \left(B \to \left(A \to C\right)\right),$$
$$\vdash \left(A \to B\right) \to \left[\left(B \to C\right) \to \left(A \to C\right)\right].$$
Posledn�mu vztahu ��k�me \emph{v�ta o skl�d�n� implikac�}, v��e je uk�z�no, �e v implikaci nez�le�� na po�ad� p�edpoklad�.
\end{dusledek}

\begin{vetaN}{o neutr�ln� formuli\vl}
Nech� $T$ je mno�ina v�rokov�ch formul�, nech� $A, B$ jsou formule. Jestli�e $T, A \vdash B$ a $T, \neg A \vdash B$, pak $T \vdash B$.
\end{vetaN}

\begin{definiceN}{Tarsk�ho definice pravdy\pl}
Pro dan� (redukovan�, tj. jen se \uv{z�kladn�mi} log. spojkami) jazyk predik�tov� logiky $L$, $\c{M}$ jeho interpretaci, ohodnocen� $e$ a formuli $A$ tohoto jazyka plat�:
\begin{penumerate}
    \item $A$ je \emph{pravdiv� v $\mathcal{M}$ p�i ohodnocen�} $e$ nebo \emph{platn� v $\c{M}$ p�i ohodnocen�} $e$ (zna��me $\c{M}\models A[e]$), kdy�:
    \begin{pitemize}
	\item $A$ je atomick� tvaru $p(t_1,\dots,t_n)$, kde $p$ nen� rovnost a $(t_1[e],\dots,t_n[e])\in p_M$.
	\item $A$ je atomick� tvaru $t_1 = t_2$ a $t_1[e]=t_2[e]$
	\item $A$ je tvaru $\neg B$ a $\c{M}\not\models B[e]$
	\item $A$ je tvaru $B\rightarrow C$ a $\c{M}\not\models B[e]$ nebo $\mathcal{M}\models C[e]$
	\item $A$ je tvaru $(\forall x)B$ a $\c{M}\models B[e(x/m)]$ pro ka�d� $m\in M$
	\item $A$ je tvaru $(\exists x)B$ a $\c{M}\models B[e(x/m)]$ pro n�jak� $m\in M$
    \end{pitemize}
    \item $A$ je \emph{pravdiv� v interpretaci} $\mathcal{M}$ nebo \emph{platn� v interpretaci} $\c{M}$ ($\c{M}\models A$), jestli�e je $A$ pravdiv� v $M$ p�i ka�d�m ohodnocen� prom�nn�ch (pro uzav�en� formule sta�� jedno ohodnocen�, spln�n� je v�dy stejn�)
   \end{penumerate}
\end{definiceN}

\begin{definiceN}{logicky pravdiv�/platn� formule\pl}
Formule $A$ je \emph{validn� (logicky pravdiv�/platn�)} (zna��me $\models A$), kdy� je platn� p�i ka�d� interpretaci dan�ho jazyka.
\end{definiceN}

\begin{definiceN}{spornost, bezespornost}
Mno�ina formul� $T$ je \emph{sporn�}, pokud je z p�edpoklad� $T$ dokazateln� libovoln� formule, jinak je $T$ \emph{bezesporn�}. $T$ je \emph{maxim�ln� bezesporn�} mno�ina, pokud je $T$ bezesporn� a nav�c jedin� jej� bezesporn� nadmno�ina je $T$ samo. Mno�ina v�ech formul� dokazateln�ch z $T$ se zna�� $\mathit{Con}(T)$.
\end{definiceN}

\begin{vetaN}{o bezespornosti a splnitelnosti\vl}
Mno�ina formul� v�rokov� logiky je bezesporn�, pr�v� kdy� je splniteln�.
\end{vetaN}

\begin{definiceN}{teorie, model -- obecn�}
Pro n�jak� jazyk $L$ prvn�ho ��du je mno�ina $T$ formul� tohoto jazyka \emph{teorie prvn�ho ��du}. Formule z $T$ jsou \emph{speci�ln� axiomy} teorie $T$. Pro interpretaci $\c{M}$ jazyka $L$ je $\c{M}$ \emph{model teorie} $T$ (zna��me $\c{M}\models T$), pokud jsou v�echny speci�ln� axiomy $T$ pravdiv� v $\c{M}$. Formule $A$ je \emph{s�mantick�m d�sledkem} $T$ (zna��me $T\models A$), jestli�e je pravdiv� v ka�d�m modelu teorie $T$.
\end{definiceN}

\subsubsection*{Rozhodnutelnost}

\begin{definiceN}{rekurzivn� funkce a mno�ina}
\emph{Rekurzivn� funkce} jsou v�echny funkce popsateln� jako $f:\Nat^k\to\Nat$, kde $k\geq 1$, tedy v�echny \uv{algoritmicky vy��sliteln�} funkce. Mno�ina p�irozen�ch ��sel je \emph{rekurzivn� mno�ina (rozhodnuteln� mno�ina)}, pokud je rekurzivn� jej� charakteristick� funkce (funkce ur�uj�c�, kter� prvky do mno�iny pat��).
\end{definiceN}

\begin{definiceN}{spo�etn� jazyk, k�d formule}
\emph{Spo�etn� jazyk} je jazyk, kter� m� nejv�� spo�etn� mnoho speci�ln�ch symbol�. Pro spo�etn� jazyk, kde lze efektivn� (rekurzivn� funkc�) o��slovat jeho speci�ln� symboly, lze ka�d� jeho formuli $A$ p�i�adit jej� \emph{k�d formule} - p�ir. ��slo $\#A$.
\end{definiceN}

\begin{definiceN}{mno�ina k�d� v�t teorie}
Pro $T$ teorii s jazykem aritmetiky definujeme \emph{mno�inu k�d� v�t teorie} $T$ jako $Thm(T)=\{\#A|A \text{ je uzav�en� formule a } T\vdash A\}$.
\end{definiceN}

\begin{definiceN}{rozhodnuteln� teorie}
Teorie $T$ s jazykem aritmetiky je \emph{rozhodnuteln�}, pokud je mno�ina $Thm(T)$ rekurzivn�. V opa�n�m p��pad� je $T$ \emph{nerozhodnuteln�}.
\end{definiceN}

\begin{vetaN}{Churchova o nerozhodnutelnosti predik�tov� logiky}
Pokud spo�etn� jazyk $L$ prvn�ho ��du obsahuje alespo� jednu konstantu, alespo� jeden funk�n� symbol arity $k>0$ a pro ka�d� p�irozen� ��slo spo�etn� mnoho predik�tov�ch symbol�, potom mno�ina $\{\#A|A \text{ je uzav�en� formule a }L\models A\}$ nen� rozhodnuteln�.
\end{vetaN}

\begin{vetaN}{o nerozhodnosti predik�tov� logiky}
Nech� $L$ je jazyk prvn�ho ��du bez rovnosti a obsahuje alespo� 2 bin�rn� predik�ty. Potom je predik�tov� logika (jako teorie) s jazykem $L$ nerozhodnuteln�.
\end{vetaN}

\begin{definiceN}{T�i popisy aritmetiky}
Je d�n jazyk $L=\{0,S,+,\cdot\,\leq\}$.
\begin{pitemize}
    \item \emph{Robinsonova aritmetika} - "$Q$" s jazykem L m� 8 n�sl. axiom�:
    \begin{penumerate}
	\item $S(x)\neq 0$
	\item $S(x)=S(y)\rightarrow x=y$
	\item $x\neq 0\rightarrow (\exists y)(x=S(y))$
	\item $x+0=x$
	\item $x+S(y)=S(x+y)$
	\item $x\cdot 0=0$
	\item $x\cdot S(y)=(x\cdot y)+x$
	\item $x\leq y\leftrightarrow (\exists z)(z+x=y)$
    \end{penumerate}

    \textit{Pozn�mka: N�kdy, pokud nen� pot�eba definovat uspo��d�n�, se posledn� axiom spolu se symbolem \uv{$\leq$} vypou�t�.}

    \item \emph{Peanova aritmetika} - "$P$" m� v�echny axiomy Robinsonovy krom� t�et�ho, nav�c m� 
	\emph{Sch�ma(axiom�) indukce} - pro formuli $A$ a prom�nnou $x$ plat�: $A_x[0]\rightarrow 
	\{(\forall x)(A\rightarrow A_x[S(x)])\rightarrow(\forall x)A\}$.
    \item \emph{�pln� aritmetika} m� za axiomy v�echny uzav�en� formule pravdiv� v $\Nat$, je-li $\Nat$
	standardn� model aritmetiky - \uv{pravdiv� aritmetika}. \emph{Teorie modelu $\Nat$} je mno�ina
	$Th(\Nat)=\{A|A\text{ je uzav�en� formule a } \Nat\models A\}$.
\end{pitemize}
Plat�: $Q\subseteq P\subseteq Th(\Nat)$. $Q$ m� kone�n� mnoho axiom�, je tedy rekurzivn� axiomatizovateln�. $P$ m� spo�etn� mnoho axiom�, k�dy axiom� sch�matu indukce tvo�� rekurzivn� mno�inu. $Th(\Nat)$ nen� rekurzivn� axiomatizovateln�.
\end{definiceN}

\begin{vetaN}{Churchova o nerozhodnutelnosti aritmetiky}
Ka�d� bezesporn� roz���en� Robinsonovy aritmetiky $Q$ je nerozhodnuteln� teorie.
\end{vetaN}

\begin{vetaN}{G�del-Rosserova o ne�plnosti aritmetiky}
��dn� bezesporn� a rekurzivn� axiomatizovateln� roz���en� Robinsonovy aritmetiky $Q$ nen� �pln� teorie.
\end{vetaN}
\subsection{V�ty o kompaktnosti a �plnosti v�rokov� a predik�tov� logiky}

\begin{e}{Definice}{0}{�pln� teorie}
Teorie $T$ s jazykem $L$ prvn�ho ��du je \emph{�pln�}, je-li bezesporn� a pro libovolnou uzav�enou formuli $A$ je jedna z formul� $A,\neg A$ dokazateln� v $T$.
\end{e}

\begin{e}{V�ta}{1}{o korektnosti\vl}
Pro teorii $T$ a formuli $A$ plat� $T \vdash A \Rightarrow T \models A$. (Ka�d� v $T$ dokazateln� formule je v $T$ pravdiv�.)
\end{e}
\begin{e}{D�kaz}{0}{0}
Indukc� na v�t�ch $T$. Ka�d� axiom je pravdiv� (ov���me p��mo) a pravidlo modus ponens tak� zachov�v� pravdivost.
\end{e}

\begin{e}{V�ta}{0}{o bezespornosti a modelech\vl}
M�-li teorie model, je bezesporn�.
\end{e}
\begin{e}{D�kaz}{0}{0}
Formule $A$ a $\neg A$ nemohou b�t z�rove� platn� v ��dn�m modelu.
\end{e}

\begin{e}{V�ta}{1}{o spoust� v�c�\vl}
Nech� $A, B$ jsou formule teorie $T$. Plat� n�sleduj�c� tvrzen�.
\begin{penumerate}
    \item
    \begin{penumerate}
        \item Teorie $T$ je sporn�, pr�v� kdy� je v n� dokazateln� spor.
        \item (D�kaz sporem.) $T, \neg A$ je sporn� pr�v� kdy� $T \vdash A$.
    \end{penumerate}
    \item Bu� $T$ maxim�ln� bezesporn� teorie. Pak plat�:
    \begin{penumerate}
        \item $T \vdash A \Leftrightarrow A \in T \Leftrightarrow T, A$ je bezesporn�
        \item $A \in T \Leftrightarrow \neg A \notin T$ a d�le $\left(A \to B\right) \in T \Leftrightarrow (\neg A \in T$ nebo $B \in T)$.
        \item Ohodnocen� $e$ takov�, �e $e(p) = 1 \Leftrightarrow p \in T$ pro ka�dou v�rokovou prom�nnou $p$, je jedin� model $T$.
    \end{penumerate}
    \item Bezesporn� teorie m� maxim�ln� bezesporn� roz���en� (v t�m�e jazyce).
    \item (\textbf{O existenci modelu}.) Teorie m� model, pr�v� kdy� je bezesporn�.
    \item (\textbf{O kompaktnosti}.) Teorie m� model (tedy je splniteln�), pr�v� kdy� ka�d� jej� kone�n� podteorie m� model (je splniteln�).
    \item (\textbf{O �plnosti}.) $T \vdash A \Leftrightarrow T \models A$ plat� pro ka�dou teorii $T$ a jej� formuli $A$. D�sledkem je bezespornost v�rokov� logiky a dokazateln� v n� jsou pr�v� tautologie.
\end{penumerate}
\begin{e}{D�kaz}{0}{0}
\begin{penumerate}
    \item
    \begin{penumerate}
        \item Je-li $A$ spor ($\vdash \neg A$) a p�itom $T \vdash A$, pak d�ky v�t� $\vdash \neg A \to (A \to B)$ plat� jak�koliv v�rok $B$.
        \item
        \begin{pitemize}
            \item[$\Rightarrow$] Je-li $T, \neg A$ sporn�, pak $T \vdash \neg A \to A$ u�it�m v�ty o dedukci. D�ky v�t� $\vdash (\neg A \to A) \to A$ pak plat� $T \vdash A$.
            \item[$\Leftarrow$] Op�t d�ky v�t�m o dedukci a $\vdash \neg A \to (A \to B)$ lze dok�zat libovolnou formuli.
        \end{pitemize}
    \end{penumerate}
    \item 
    \begin{penumerate}
        \item z definic a maxim�lnosti
        \item $\neg A \notin T \Leftrightarrow T, \neg A$ je sporn� $\Leftrightarrow T \vdash A \Leftrightarrow A \in T$ dle 2a) a d�kazu sporem.\\
        Tvrzen� o implikaci: Kdy� $A \to B \in T$, tak z $\neg A \notin T$ plyne $A \in T$. Pak $T \vdash B$ a d�ky 2a) je $B \in T$. Kdy� $\neg A \in T$, tak $T \vdash A \to B$ d�ky v�t� $\vdash \neg C \to (C \to D)$, tedy $A \to B \in T$ d�ky a). Podobn� kdy� $B \in T$, tak $T, A \vdash B$, tud� $T \vdash A \to B$.
        \item Plat� $A \in T \Leftrightarrow e(A) = 1$, co� plyne indukc� dle slo�itosti $A$ ihned u�it�m b). Tedy $e \models T$. Kone�n� pro $e' \models T$ m�me $e'(p) = 1 \Leftrightarrow p \in  T$ pro ka�dou v�rokovou prom�nnou $p$, tedy $e' = e$.
    \end{penumerate}
    \item Plyne z principu maximality (ekvivalentn�ho s axiomem v�b�ru), aplikujeme-li jej na mno�inu v�ech bezesporn�ch teori� $S$ s $S \supseteq T$, na n� uva�ujeme uspo��d�n� inkluz�. Ka�d� �et�zec $R$ v popsan�m uspo��d�n� m� majorantu, kterou je jeho sjednocen� $\bigcup R$, nebo� to je teorie roz�i�uj�c� $T$, kter� je bezesporn�, proto�e spor v n� je sporem v n�jak� teorii z $R$.
    \item
    \begin{pitemize}
	    \item[$\Leftarrow$]M�-li $T$ model $e$ a $T \vdash A$, tak $e(A) = 1$, tedy $e(\neg A) = 0$, tedy $T \not \vdash \neg A$ a $T$ je bezesporn�.
	    \item[$\Rightarrow$]Nech� je $T$ bezesporn�. Dle 3) existuje maxim�ln� bezesporn� teorie $T' \supseteq T$ a dle 2c) existuje model teorie $T'$, co� je i model $T$.
    \end{pitemize}
    \item Plyne z 4) z toho, �e teorie je bezesporn�, pr�v� kdy� je bezesporn� ka�d� jej� kone�n� podteorie.
    \item $\Rightarrow$ je v�ta o korektnosti, v opa�n�m sm�ru p�edpokl�d�me $T \models A$. Pak je $T, \neg A$ sporn� dle tvrzen� o existenci modelu 4), tedy $T \vdash A$ dle d�kazu sporem 1b).
\end{penumerate}
\end{e}
\end{e}

\begin{e}{V�ta}{0}{o ekvivalenci\vl}
Vznikne-li formule $A'$ z $A$ nahrazen�m n�kter�ho v�skytu podformule $B$ formul� $B'$, tak plat�:
\begin{penumerate}
    \item $\vdash B \leftrightarrow B' \to A \leftrightarrow A'$,
    \item $T \vdash B \leftrightarrow B' \Rightarrow T \vdash A \leftrightarrow A'$.
\end{penumerate}
\end{e}

\begin{e}{V�ta}{1}{o existenci modelu, �plnosti a kompaktnosti\pl}
\begin{penumerate}
    \item (\textbf{O existenci modelu}.) Ka�d� bezesporn� teorie m� model kardinality nejv��e $\|L(T)\|$.
    \item (\textbf{O �plnosti}.) Formule teorie $T$ je dokazateln�, pr�v� kdy� je pravdiv� ($T \vdash A \Leftrightarrow T \models A$).
    \item (\textbf{O kompaktnosti}.) Teorie m� model (je splniteln�), pr�v� kdy� ka�d� jej� kone�n� ��st m� model. ($T \models A$ pr�v� kdy� $T' \models A$ pro n�jakou kone�nou podteorii $T' \subseteq T$.)
\end{penumerate}
\begin{e}{D�kaz}{0}{0}
\begin{penumerate}
    \item[2.] Pro formuli $A(x)$ u�it�m pravidla generalizace, d�kazu sporem a v�ty o existenci modelu m�me: $T \not\vdash A \Leftrightarrow T \not\vdash (\forall x)A \Leftrightarrow T, (\exists x)\neg A$ je bezesporn� $\Leftrightarrow T, (\exists x)\neg A$ m� model $\Leftrightarrow T \not\models A$.
    \item[3.] Plyne z toho, �e teorie je sporn�, pr�v� kdy� je n�jak� jej� kone�n� ��st sporn�.
\end{penumerate}
\end{e}
\end{e}

\subsection{Norm�ln� tvary v�rokov�ch formul�, prenexn� tvary formul� predik�tov� logiky}

\begin{poznamkaN}{Vlastnosti log. spojek}
Plat�:
\begin{penumerate}
    \item $A\wedge B\vdash A$; $A,B\vdash A\wedge B$
    \item $A\leftrightarrow B\vdash A\rightarrow B$; $A\rightarrow B, B\rightarrow A\vdash A\leftrightarrow B$
    \item $\wedge$ je idempotentn�, komutativn� a asociativn�.
    \item $\vdash(A_1\rightarrow\dots(A_n\rightarrow B)\dots)
	\leftrightarrow((A_1\wedge\dots\wedge A_n)\rightarrow B)$
    \item DeMorganovy z�kony: $\vdash\neg(A\wedge B)\leftrightarrow(\neg A\vee\neg B)$;
	$\vdash\neg(A\vee B)\leftrightarrow(\neg A\wedge\neg B)$
    \item $\vee$ je monotonn� ($\vdash A\rightarrow A\vee B$), idempotentn�, komutativn� a asociativn�.
    \item $\vee$ a $\wedge$ jsou navz�jem distributivn�.
\end{penumerate}
\end{poznamkaN}

\begin{vetaN}{o ekvivalenci ve v�rokov� logice}
Jestli�e jsou podformule $A_1\dots A_n$ formule $A$ ekvivalentn� s $A'_1\dots A'_n$ ($\vdash A'_i \leftrightarrow A_i$) a $A'$ vytvo��m nahrazen�m $A'_i$ m�sto $A_i$, je i $A$ ekvivalentn� s $A'$. (D�kaz indukc� podle slo�itosti formule, rozborem p��pad� $A_i$ tvaru $\neg B$, $B\rightarrow C$)
\end{vetaN}

\begin{lemmaN}{o d�kazu rozborem p��pad�}
Je-li $T$ mno�ina formul� a $A,B,C$ formule, pak $T,(A\vee B)\vdash C$ plat� pr�v� kdy� $T,A\vdash C$ a $T,B\vdash C$.
\end{lemmaN}

\begin{definiceN}{Norm�ln� tvary}
V�rokovou prom�nnou nebo jej� negaci nazveme \emph{liter�l}. \emph{Klauzule} budi� disjunkce n�kolika liter�l�. \emph{Formule v norm�ln�m konjunktivn�m tvaru (CNF)} je konjunkce klauzul�. \emph{Formule v disjunktivn�m tvaru (DNF)} je disjunkce konjunkc� liter�l�.
\end{definiceN}

\begin{vetaN}{o norm�ln�ch tvarech}
Pro ka�dou formuli $A$ lze sestrojit formule $A_k,A_d$ v konjunktivn�m, resp. disjunktivn�m tvaru tak, �e $\vdash A\leftrightarrow A_d$, $\vdash A\leftrightarrow A_k$. (D�kaz z DeMorganov�ch formul� a distributivity, indukc� podle slo�itosti formule)
\end{vetaN}

\subsubsection*{Prenexn� tvary formul� predik�tov� logiky}

\begin{vetaN}{o ekvivalenci v predik�tov� logice}
Nech� formule $A'$ vznikne z $A$ nahrazen�m n�kter�ch v�skyt� podformul� $B_1,\dots,B_n$ po �ad� formulemi $B'_1,\dots,B'_n$. Je-li $\vdash B_1\leftrightarrow B'_1,\dots,\vdash B_n\leftrightarrow B'_n$, potom plat� i $\vdash A\leftrightarrow A'$.
\end{vetaN}

\begin{definiceN}{Prenexn� tvar}
Formule predik�tov� logiky $A$ je v \emph{prenexn�m tvaru}, je-li $$A\equiv (Q_1 x_1)(Q_2 x_2)\dots(Q_n x_n)B,$$ kde $n\geq 0$ a $\forall i\in\{1,\dots,n\}$ je $Q_i\equiv \forall$ nebo $\exists$, $B$ je otev�en� formule a kvantifikovan� prom�nn� jsou navz�jem r�zn�. $B$ je \emph{otev�en� j�dro} $A$, ��st s kvantifik�tory je \emph{prefix} $A$.
\end{definiceN}

\begin{definiceN}{Varianta formule predik�tov� logiky}
Formule $A'$ je \emph{varianta} $A$, jestli�e vznikla z $A$ postupn�m nahrazen�m podformul� $(Q x)B$ (kde $Q$ je $\forall$ nebo $\exists$) formulemi $(Q y)B_x[y]$ a $y$ nen� voln� v $B$. Podle \emph{v�ty o variant�ch} je varianta s p�vodn� formul� ekvivalentn�.
\end{definiceN}

\begin{lemmaN}{o prenexn�ch operac�ch}
Pro p�evod formul� do prenexn�ho tvaru se pou��vaj� tyto operace (v�sledn� formule je s p�vodn� ekvivalentn�). Pro podformule $B$, $C$, kvantifik�tor $Q$ a prom�nnou $x$:
\begin{penumerate}
    \item podformuli lze nahradit n�jakou jej� variantou
    \item $\vdash \neg(Q x)B\leftrightarrow(\overline{Q} x)\neg B$
    \item $\vdash (B\rightarrow (Q x)C)\leftrightarrow(Q x)(B\rightarrow C)$, pokud $x$ nen� voln� v $B$
    \item $\vdash ((Q x)B\rightarrow C)\leftrightarrow(\overline{Q} x)(B\rightarrow C)$, pokud $x$ nen� 
	voln� v $C$
    \item $\vdash ((Q x)B\wedge C)\leftrightarrow (Q x)(B\wedge C)$, pokud $x$ nen� voln� v $C$
    \item $\vdash ((Q x)B\vee C)\leftrightarrow (Q x)(B\vee C)$, pokud $x$ nen� voln� v $C$
\end{penumerate}
\end{lemmaN}

\begin{vetaN}{o prenexn�ch tvarech}
Ke ka�d� formuli $A$ predik�tov� logiky lze sestrojit ekvivalentn� formuli $A'$, kter� je v prenexn�m tvaru. (D�kaz: indukc� podle slo�itosti formule a z prenexn�ch operac�, n�kdy je nutn� p�ejmenovat voln� prom�nn�)
\end{vetaN}


\end{document}
