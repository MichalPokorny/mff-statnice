\clearpage %&latex
\documentclass[a4paper]{article}

\frenchspacing

\usepackage[cp1250]{inputenc}
\usepackage[czech]{babel}

\usepackage{a4wide}
\usepackage{amsmath, amsthm, amssymb, amsfonts}
\usepackage[mathcal]{eucal}
\usepackage{graphicx}
\usepackage{url}
\usepackage{color}
\usepackage{wrapfig}
\usepackage{capt-of}
\usepackage{float}



% sirka a vyska textu nastavena jako papir, vsechny okraje vynulovany a pridano 20pt na kazdou stranu
% horizontalni rozmery
\setlength{\textwidth}{\paperwidth}
\addtolength{\textwidth}{-40pt}
\addtolength{\hoffset}{-1in}
\addtolength{\hoffset}{20pt}
\setlength{\oddsidemargin}{0in}
\setlength{\marginparsep}{0in}
% vertikalni rozmery
\setlength{\textheight}{\paperheight}
\addtolength{\textheight}{-60pt}
\addtolength{\voffset}{-1in}
\addtolength{\voffset}{20pt}
\setlength{\topmargin}{0in}
\setlength{\headheight}{0in}
\setlength{\headsep}{0in}


%Obrazek na miste
%pouziti
%%\obrazeknahore{adresa}{popisek}{label}
\long\def\obrazeknahore#1#2#3 {

\begin{figure}[t]
    \centering
    \includegraphics[width=0.8\textwidth]{#1}
    
    \caption{#2}
    \label{#3}
    
\end{figure}

}


%==========================================
%PEKELNA MAKRA NA ZAROVNANI OBRAZKU DOPRAVA

\makeatletter


%tohle je makro, ktere mi dovoluje obtekani i u kratkych environmentu
%ABSOLUTNE nechapu, jak to funguje, ale funguje to
%viz http://tex.stackexchange.com/questions/26078/ 
\def\odrovnej{\@@par
\ifnum\@@parshape=\z@ \let\WF@pspars\@empty \fi % reset `parshape'
\global\advance\c@WF@wrappedlines-\prevgraf \prevgraf\z@
\ifnum\c@WF@wrappedlines<\tw@ \WF@finale \fi}

\makeatother



%---
%makro, co da obrazek doprava a ostatni text ho obteka
%(bez toho predchazejiciho makra to ale poradne nebeha)
%pouziti:
%\obrazekvpravo{adresa}{popisek}{label}{procento sirky}
\long\def\obrazekvpravo#1#2#3#4{

\setlength\intextsep{-20pt}

    \begin{wrapfigure}{r}{#4\textwidth}
      \begin{center}
          \vspace{-10pt}
          
        \includegraphics[width=#4\textwidth]{#1}
        \vspace{-10pt}
        
      \end{center}
      
      \caption{#2}
      \label{#3}
      
      
    \end{wrapfigure}

\setlength\intextsep{0pt}

    
}




%---
%makro pro pripady, kdy wrapfigure neco mrsi
%je to docela pekelne
%je nutne mu dat jak text vpravo, tak text vlevo
%a nevim, jestli bude 100% fungovat, ale doufam, ze jo

%pouziti:
%\obrazekvpravominipage{adresa}{popisek}{label}{procento sirky}{1 - procento sirky}{text vlevo}
\long\def\obrazekvpravominipage#1#2#3#4#5#6{

\noindent\begin{minipage}{#5\linewidth}
\vspace{0pt}
#6
\end{minipage}
\hspace{0.5cm}
\noindent\begin{minipage}{#4\linewidth}
\vspace{0pt}
\centering
\includegraphics[width=0.9\textwidth]{#1}
\captionof{figure}{#2}
\label{#3}
\end{minipage}

}

%KONEC PEKELNYCH MAKER
%=====================

% makra pro poznamku u vyrokove a predikatove logiky
\def\vl{ -- ve v�rokov� logice}
\def\pl{ -- v predik�tov� logice}


%Vacsina prostredi je dvojjazicne. V pripade, ze znenie napr pozorovania je pisane po slovensky, malo by byt po slovensky aj oznacenie.

\newenvironment{pozadavky}{\pagebreak[2]\noindent\textbf{Po�adavky}\par\noindent\leftskip 10pt}{\odrovnej\par\bigskip}
\newenvironment{poziadavky}{\pagebreak[2]\noindent\textbf{Po�iadavky}\par\noindent\leftskip 10pt}{\odrovnej\par\bigskip}


\newenvironment{definiceSkull}{\pagebreak[2]\noindent\textbf{$\bigstar$ Definice}\par\noindent\leftskip 10pt}{\odrovnej\par\bigskip}
\newenvironment{definiceNSkull}[1]{\pagebreak[2]\noindent\textbf{$\bigstar$ Definice~}\emph{(#1)}\par\noindent\leftskip 10pt}{\odrovnej\par\bigskip}

\newenvironment{definice}{\pagebreak[2]\noindent\textbf{Definice}\par\noindent\leftskip 10pt}{\odrovnej\par\bigskip}
\newenvironment{definiceN}[1]{\pagebreak[2]\noindent\textbf{Definice~}\emph{(#1)}\par\noindent\leftskip 10pt}{\odrovnej\par\bigskip}
\newenvironment{definicia}{\pagebreak[2]\noindent\textbf{Defin�cia}\par \noindent\leftskip 10pt}{\odrovnej\par\bigskip}
\newenvironment{definiciaN}[1]{\pagebreak[2]\noindent\textbf{Defin�cia~}\emph{(#1)}\par\noindent\leftskip 10pt}{\odrovnej\par\bigskip}

\newenvironment{vetaSkull}{\pagebreak[2]\noindent\textbf{$\bigstar$ V�ta}\par\noindent\leftskip 10pt}{\odrovnej\par\bigskip}
\newenvironment{vetaNSkull}[1]{\pagebreak[2]\noindent\textbf{$\bigstar$ V�ta~}\emph{(#1)}\par\noindent\leftskip 10pt}{\odrovnej\par\bigskip}

\newenvironment{pozorovani}{\pagebreak[2]\noindent\textbf{Pozorov�n�}\par\noindent\leftskip 10pt}{\odrovnej\par\bigskip}
\newenvironment{pozorovanie}{\pagebreak[2]\noindent\textbf{Pozorovanie}\par\noindent\leftskip 10pt}{\odrovnej\par\bigskip}
\newenvironment{poznamka}{\pagebreak[2]\noindent\textbf{Pozn�mka}\par\noindent\leftskip 10pt}{\odrovnej\par\bigskip}
\newenvironment{poznamkaN}[1]{\pagebreak[2]\noindent\textbf{Pozn�mka~}\emph{(#1)}\par\noindent\leftskip 10pt}{\odrovnej\par\bigskip}
\newenvironment{lemma}{\pagebreak[2]\noindent\textbf{Lemma}\par\noindent\leftskip 10pt}{\odrovnej\par\bigskip}
\newenvironment{lemmaN}[1]{\pagebreak[2]\noindent\textbf{Lemma~}\emph{(#1)}\par\noindent\leftskip 10pt}{\odrovnej\par\bigskip}
\newenvironment{veta}{\pagebreak[2]\noindent\textbf{V�ta}\par\noindent\leftskip 10pt}{\odrovnej\par\bigskip}
\newenvironment{vetaN}[1]{\pagebreak[2]\noindent\textbf{V�ta~}\emph{(#1)}\par\noindent\leftskip 10pt}{\odrovnej\par\bigskip}
\newenvironment{vetaSK}{\pagebreak[2]\noindent\textbf{Veta}\par\noindent\leftskip 10pt}{\odrovnej\par\bigskip}
\newenvironment{vetaSKN}[1]{\pagebreak[2]\noindent\textbf{Veta~}\emph{(#1)}\par\noindent\leftskip 10pt}{\odrovnej\par\bigskip}

\newenvironment{dusledek}{\pagebreak[2]\noindent\textbf{D�sledek}\par\noindent\leftskip 10pt}{\odrovnej\par\bigskip}
\newenvironment{dosledok}{\pagebreak[2]\noindent\textbf{D�sledok}\par\noindent\leftskip 10pt}{\odrovnej\par\bigskip}

\newenvironment{dokaz}{\pagebreak[2]\noindent\leftskip 10pt\textbf{D�kaz}\par\noindent\leftskip 10pt}{\odrovnej\par\bigskip}
\newenvironment{dukaz}{\pagebreak[2]\noindent\leftskip 10pt\textbf{D�kaz}\par\noindent\leftskip 10pt}{\odrovnej\par\bigskip}

\newenvironment{ideadukazu}{\pagebreak[2]\noindent\leftskip 10pt\textbf{Idea d�kazu}\par\noindent\leftskip 10pt}{\odrovnej\par\bigskip}


\newenvironment{priklad}{\pagebreak[2]\noindent\textbf{P��klad}\par\noindent\leftskip 10pt}{\odrovnej\par\bigskip}
\newenvironment{prikladN}[1]{\pagebreak[2]\noindent\textbf{P��klad~}\emph{(#1)}\par\noindent\leftskip 10pt}{\odrovnej\par\bigskip}

\newenvironment{prikladSK}{\pagebreak[2]\noindent\textbf{Pr�klad}\par\noindent\leftskip 10pt}{\odrovnej\par\bigskip}
\newenvironment{priklady}{\pagebreak[2]\noindent\textbf{P��klady}\par\noindent\leftskip 10pt}{\odrovnej\par\bigskip}
\newenvironment{prikladySK}{\pagebreak[2]\noindent\textbf{Pr�klady}\par\noindent\leftskip 10pt}{\odrovnej\par\bigskip}

\newenvironment{algoritmusN}[1]{\pagebreak[2]\noindent\textbf{Algoritmus~}\emph{(#1)}\par\noindent\leftskip 10pt}{\odrovnej\par\bigskip}
%obecne prostredie, ktore ma vyuzitie pri specialnych odstavcoch ako (uloha, algoritmus...) aby nevzniklo dalsich x prostredi
\newenvironment{obecne}[1]{\pagebreak[2]\noindent\textbf{#1}\par\noindent\leftskip 10pt}{\odrovnej\par\bigskip}

\newenvironment{report}{\pagebreak[2]\noindent\textbf{Report}\em\par\noindent\leftskip 10pt}{\par\bigskip}

%\newenvironment{reportN}[1]{\pagebreak[2]\noindent\textbf{Report~}\emph{(#1)}\emph\par\noindent\leftskip 10pt}{\odrovnej\par\bigskip}
\newenvironment{reportN}[1]{\pagebreak[2]\noindent\textbf{Report~}\emph{(#1)}\em\par\noindent\leftskip 10pt}{\odrovnej\par\bigskip}

\newenvironment{penumerate}{
\begin{enumerate}
  \setlength{\itemsep}{1pt}
  \setlength{\parskip}{0pt}
  \setlength{\parsep}{0pt}
  %\setlength{\topsep}{200pt}
  \setlength{\partopsep}{200pt}
}{\end{enumerate}}

\def\pismenka{\numberedlistdepth=2} %pouzit, ked clovek chce opismenkovany zoznam...

\newenvironment{pitemize}{
\begin{itemize}
  \setlength{\itemsep}{1pt}
  \setlength{\parskip}{0pt}
  \setlength{\parsep}{0pt}
}{\end{itemize}}

%\definecolor{gris}{gray}{0.95}
\newcommand{\ramcek}[2]{\begin{center}\fcolorbox{white}{gris}{\parbox{#1}{#2}}\end{center}\par}
 \clearpage
\title{\LARGE U�ebn� texty k st�tn� bakal��sk� zkou�ce \\ Spr�va po��ta�ov�ch syst�m� \\ Datab�ze}
\begin{document}
\maketitle
\newpage
\setcounter{section}{2}
\section{Datab�ze}
\begin{e}{Po�adavky}{0}{0}
\begin{pitemize}
\item Podstata a architektury DB syst�m�
\item Norm�ln� formy
\item Referen�n� integrita
\item Transak�n� zpracov�n�, vlastnosti transakc�, uzamykac� protokoly, zablokov�n�
\item Z�klady SQL
\item Indexy, triggery, ulo�en� procedury, u�ivatel�, u�ivatelsk� pr�va
\item V�cevrstevn� architektury
\item Vazba datab�z� na internetov� technologie
\item Spr�va datab�zov�ch syst�m�
\end{pitemize}
\end{e}
\subsection{Podstata a architektury DB system�}

Zdroje: Wikipedie, slidy Dr. T. Skopala k Datab�zov�m syst�m�m
\bigskip

\begin{e}{Definice}{0}{Datab�ze}
Datab�ze je logicky uspo��dan� (integrovan�) kolekce navz�jem souvisej�c�ch dat. Je sebevysv�tluj�c�, proto�e data jsou uchov�v�na spole�n� s�popisy, zn�m�mi jako metadata (tak� sch�ma datab�ze). Data jsou ukl�d�na tak, aby na nich bylo mo�n� prov�d�t strojov� dotazy -- z�skat pro n�jak� parametry vyhovuj�c� podmno�inu z�znam�.

N�kdy se slovem \uv{datab�ze} mysl� obecn� cel� datab�zov� syst�m.
\end{e}

\begin{e}{Definice}{0}{Syst�m ��zen� b�ze dat}
Syst�m ��zen� b�ze dat (S�BD, anglicky database management system, DBMS) je obecn� softwarov� syst�m, kter� ��d� sd�len� p��stup k�datab�zi, a poskytuje mechanismy, pom�haj�c� zajistit bezpe�nost a integritu ulo�en�ch dat. Spravuje datab�zi a zaji��uje prov�d�n� dotaz�.
\end{e}

\begin{e}{Definice}{0}{Datab�zov� syst�m}
Datab�zov�m syst�mem rozum�me trojici, sest�vaj�c� z:
\begin{pitemize}
    \item datab�ze
    \item syst�mu ��zen� b�ze dat
    \item chud�ka admina
\end{pitemize}
\end{e}

\begin{obecne}{Smysl datab�z�}
Hlavn�m smyslem datab�ze je schra�ovat datov� z�znamy a informace za ��elem:
\begin{pitemize}
    \item sd�len� dat v�ce u�ivateli,
    \item zaji�t�n� unifikovan�ho rozhran� a jazyk� definice dat a manipulace s daty,
    \item znovuvyu�itelnosti dat,
    \item bezespornosti dat a
    \item sn�en� objemu dat (odstran�n� redundance).
\end{pitemize}
\end{obecne}

\subsubsection*{Datab�zov� modely}

\begin{e}{Definice}{0}{sch�ma, model}
Typicky pro ka�dou datab�zi existuje struktur�ln� popis druh� dat v n� udr�ovan�ch, ten naz�v�me \emph{sch�ma}. Sch�ma popisuje objekty reprezentovan� v datab�zi a vztahy mezi nimi. Je n�kolik mo�n�ch zp�sob� organizace sch�mat (modelov�n� datab�zov� struktury), zn�m�ch jako \emph{modely}. V modelu jde nejen o zp�sob strukturov�n� dat, definuje se tak� sada operac� nad daty provediteln�. Rela�n� model nap��klad definuje operace jako \uv{select} nebo \uv{join}. I kdy� tyto operace se nemusej� p��mo vyskytovat v dotazovac�m jazyce, tvo�� z�klad, na kter�m je jazyk postaven. Nejd�le�it�j�� modely v t�to sekci pop�eme.
\end{e}

\begin{e}{Pozn�mka}{0}{0}
V�t�ina datab�zov�ch syst�m� je zalo�ena na jednom konkr�tn�m modelu, ale ��m d�l �ast�j�� je podpora v�ce p��stup�. Pro ka�d� logick� model existuje v�ce fyzick�ch p��stup� implementace a v�t�ina syst�m� dovol� u�ivateli n�jakou �rove� jejich kontroly a �prav, proto�e toto m� velk� vliv na v�kon syst�mu. P��kladem nech� jsou indexy, provozovan� nad rela�n�m modelem.
\end{e}

\begin{obecne}{\uv{Ploch�} model}
Toto sice nevyhovuje �pln� definici modelu, p�esto se jako trivi�ln� p��pad uv�d�. P�edstavuje jedinou dvoudimension�ln� tabulku, kde data v jednom sloupci jsou pova�ov�na za popis stejn� vlastnosti (tak�e maj� podobn� hodnoty) a data v jednom ��dku se uva�uj� jako popis jedin�ho objektu.
\end{obecne}

\begin{obecne}{Rela�n� model}
Rela�n� model je zalo�en na predik�tov� logice a teorii mno�in. V�t�ina fyzicky implementovan�ch datab�zov�ch syst�m� ve skute�nosti pou��v� jen aproximaci matematicky definovan�ho rela�n�ho modelu. Jeho z�kladem jsou \emph{relace} (dvoudimension�ln� tabulky), \emph{atributy} (jejich pojmenovan� sloupce) a \emph{dom�ny} (mno�iny hodnot, kter� se ve sloupc�ch m��ou objevit). Hlavn� datovou strukturou je tabulka, kde se nach�z� informace o n�jak� konkr�tn�  t��d� entit. Ka�d� entita t� t��dy je potom reprezentov�na ��dkem v tabulce -- $n$-tic� atribut�.

V�echny relace (tj. tabulky) mus� spl�ovat z�kladn� pravidla -- po�ad� sloupc� nesm� hr�t roli, v tabulce se nesm� vyskytovat identick� ��dky a ka�d� ��dek mus� obsahovat jen jednu hodnotu pro ka�d� sv�j atribut. Rela�n� datab�ze obsahuje v�ce tabulek, mezi kter�mi lze popisovat vztahy (v�ech r�zn�ch kardinalit, tj. $1:1$, $1:n$ apod.). Vztahy vznikaj� i implicitn� nap�. ulo�en�m stejn� hodnoty jednoho atributu do dvou ��dk� v tabulce. K tabulk�m lze p�idat informaci o tom, kter� podmno�ina atribut� funguje jako \emph{kl��}, tj. unik�tn� identifikuje ka�d� ��dek, n�kter� z kl��� m��e b�t ozna�en jako prim�rn�. N�kter� kl��e m��ou m�t n�jak� vztah k vn�j��mu sv�tu, jin� jsou jen pro vnit�n� pot�eby sch�matu datab�ze (generovan� ID).
\end{obecne}

\begin{obecne}{Hierarchick� model}
V hierarchick�m modelu jsou data organizov�na do stromov� struktury -- ka�d� uzel m� odkaz na nad��zen� (k popisu hierarchie) a set��d�n� pole z�znam� na stejn� �rovni. Tyto struktury byly pou��v�ny ve star�ch mainframeov�ch datab�z�ch, nyn� je m��eme vid�t nap� ve struktu�e XML dokument�. Dovoluj� vztahy $1:N$ mezi dv�ma druhy dat, co� je velice efektivn� k popisu r�zn�ch re�ln�ch vztah� (obsahy, �azen� odstavc� textu, t��d�n� informace). Nev�hodou je ale nutnost zn�t celou cestu k z�znamu ve struktu�e a neschopnost syst�mu reprezentovat redundance v datech (strom nem� cykly).
\end{obecne}

\begin{obecne}{S�ov� model}
S�ov� model organizuje data pomoc� dvou hlavn�ch prvk�, \emph{z�znam�} a \emph{mno�in}. Z�znamy obsahuj� pole dat, mno�iny definuj� vztahy $1:N$ mezi z�znamy (jeden \emph{vlastn�k}, mnoho \emph{prvk�}). Z�znam m��e b�t vlastn�kem i prvkem v n�kolika r�zn�ch mno�in�ch. Jde vlastn� o variantu hierarchick�ho modelu, proto�e s�ov� model je tak� zalo�en na konceptu v�ce struktur ni��� �rovn� z�visl�ch na struktur�ch �rovn� vy���. U� ale umo��uje reprezentovat i redundantn� data. Operace nad t�mto modelem prob�haj� \uv{naviga�n�m} stylem: program si uchov�v� svoji sou�asnou pozici mezi z�znamy a postupuje podle z�vislost�, ve kter�ch se dan� z�znam n�ch�z�. Z�znamy mohou b�t i vyhled�v�ny podle kl��e. 

Fyzicky jsou v�t�inou mno�iny -- vztahy -- reprezentov�ny p��mo ukazateli na um�st�n� dat na disku, co� zaji��uje vysok� v�kon p�i vyhled�v�n�, ale zvy�uje n�klady na reorganizace. Smysl s�ov� navigace mezi objekty se pou��v� i v objektov�ch modelech.
\end{obecne}

\begin{obecne}{Objektov� model}
Objektov� model je aplikac� p��stup� zn�m�ch z objektov�-orientovan�ho programov�n�. Je zalo�en na sbli�ov�n� programov� aplikace a datab�ze, hlavn� ve smyslu pou�it� datov�ch typ� (objekt�) definovan�ch na jednom m�st�; ty zp��stup�uje k pou�it� v n�jak�m b�n�m programovac�m jazyce. Odstran� se tak nutnost zbyte�n�ch konverz� dat. P�in�� do datab�z� tak� v�ci jako zapouzd�en� nebo polymorfismus. Probl�mem objektov�ch model� je neexistence standard� (nebo sp� produkt�, kter� by je implementovaly).

Kombinac� objektov�ho a rela�n�ho p��stupu vznikaj� \emph{objektov�-rela�n�} datab�ze -- rela�n� datab�ze, dovoluj�c� u�ivateli definovat vlastn� datov� typy a operace na nich. Obsahuj� pak hybrid mezi procedur�ln�m a dotazovac�m programovac�m jazykem.

\end{obecne}


\subsubsection*{Architektury datab�zov�ch syst�m�}

Zdroj: Wiki �VUT (st�tnice na FELu ;-))
\bigskip

Architektury datab�zov�ch syst�m� se obecn� d�l� na 
\begin{pitemize}
    \item \emph{centralizovan�} (kde se datab�ze p�edpokl�d� fyzicky na jednom po��ta�i) a
    \item \emph{distribuovan�},
\end{pitemize}
p��padn� na
\begin{pitemize}
    \item \emph{jednou�ivatelsk�} a
    \item \emph{v�ceu�ivatelsk�}.
\end{pitemize}

\begin{obecne}{Distribuovan� datab�zov� syst�my}
\emph{Distribuovan� syst�m ��zen� b�ze dat} je vlastn� speci�ln�m p��padem obecn�ho distribuovan�ho v�po�etn�ho syst�mu. Jeho implementace zahrnuje fyzick� rozlo�en� dat (v�etn� mo�n�ch replikac� datab�ze) na v�ce po��ta�� -- \emph{uzl�}, p�i�em� jejich popis je integrov�n v glob�ln�m datab�zov�m sch�matu. Data v uzlech mohou b�t zpracov�v�na lok�ln�mi S�BD, komunikace je organizov�na v s�ov�m provozu pomoc� speci�ln�ho softwaru, kter� um� zach�zet s distribuovan�mi daty. Fyzicky se �e�� rozlo�en� do uzl�, sv�zan�ch komunika�n�mi kan�ly, a jeho transparence (neviditelnost -- navenek se m� tv��it jako jednolit� syst�m). Ka�d� uzel v s�ti je s�m o sob� datab�zov� syst�m a z ka�d�ho uzlu lze zp��stupnit data kdekoliv v s�ti.

D�le se d�l� na dva typy:
\begin{pitemize}
    \item Federativn� datab�ze -- neexistuje glob�ln� sch�ma ani centr�ln� ��d�c� autorita, ��zen� je tak� distribuovan�.
    \item Heterogenn� datab�zov� syst�my -- jednotliv� autonomn� S�BD existuj� (vznikly nez�visle na sob�) a jsou integrov�ny, aby spolu mohly komunikovat.
\end{pitemize}

V�hodou oproti centralizovan�m syst�m�m je vy��� efektivita (data mohou b�t ulo�ena bl�zko m�sta nej�ast�j��ho pou��v�n�), zv��en� dostupnost, v�konnost a roz�i�itelnost; nev�hodou z�st�v� probl�m slo�itosti implementace, distribuce ��zen� a ni��� bezpe�nost takov�ch �e�en�.
\end{obecne}

\begin{obecne}{V�ceu�ivatelsk� datab�zov� syst�my}
\emph{V�ceu�ivatelsk�} jsou takov� syst�my, kter� umo��uj� v�cen�sobn� u�ivatelsk� p��stup k dat�m ve stejn�m okam�iku. V d�sledku mo�n�ho sou�asn�ho p��stupu v�ce u�ivatel� je nutn� syst�m zabezpe�it tak, aby i nad�le zaji��oval integritu a konzistenci ulo�en�ch dat. Existuj� obecn� dva mo�n� p��stupy:
\begin{pitemize}
    \item Uzamyk�n� -- D��ve �asto pou��van� metoda zalo�en� na uzamyk�n� aktualizovan�ch z�znam�, v p��pad� masivn�ho vyu�it� aktualiza�n�ch p��kaz� u n� ale m��e doch�zet k zna�n�m prodlev�m. 
    \item Multiversion Concurency Control -- Modern�j�� vyn�lez. Jeho princip spo��v� v tom, �e p�i po�adavku o aktualizaci z�znamu v tabulce je vytvo�ena kopie z�znamu, kter� nen� pro ostatn� u�ivatele a� do proveden�ho commitu viditeln�.
\end{pitemize}
\end{obecne}

\subsection{Normální formy}

Je zřejmé, že u složitějších aplikací není každé schéma vhodné. Například pokud
by ve schématu docházelo k redundanci dat. Mohla by potom nastat situace, že při
nějaké aktualizaci, bychom změnili pouze některé výskyty a tím bychom databázi
přivedli do nekonzistentního stavu. Jedním ze způsobů jak se těchto problému
vyvarovat je mít databázi v (první, druhé \dots) \textbf{normální formě}.

Normálních forem je několik. Čím větší normální formu schéma splňuje tím je
snadnější ho udržovat v konzistentním stavu a později rozšiřovat. Vyšší stupeň
normalizovanosti vede obvykle k většímu počtu tabulek a tedy k nutnosti
používání většího množství operací typu \emph{join}. To může zapříčinit snížení
výkonu. Díky tomu jsou vysoce normalizovaná schémata typicky používána v
databázových aplikací, které používají spoustu izolovaných transakcí (např.
automatický bankovní systém). Naproti tomu méně normalizovaná schémata jsou
nasazována v aplikacích, které obsahují data převážně pro čtení (např.
zpravodajské systémy).

\begin{definice}
Řekneme, že atribut \emph{B} je \textbf{funkčně závislý} na atributu \emph{A}
(značíme $A\rightarrow B$), jestliže pro každou hodnotu atributu \emph{A}
existuje právě jedna hodnota atributu \emph{B}.\\[5mm]
\textbf{Nadklíčem}, někdy též \textbf{superklíčem}, schématu $A$ rozumíme každou
podmnožinu množiny $A$ na níž $A$ funkčně závisí. Jinak řečeno nadklíč je množina
atributů, která jednoznačně určuje řádek tabulky.\\[5mm]
\textbf{Klíč}, nebo také \textbf{potenciální klíč}(candidate key), schématu $A$
je takový nadklíč schématu $A$, jehož žádná vlastní podmnožina není nadklíčem
$A$. Čili minimální nadklíč.\\[5mm]
Každý atribut, který je obsažen alespoň v jednom potenciálním klíči se nazývá
\textbf{klíčový}, ostatní atributy jsou \textbf{neklíčové}.
\end{definice}

\subsubsection*{První normální forma (1NF)}

\begin{definiceN}{1NF} Schéma je v \emph{první normální formě}, jestliže každý atribut
schématu je elementárního (jednoduchého) typu, je nestrukturovaný.
\end{definiceN}

Jiná definice říká, že by schéma mělo být reprezentací nějaké \emph{relace} a
neobsahovat \emph{opakující se skupiny (repeating groups)}. Jelikož ale význam
\emph{opakujících se skupin} není přesně stanoven, existují jisté spory ohledně
toho které schéma 1NF splňuje a které ne.

\subsubsection*{Druhá normální forma (2NF)}

\begin{definiceN}{2NF}
Schéma je v \emph{druhé normální formě}, jestliže je v první normální formě a
žádný neklíčový atribut není funkčně závislý na žádné podmnožině klíče.
\end{definiceN}

To znamená, že neklíčový atribut může závist pouze na celém klíči. Pokud by
závisel jen na jeho části, měli bychom tabulku rozdělit na dvě. Mějme například
tabulku zaměstnanců s atributy: jméno, schopnosti, adresa. Ve které dvojice
\{jméno, schopnosti\} je klíč, čili jednoznačně určuje záznam. Nechť adresa
závisí pouze na jméně. Potom tabulka není v 2NF.

\begin{poznamka} Všimněme si pokud je schéma v 1NF a zároveň všechny její
potenciální klíče sestávají pouze z jednoho atributu, můžeme rovnou říct, že
schéma splňuje 2NF. \end{poznamka}

\subsubsection*{Třetí normální forma (3NF)}

\begin{definiceN}{3NF}
Schéma je ve \emph{třetí normální formě}, jestliže je v 2NF a žádný neklíčový
atribut není tranzitivně závislý na žádném klíči.
\end{definiceN}

Tranzitivní závislost, je taková funkční závislost $X \rightarrow Y$, že $Y$
nezávisí na $X$ přímo, ale existuje nějaké $Z$ takové, že $X \rightarrow Z$ a $Z
\rightarrow Y$. Jinak řečeno neklíčové atributy musí na klíči záviset přímo a ne
přes nějaký jiný atribut.

Alternativní definice říká, že schéma je v 3NF právě tehdy, když pro každou
funkční závislost $X \rightarrow Y$ platí alespoň jedna z následujících podmínek:
\begin{pitemize}
\item závislost je triviální, tj. $X$ obsahuje $Y$, 
\item $X$ je nadklíč schématu,
\item $Y$ je klíčový atribut, tj. $Y$ je obsažen v nějakém potenciálním klíči.
\end{pitemize}

\subsubsection*{Boyce-Coddova normální forma(BCNF)}

\begin{definiceN}{BCNF}
Schéma je v Boyce-Coddově normální formě právě tehdy, když pro každou
netriviální funkční závislost $X \rightarrow Y$ platí, že $X$ je nadklíč
schématu.
\end{definiceN}

BCNF je o něco silnější než 3NF. Pokud se podíváme na alternativní definici 3NF, je
dobře vidět rozdíl. Vynecháme-li třetí podmínku, dostaneme definici BCNF. Jinak
řečeno žádný samostatný atribut nesmí záviset na ničem jiném než na nadklíči.


\subsection{Referenční integrita}

Jedná se o nástroj databázového stroje, který pomáhá udržovat
vztahy v relačně propojených databázových tabulkách.

Referenční integrita se definuje \emph{cizím klíčem}, a to pro dvojici tabulek,
nebo nad jednou tabulkou, která obsahuje na sobě závislá data (například
stromové struktury). Tabulka, v níž je pravidlo uvedeno, se nazývá podřízená
tabulka (používá se také anglický termín slave). Tabulka, jejíž jméno je v
omezení uvedeno je nadřízená tabulka (master). Pravidlo referenční integrity
vyžaduje, aby každý záznam použitý v podřízené tabulce existoval v nadřízené
tabulce. To znamená, že každý záznam musí v cizím klíči obsahovat hodnoty
odpovídající primárními klíči v nadřízené tabulce, nebo null.

To sebou nese dva důsledky:
\begin{pitemize}
\item při přidání záznamu do podřízené tabulky se kontroluje, zda stejná hodnota
klíče existuje v nadřízené tabulce – porušení pravidla vyvolá chybu,
\item při mazání nebo úpravě záznamů v nadřízené tabulce se kontroluje, zda
v podřízené tabulce není záznam se stejnou hodnotou klíče – porušení pravidla
může vyvolat chybu nebo úpravu dat v podřízené tabulky v souladu s definovanými
akcemi.
\end{pitemize}

\begin{obecne}{Příklad}
V databázi spolku přátel psů máme následující tabulky:
\begin{pitemize}
 \item \emph{osoby} se sloupci \emph{\underline{osoba-id}} a \emph{jméno},
 \item \emph{psi} se sloupci \emph{\underline{pes-id}}, \emph{majitel} a \emph{rasa}.
\end{pitemize}

Aby byla data v databázi korektní, je třeba, aby každý záznam psa měl uvedeného
platného majitele. Proto označíme v tabulce psi sloupec \emph{majitel} jako cizí
klíč, vztažený k sloupci (klíči) \emph{osoba-id} v tabulce osoby. Když je poté
přidán záznam pro psa, databáze bude vyžadovat, aby číslo v poli \emph{majitel}
nabývalo některé z existujících hodnot \emph{osoba-id} tabulky osoby. Zároveň
můžeme určit, zda se při smazání osoby smažou i záznamy všech psů, kterým je
osoba majitelem, nebo zda má pokus o smazání osoby vlastnící alespoň jednoho psa
selhat.
\end{obecne}

Existuje tedy několik \emph{referenčních akcí}, které mohou být vyvolány, když
dochází ke změně nebo mazání v závislých tabulkách:
\begin{description}
  \item[CASCADE] -- Pokud je smazán řádek v nadřízené tabulce, řádky z podřízené
  tabulky obsahující mazaný cizí klíč budou smazány také.

  \item[RESTRICT] -- Řádek v nadřízené tabulce nesmí být smazán ani změněn, pokud
  v podřízené tabulce existují závisející řádky. Nedojde ani k pokusu o změnu dat.
  
  \item[NO ACTION] -- Operace Update nebo Delete je spuštěna na nadřízenou tabulku
  a na konci je teprve vyhodnoceno, jestli nedošlo k porušení integrity. Rozdíl
  oproti akci RESTRICT je ten, že samotným dotazem, nebo například triggerem může
  být zařízeno, aby k porušení integrity nedošlo. Potom je operace normálně
  provedena.
  
  \item[SET NULL] -- Cizí klíč v podřízené tabulce je nastaven na \emph{null},
  pokud dojde ke změně či smazání v odpovídajícím řádku nadřízené tabulky.

  \item[SET DEFAULT] -- Téměř to samé jako SET NULL, hodnoty cizího klíče jsou
  nastaveny na defaultní hodnotu sloupce, pokud dojde ke změně či smazání
  odpovídajícího řádku v nadřízené tabulce.
\end{description}

\subsection{Transakční zpracování, vlastnosti transakcí, uzamykací protokoly, zablokování}

\begin{definiceN}{Transakce}
\emph{Transakce} je jistá posloupnost nebo specifikace posloupnosti akcí práce s databází, jako
jsou čtení, zápis nebo výpočet, se kterou se zachází jako s jedním celkem.
\end{definiceN}

Hlavním smyslem používání transakcí, tj. \emph{transakčního zpracování}, je
udržení databáze v konzistentním stavu. Jestliže na sobě některé operace závisí,
sdružíme je do jedné transakce a tím zabezpečíme, že budou vykonány buď
všechny, nebo žádná. Databáze tak před i po vykonání transakce bude v
konzistentním stavu. Aby se uživateli transakce jevila jako jedna atomická
operace, je nutné zavést příkazy COMMIT a ROLLBACK. První z nich signalizuje
databázi úspěšnost provedení transakce, tj. veškeré změny v databázi se stanou
trvalými a jsou zviditelněny pro ostatní transakce, druhý příkaz signalizuje
opak, tj. databáze musí být uvedena do původního stavu.

Tyto příkazy většinou není nutné volat explicitně, např. příkaz COMMIT je vyvolán po
normálním ukončení programu realizujícího transakci. Příkaz ROLLBACK pro svou
funkci vyžaduje použití tzv. \emph{žurnálu} (logu) na nějakém stabilním
paměťovém médiu. Žurnál obsahuje historii všech změn databáze v jisté časové
periodě.

Jednoduchá transakce vypadá většinou takto:
\begin{penumerate}
  \item Začátek transakce,
  \item provedení několika dotazů -- čtení a zápisů (žádné změny v databázi nejsou zatím vidět pro
  okolní svět),
  \item Potvrzení (příkaz COMMIT) transakce (pokud se transakce povedla, změny
  v databázi se stanou viditelné).
\end{penumerate}
Pokud nějaký z provedených dotazů selže, systém by měl celou transakci zrušit a
vrátit databázi do stavu v jakém byla před zahájením transakce (operace ROLLBACK).

Transakční zpracování je také ochrana databáze před hardwarovými nebo
softwarovými chybami, které mohou zanechat databázi po částečném zpracování
transakce v nekonzistentním stavu. Pokud počítač selže uprostřed provádění
některé transakce, transakční zpracování zaručí, že všechny operace z
nepotvrzených (\uv{uncommitted}) transakcí budou zrušeny. 

\subsubsection*{Vlastnosti transakcí}

Podívejme se nyní na vlastnosti požadované po transakcích. Obvykle se používá
zkratka prvních písmen anglických názvů vlastností \textbf{ACID}~-- atomicity,
consistency, isolation (independence), durability. 
\begin{description}
  \item[atomicita] -- transakce se tváří jako jeden celek, musí buď proběhnout
  celá, nebo vůbec ne.
  \item[konzistence] -- transakce transformuje databázi z jednoho konzistentního
  stavu do jiného konzistentního stavu.
  \item[nezávislost] -- transakce jsou nezávislé, tj. dílčí efekty transakce
  nejsou viditelné jiným transakcím.
  \item[trvanlivost] -- efekty úspěšně ukončené (potvrzené,\uv{commited})
  transakce jsou nevratně uloženy do databáze a nemohou být zrušeny.
\end{description}

Transakce mohou být v uživatelských programech prováděny paralelně (spíše
zdánlivě paralelně, stejně jako je paralelismus multitaskingu na jednoprocesorových
strojích jen zdánlivý, zajistí to ale možnost paralelizace \uv{nedatabázových} 
akcí a pomalé transakce nebrzdí rychlé). Je
zřejmé, že posloupnost transakcí může být zpracována paralelně různým způsobem.
Každá transakce se skládá z několika akcí. Stanovené pořadí provádění akcí
více transakcí v čase nazveme \textbf{rozvrhem}.

Rozvrh, který splňuje následující podmínky, budeme nazývat \textbf{legální}:
\begin{pitemize}
  \item Objekt je nutné mít uzamknutý, pokud k němu chce transakce přistupovat.
  \item Transakce se nebude pokoušet uzamknout objekt již uzamknutý jinou
  transakcí (nebo musí počkat, než bude objekt odemknut).
\end{pitemize}

Důležitými pojmy pro paralelní zpracování jsou sériovost či uspořádatelnost.
\textbf{Sériové rozvrhy} zachovávají operace každé transakce pohromadě (a 
provádí se jen jedna transakce najednou). Pro $n$
transakcí tedy existuje $n!$ různých sériových rozvrhů. Pro získání korektního
výsledku však můžeme použít i rozvrhu, kde jsou operace různých transakcí
navzájem prokládány.
Přirozeným požadavkem na korektnost je, aby efekt paralelního zpracování
transakcí byl týž, jako kdyby transakce byly provedeny v nějakém sériovém rozvrhu.
Předpokládáme-li totiž, že každá transakce je korektní program, měl by vést
výsledek sériového zpracování ke konzistentnímu stavu. O systému zpracování
transakcí, který zaručuje dosažení konzistentního stavu nebo stejného stavu
jako sériové rozvrhy, se říká, že zaručuje \textbf{uspořádatelnost}.

Mohou se vyskytnout problémy, které uspořádatelnosti zamezují. Ty nazýváme \emph{konflikty}. Plynou z pořadí dvojic akcí různých transakcí na stejném objektu. Existují tři typy konfliktních situací:
\begin{penumerate}
    \item WRITE-WRITE -- přepsání nepotvrzených dat
    \item READ-WRITE -- neopakovatelné čtení
    \item WRITE-READ -- čtení nepotvrzených (\uv{uncommitted}) dat
\end{penumerate}

Řekneme, že rozvrh je \emph{konfliktově uspořádatelný}, je-li konfliktově ekvivalentní nějakému sériovému rozvrhu (tedy jsou v něm stejné, tj. žádné konflikty). Test na konfliktovou uspořádatelnost se dá provést jako test acykličnosti grafu, ve kterém konfliktní situace představují hrany a transakce vrcholy. Konfliktová uspořádatelnost je slabší podmínka než uspořádatelnost -- nezohledňuje ROLLBACK (\emph{zotavitelnost} -- zachování konzistence, i když kterákoliv transakce selže) a dynamickou povahu databáze (vkládání a mazání objektů). Zotavitelnosti se dá dosáhnout tak, že každá transakce $T$ je potvrzena až poté, co jsou potvrzeny všechny ostatní transakce, které změnily data čtená v $T$. Pokud v zotavitelném rozvrhu dochází ke čtení změn pouze potvrzených transakcí, nemůže dojít ani k jejich \emph{kaskádovému rušení}.

Při zpracování (i uspořádatelného) rozvrhu může dojít k situaci \emph{uváznutí} -- \emph{deadlocku}. To nastane tehdy, pokud jedna transakce $T_1$ čeká na zámek na objekt, který má přidělený $T_2$ a naopak. Situaci lze zobecnit i na více transakcí. Uváznutí lze buď přímo zamezit charakterem rozvrhu, nebo detekovat (hledáním cyklu v grafu čekajících transakcí, tzv. \uv{waits-for} grafu) a jednu z transakcí \uv{zabít} a spustit znova.

\medskip
K zajištění uspořádatelnosti a zotavitelnosti a zabezpečení proti kaskádovým rollbackům a deadlocku se používají různá schémata (požadavky na rozvrhy). Jedním z nich jsou uzamykací protokoly.

\subsubsection*{Uzamykací protokoly}

Vytváření rozvrhů a testování jejich uspořádatelnosti není pro praxi zřejmě ten
nejvhodnější způsob. Pokud ale budeme transakce konstruovat podle určitých
pravidel, tak za určitých předpokladů bude každý jejich rozvrh uspořádatelný.
Soustavě takových pravidel se říká \textbf{protokol}.

Nejznámější protokoly jsou založeny na dynamickém zamykání a odemykání objektů v
databázi. Zamykání (operace LOCK) je akce, kterou vyvolá transakce na objektu,
aby ho chránila před přístupem ostatních transakcí.

\begin{definiceN}{Dobře formovaná transakce}
Transakci nazveme \textbf{dobře formovanou} pokud podporuje přirozené požadavky
na transakce:
\begin{penumerate}
  \item transakce zamyká objekt, chce-li k němu přistupovat,
  \item transakce nezamyká objekt, který již je touto transakcí uzamčený,
  \item transakce neodmyká objekt, který není touto transakcí zamčený,
  \item po ukončení transakce jsou všechny objekty uzamčené touto transakcí
  odemčeny.
\end{penumerate}
\end{definiceN}

\paragraph{Dvoufázový protokol (2PL)} -- Dvoufázová transakce v první fázi
zamyká vše co je potřeba a od prvního odemknutí (druhá fáze) již jen odemyká co
měla zamčeno (již žádná operace LOCK). Tedy transakce musí mít všechny objekty
uzamčeny předtím, než nějaký objekt odemkne. Dá se dokázat, že pokud jsou
všechny transakce v dané množině transakcí dobře formované a dvoufázové, pak
každý jejich legální rozvrh je uspořádatelný.

Dvoufázový protokol zajišťuje uspořádatelnost, ale ne zotavitelnost ani
bezpečnost proti kaskádovému rušení transakcí nebo uváznutí.

\paragraph{Striktní dvoufázový protokol (S2PL)} -- Problémy 2PL jsou nezotavitelnost
a kaskádové rušení transakcí. Tyto nedostatky lze odstranit pomocí striktních
dvoufázových protokolů, které uvolňují zámky až po skončení transakce (COMMIT).
Zřejmá nevýhoda je omezení paralelismu. 2PL navíc stále nevylučuje možnost deadlocku.

\paragraph{Konzervativní dvoufázový protokol (C2PL)} -- Rozdíl oproti 2PL je
ten, že transakce žádá o všechny své zámky, ještě než se začne
vykonávat. To sice vede občas k zbytečnému zamykání (nevíme co přesně budeme
potřebovat, tak radši zamkneme víc), ale stačí to již k prevenci uváznutí
(deadlocku).

\subsubsection*{\uv{Vylepšení} zamykacích protokolů}

\paragraph{Sdílené a výlučné zámky} -- Nevýhodou 2PL je, že objekt může mít
uzamčený pouze jedna transakce. Abychom uzamykání provedli precizněji, je dobré
vzít na vědomí rozdíl mezi operacemi READ a WRITE. \emph{Výlučný zámek}
(W\_LOCK) může být aplikován na objekty jak pro operaci READ tak pro WRITE,
\emph{sdílený zámek} (R\_LOCK) uzamyká objekt, který chceme pouze číst. Jeden
objekt potom může být uzamčen sdíleným zámkem více transakcí a zvyšuje se tak
možnost paralelního zpracování. Budeme-li s těmito zámky zacházet stejně jako u
2PL, opět máme zaručenou uspořádatelnost rozvrhu, ovšem nikoliv absenci uváznutí.


\paragraph{Strukturované uzamykání (multiple granularity)} -- Objekty jsou v
tomto případě chápány hierarchicky dle relace \emph{obsahuje}. Například
databáze obsahuje soubory, které obsahují stránky a ty zase obsahují jednotlivé
záznamy. Na tuto hierarchii se můžeme dívat jako na strom, ve kterém každý
vrchol obsahuje své potomky. Když transakce zamyká objekt (vrchol) zamyká také
všechny jeho potomky. Protokol se tak snaží minimalizovat počet zámků, tím
snížit režii a zvýšit možnosti paralelního zpracování.


\subsubsection*{Alternativní protokoly}

\paragraph{Časová razítka} -- Další z protokolů zaručující uspořádatelnost je
využití časových razítek. Na začátku dostane transakce $T$ \emph{časové
razítko}~-- $TS(T)$ (časová razítka jsou unikátní a v čase rostou), abychom věděli
pořadí, ve kterém by měli být transakce vykonány. Každý objekt v databázi má
\emph{čtecí razítko}~-- $RTS(O)$ (read timestamp), které je aktualizováno, když je
objekt čten, a \emph{zapisovací razítko}~-- $WTS(O)$ (write timestamp), které je
aktualizováno, když nějaká transakce objekt mění.

Pokud chce transakce $T$ číst objekt $O$ mohou nastat dva případy:
\begin{pitemize}

  \item $TS(T) < WTS(O)$, tzn. někdo změnil objekt $O$ potom co byla spuštěna
  transakce $T$. V tomto případě musí být transakce zrušena a spouštěna znovu (a
  tedy s jiným časovým razítkem).

  \item $TS(T) > WTS(O)$, tzn. je bezpečné objekt číst. V tomto případě $T$
  přečte $O$ a $RTS(O)$ je nastaveno na $\max\{TS(T),\ RTS(O)\}$.

\end{pitemize}

Pokud chce transakce $T$ zapisovat do objektu $O$ rozlišujeme případy tři:
\begin{pitemize}

  \item $TS(T) < RTS(O)$, tzn. někdo četl $O$ poté co byla spuštěna $T$ a
  předpokládáme, že si pořídil lokální kopii. Nemůžeme tedy $O$ změnit, protože
  by lokální kopie přestala být platná a tedy je nutné $T$ zrušit a spustit
  znova.

  \item $TS(T) < WTS(O)$, tzn. někdo změnil $O$ po startu $T$. V tomto případě
  přeskočíme write operaci a pokračujeme dále normálně. $T$ nemusí být
  restartována.

  \item V ostatních případech $T$ změní $O$ a $WTS(O)$ je nastaveno na $TS(T)$.
\end{pitemize}

\paragraph{Optimistické protokoly} -- V situaci kdy se většina transakcí
neovlivňuje, je režie výše uvedených protokolů zbytečně velká a můžeme použít
takzvaný optimistický protokol. V protokolu můžeme rozlišit tři fáze.
\begin{penumerate}

  \item \textbf{Fáze čtení:} Čtou se objekty z databáze do lokální paměti a jsou
  na nich prováděny potřebné změny.

  \item \textbf{Fáze kontroly:} Po dokončení všech změn v lokální paměti je
  vyvolán pokus o zapsání výsledků do databáze. Algoritmus zkontroluje, zda
  nehrozí potenciální kolize s již potvrzenými transakcemi, nebo s některými
  právě probíhajícími. Pokud konflikt existuje, je třeba spustit algoritmus pro
  řešení kolizí, který se je snaží vyřešit. Pokud se mu to nepodaří, je využita
  poslední možnost a tou je zrušení a restartování transakce.

  \item \textbf{Fáze zápisu:} Pokud nehrozí žádné konflikty, jsou data z lokální
  paměti zapsány do databáze a transakce potvrzena.

\end{penumerate}



\subsection{Z�klady SQL}
TODO: p�evzato od \uv{program�tor�} z ot�zky \uv{SQL}, vzhledem k tomu, �e u n�s
se to jmenuje \uv{z�klady SQL} tak to mo�n� nemus� b�t tak podrobn�

Zdroje: slidy z p�edn�ek Datab�zov� syst�my a Datab�zov� aplikace Dr. T. Skopala a Dr. M. Kopeck�ho.

\subsubsection*{Standardy SQL}

SQL (\emph{Structured query language}) je standardn� jazyk pro p��stup k rela�n�m datab�z�m (a dotazov�n� nad nimi). Je z�rove� jazykem pro definici dat (definition data language), vytv��en� a modifikace sch�mat (tabulek), manipulaci s daty (data manipulation language), vkl��n�, aktualizace, maz�n� dat, ��zen� transakc�, definici integritn�ch omezen� aj. Jeho syntaxe odr�� snahu o co nejp�irozen�j�� formulace po�adavk� -- je podobn� anglick�m \uv{v�t�m}.

SQL je standard podle norem ANSI/ISO a existuje v n�kolika (zp�tn� kompatibiln�ch) verz�ch (ozna�ovan�ch podle roku uveden�):
\begin{description}
    \item[SQL 86] -- prvn� \uv{n�st�el}, pr�nik implementac� SQL firmy IBM
    \item[SQL 89] -- mal� revize motivovan� komer�n� sf�rou, mnoho detail� ponech�no implementaci
    \item[SQL 92] -- mnohem siln�j�� a obs�hlej�� jazyk. Zahrnuje u�
    \begin{pitemize}
	\item modifikace sch�mat, tabulky s metadaty, 
	\item vn�j�� spojen�, mno�inov� operace
	\item kask�dov� maz�n�/aktualizace podle ciz�ch kl���, transakce
	\item kurzory, v�jimky
    \end{pitemize}
    Standard existuje ve �ty�ech verz�ch: Entry, Transitional, Intermediate a Full.
    \item[SQL 1999] -- p�in�� mnoho nov�ch vlastnost�, nap�. 
    \begin{pitemize}	
	\item objektov�-rela�n� roz���en�
	\item nov� datov� typy -- reference, pole, full-text
	\item podpora pro extern� datov� soubory, multim�dia
	\item triggery, role, programovac� jazyk, regul�rn� v�razy, rekurzivn� dotazy ...
    \end{pitemize}
    \item[SQL 2003] -- dal�� roz���en�, nap�. XML management
\end{description}

Komer�n� syst�my implementuj� SQL podle r�zn�ch norem, n�kdy jenom SQL-92 Entry, dnes nej�ast�ji SQL-99, ale nikdy �pln� striktn�. N�kter� v�ci chyb� a naopak maj� v�echny spoustu nep�enositeln�ch roz���en� -- nap�. specifick� roz���en� pro procedur�ln�, transak�n� a dal�� funkcionalitu (T-SQL (Microsoft SQL Server), PL-SQL (Oracle) ). S nov�mi verzemi se kompatibilita zlep�uje, �asto je mo�n� pou��vat oboj� syntax. P�enos aplikace za b�hu na jinou platformu je ale st�le velice n�ro�n� -- a to t�m n�ro�n�j��, ��m v�c v�c� mimo SQL-92 Entry obsahuje.Pro otestov�n�, zda je �patn� syntax SQL, nebo zda jen dan� datab�zov� platforma nepodporuje n�kter� prvek, slou�� SQL valid�tory (kter� testuj� SQL podle norem.


\subsubsection*{Dotazy v SQL}

Hlavn�m n�strojem dotaz� v SQL je p��kaz \texttt{SELECT}. Sd�l� prvky rela�n�ho kalkulu i rela�n� algebry -- obsahuje pr�ci se sloupci, kvantifik�tory a agrega�n� funkce z rela�n�ho kalkulu a dal�� operace -- projekce, selekce, spojen�, mno�inov� operace -- z rela�n� algebry. Na rozd�l od striktn� formulace rela�n�ho modelu datab�ze povoluje duplik�tn� ��dky a NULLov� hodnoty atribut�.

Net��d�n� dotaz v SQL sest�v� z:
\begin{pitemize}
    \item p��kazu(�) \texttt{SELECT} (hlavn� logika dotazov�n�), to obsahuje v�dy
    \item m��e obsahovat i mno�inov� operace nad v�sledky p��kaz� \texttt{SELECT} -- \texttt{UNION}, \texttt{INTERSECTION} ...
\end{pitemize}
V�sledky nemaj� definovan� uspo��d�n� (resp. jejich po�ad� je ur�eno implementac� vyhodnocen� dotazu).

P��kaz \texttt{SELECT} vypad� n�sledovn� (tato verze u� zahrnuje i t��d�n� v�sledk�):
\begin{verbatim}
SELECT [DISTINCT]
 v�raz1 [[AS] c_alias1] [, ...]
FROM
 zdroj1 [[AS] t_alias1] [, ...]
[WHERE podm�nka_�]
[GROUP BY v�raz_g1 [, �]
[HAVING podm�nka_s]]
[ORDER BY v�raz_o1 [, �] ASC/DESC]
\end{verbatim}
Kde
\begin{pitemize}
    \item v�razy mohou b�t sloupce, sloupce s agrega�n�mi funkcemi, v�sledky dal��ch funkc� ...

\noindent \texttt{ v�raz = <n�zev sloupce>, <konstanta>, \\
 (DISTINCT) COUNT(~<n�zev sloupce>~),\\
{}[DISTINCT] [~SUM~|~AVG~](~<v�raz>~),\\
{}[~MIN~|~MAX~](~<v�raz>~)}\\
a nav�c lze pou��t oper�tory $+,-,*,/$.

    \item zdroje jsou tabulky nebo vno�en� selecty
    \item v�razy i zdroje b�t p�ejmenov�ny pomoc� \texttt{AS}, nap�. pro odkazov�n� uvnit� dotazu nebo jm�na na v�stupu (od SQL-92)
    \item podm�nka je logick� podm�nka (spojovan� logick�mi spojkami \texttt{AND, OR}) na hodnoty dat ve zdroj�ch:

\texttt{podm�nka = <v�raz> BETWEEN <x> AND <y>, <v�raz> LIKE "\%\_ ... ",\\
<v�raz> IS [NOT] NULL,\\
<v�raz> > = <> <= < > [<v�raz>/ ALL / ANY <dotaz>],\\
<v�raz> NOT IN [<seznam hodnot> / <dotaz>], EXIST ( <dotaz> )}

    \item \texttt{GROUP BY} znamen� agregaci podle unik�tn�ch hodnot jmenovan�ch sloupc� (v ostatn�ch sloupc�ch vznikaj� mno�iny hodnot, kter� se spolu s on�mi unik�tn�m� vyskytuj� na stejn�ch ��dk�ch
    \item \texttt{HAVING} ozna�uje podm�nku na agregaci
    \item \texttt{ORDER BY} definuje, podle hodnot ve kter�ch sloupc�ch nebo podle kter�ch jin�ch v�raz� nad nimi proveden�ch se m� v�sledek set��dit (\texttt{ASC} po�aduje vzestupn� set��d�n�, \texttt{DESC} sestupn�).
\end{pitemize}

SQL nem� p��kaz na omezen� rozsahu na n�kter� ��dky (jako nap�. \uv{pot�ebuji jen 50.-100. ��dek v�pisu}), a to lze �e�it bu� slo�it� standardn� (po��t�n� kolik hodnot je men��ch ne� vybran�, nav�c n�ro�n� na hardware) nebo pomoc� n�kter�ho nep�enositeln�ho roz���en�.

\medskip\noindent
Po�ad� vyhodnocov�n� jednoho p��kazu \texttt{SELECT} (nebereme v �vahu optimalizace):
\begin{penumerate}
    \item Nejprve se zkombinuj� data ze v�ech zdroj� (tabulek, pohled�, poddotaz�). Pokud jsou odd�leny ��rkami, provede se kart�zsk� sou�in (to sam� co \texttt{CROSS JOIN}), v SQL-92 a vy���m i slo�it�j�� spojen� -- \texttt{JOIN ON} (vnit�n� spojen� podle podm�nky), \texttt{NATURAL JOIN} (\uv{p�irozen�} spojen� podle stejn�ch hodnot stejn� pojmenovan�ch sloupc�), \texttt{OUTER JOIN} (\uv{vn�j��} spojen�, do kter�ho jsou zahrnuty i z�znamy, pro kter� v jednom ze zdroj� nen� nalezeno nic, co by odpov�dalo podm�nce, dopln�nn� NULLov�mi hodnotami) atd.
    \item Vy�ad� se vznikl� ��dky, kter� nevyhovuj� podm�nce (\texttt{WHERE})
    \item Zbyl� ��dky se seskup� do skupin se stejn�mi hodnotami uveden�ch v�raz� (\texttt{HAVING}), ka�d� skupina obsahuje atomick� sloupce s hodnotami uveden�ch v�raz� a mno�inov� sloupce se skupinami ostatn�ch hodnot sloupc�.
    \item Vy�ad� se skupiny, nevyhovuj�c� podm�nce (\texttt{HAVING})
    \item V�sledky se set��d� podle po�adavk�
    \item Vygeneruje se v�stup s po�adovan�mi hodnotami
    \item V p��pad� \texttt{DISTINCT} se vy�ad� duplicitn� ��dky
\end{penumerate}


\begin{e}{Pozn�mka}{0}{0}
\begin{pitemize}
    \item Klauzule \texttt{GROUP BY} set��d� p�ed vytvo�en�m skupin v�echny ��dky dle v�raz� v klauzuli. Proto by se m�l seskupovat co nejmen�� mo�n� po�et ��dek. Pokud je mo�n� ��dky odfiltrovat pomoc� WHERE, je v�sledek efektivn�j��, ne� n�sledn� odstra�ov�n� cel�ch skupin.
    \item  Klauzule \texttt{DISTINCT} t��d� v�sledn� z�znamy (p�ed operac� ORDER BY), aby na�la duplicitn� z�znamy. Pokud to jde, je vhodn� se bez n� obej�t.
    \item Klauzule \texttt{ORDER BY} by m�la b�t pou�ita jen v nutn�ch p��padech. Nen� p��li� vhodn� ji pou��vat v definic�ch pohled�, nad kter�mi se d�le d�laj� dal�� dotazy
\end{pitemize}
\end{e}


\subsubsection*{Definice a manipulace s daty, ostatn� p��kazy}

Standard SQL podporuje n�kolik druh� datov�ch typ�:
\begin{pitemize}
    \item textov� v n�rodn� a glob�ln� (UTF) znakov� sad� (n�kolika druh� -- prom�nn� a pevn� d�lky): \texttt{CHARACTER(n)}, \texttt{NCHAR(n)},
    \texttt{CHAR VARYING(n)}
    \item ��seln� typy -- \texttt{ NUMERIC(p[,s]), INTEGER, INT, SMALLINT,\\  FLOAT(presnost), REAL, DOUBLE PRECISION}
    \item datumov� typy -- \texttt{DATE, TIME, TIMESTAMP, TIMESTAMP(presnost\_sekund) WITH TIMEZONE}
\end{pitemize}
Datab�zov� servery ne v�dy podporuj� v�echny uveden� typy. Nemus� je podporovat nativn�, n�kdy si pouze \uv{p�elo��} n�zev typu na podobn� nativn� podporovan� typ.

\medskip
\begin{obecne}{P��kaz \texttt{CREATE TABLE}}
Tento p��kaz slou�� k vytvo�en� nov� tabulky. Je nutn� definovat jej� n�zev, atributy a jejich dom�ny (datov� typy); d�le je mo6n� definovat integritn� omezen� (kl��e, ciz� kl��e, odkazy, podm�nky). P��kaz vypad� n�sledovn�:
\begin{center}
\texttt{CREATE TABLE <n�zev> <def. sloupce/i.o. tabulky, ...> }
\end{center}
A uvnit� potom
\begin{verbatim}
def. sloupce = <n�zev> <dat.typ> 
    [DEFAULT NULL|<hodnota>] [<i.o.sloupce>] 
dat.typ = [VARCHAR(n) | BIT(n) | INTEGER | FLOAT | DECIMAL ...] 
i.o.sloupce = [CONSTRAINT <jm�no>] [NOT NULL / UNIQUE / PRIMARY KEY], 
    REFERERENCES <tabulka>(<sloupec>) <akce>, CHECK <podm�nka> 
akce = [ON UPDATE / ON DELETE] 
    [CASCADE / SET NULL / SET DEFAULT / NO ACTION(hl�en� chyby) ] 
i.o.tabulky = UNIQUE, PRIMARY KEY <sloupec, ... >, 
    FOREIGN KEY <sloupec, ... >, 
    REFERENCES <tabulka>(<sloupec, ... >), 
    CHECK( <podm�nka> )
\end{verbatim}
\end{obecne}

\medskip
\begin{obecne}{P��kazy pro manipulaci se sch�matem}
\begin{pitemize}
    \item �prava tabulky:
\begin{verbatim}
ALTER TABLE <n�zev> ADD {COLUMN} <def.sloupce>, ADD <i.o.tabulky>, 
    ALTER COLUMN <sloupec> [ SET / DROP ], DROP COLUMN <sloupec>, 
    DROP CONSTRAINT <jm�no i.o.> 
\end{verbatim}
    \item Smaz�n� tabulky (nen� to sam� jako vymaz�n� v�ech dat z tabulky!):
\begin{verbatim}
DROP TABLE <tabulka> 
\end{verbatim}
    \item Vytvo�en� \uv{pohledu} -- navenek se chov� jako tabulka, ale vnit�n� se p�i ka�d�m dotazu provede vno�en� dotaz (kter� definic� pohledu zapisuji):
\begin{verbatim}
CREATE VIEW <n�zev "tabulky"> ( <sloupec, ... > ) 
    AS <dotaz> {WITH [ LOCAL / CASCADED ] CHECK OPTION }
\end{verbatim}
    N�kter� datab�zov� platformy umo��uj� do takto vytvo�en�ch pohled� i zapisovat.
\end{pitemize}
\end{obecne}

\medskip
\begin{obecne}{P��kazy pro manipulaci s daty}
\begin{pitemize}
    \item Vlo�en� nov�ch dat do tabulky
\begin{verbatim}
INSERT INTO <tabulka> ( <sloupec, ... > ) 
    [VALUES ( <v�raz, ... > ) / (<dotaz>) ] 
\end{verbatim}
    \item �prava dat (na ��dc�ch kter� vyhovuj� podm�nce se nastav� zadan� hodnoty vybran�m sloupc�m):
\begin{verbatim}
UPDATE <tabulka> SET 
    ( <sloupec> = [ NULL / <v�raz> / <dotaz> ] , ... ) 
    WHERE (<podm�nka>) 
\end{verbatim}
    \item Smaz�n� ��dk� vyhovuj�c�ch podm�nce z tabulky:
\begin{verbatim}
DELETE FROM <tabulka> ( WHERE <podm�nka> ) 
\end{verbatim}
\end{pitemize}
\end{obecne}

\subsection{Indexy, triggery, ulo�en� procedury, u�ivatel�}

TODO: po��dn� definice indexu

\begin{obecne}{Index}
Index je obvykle definov�n v�b�rem tabulky a jej�ho konkr�tn�ho sloupce (nebo sloupc�), nad kter�mi si design�r datab�ze p�eje dotazov�n� urychlit; d�le pak technick�m ur�en�m typu. Chov�n� a zp�soby ulo�en� index� se mohou v�znamn� li�it podle pou�it� datab�zov� technologie.
V�jimku mohou tvo�it nap��klad full-textov� indexy, kter� jsou v n�kter�ch p��padech (nerela�n� datab�ze typu Lotus Notes) definov�ny nad celou datab�z�, nikoliv nad konkr�tn� tabulkou. 
\end{obecne}

\begin{obecne}{Pou�it� indexu}
Na prvn� pohled by se mohlo zd�t, �e ��m v�c index�, t�m lep�� chov�n� datab�ze a �e po vytvo�en� index� pro v�echny sloupce v�ech tabulk�ch dos�hneme maxim�ln�ho zrychlen�. Tento p��stup nar�� bohu�el na dva z�sadn� probl�my: 
\begin{penumerate}
    \item Ka�d� index zab�r� v pam�ti vyhrazen� pro datab�zi nezanedbateln� mno�stv� m�sta (vzhledem k pam�ti vyhrazen� pro tabulku). P�i existenci mnoha index� se m��e st�t, �e pam� zabran� pro jejich chod je skoro stejn� velk�, jako pam� zabran� jej�mi daty - zvl�t� u rozs�hl�ch tabulek (typu faktov�ch tabulek v datov�m skladu) m��e n�co takov�ho b�t nep�ijateln�. 
    \item Ka�d� index zpomaluje operace, kter� m�n� obsah indexovan�ch sloupc� (nap��klad SQL p��kazy UPDATE, INSERT). To je d�no t�m, �e datab�ze se v p��pad� takov� operace nad indexovan�m sloupcem mus� postarat nejen o zm�ny v datech tabulky, ale i o zm�ny v datech indexu. 
\end{penumerate}
\end{obecne}

\begin{obecne}{Typy index�}
Indexy mohou m�t sv�j typ, kter� bl�e ur�uje, jak�m zp�sobem m� b�t p�istupov�n� k dat�m tabulky optimalizov�no. Ozna�en� se r�zn�, ale nej�ast�ji je to:
\begin{pitemize}
    \item PRIMARY - Tento typ se v ka�d� tabulce m��e vyskytovat nejv��e jednou. Definuje sloupec tabulky, kter� svou hodnotou jednozna�n� identifikuje z�znam. Ve v�t�in� p��pad� se dodr�uje konvence takov� sloupec nazvat ID a jeho datov� typ stanovit jako cel� ��slo (nen�-li pot�eba jinak). Datab�zov� server by m�l b�t schopen nedopustit, aby byla do sloupce, k n�mu� se tento typ indexu vztahuje, byla vlo�ena hodnota, kter� ji� v tabulce existuje (v�t�inou takov� pokus kon�� chybovou hl�kou). 
    \item UNIQUE - Tento typ je podobn� PRIMARY co do jednozna�nosti z�znamu v tabulce (jak nazna�uje i jeho n�zev) a dopadu, kter� to na pr�ci s datab�z� m�; ale m��e se vyskytovat u v�ce sloupc� tabulky. Podle ��elu, ke kter�mu m� tabulka slou�it, se ob�as definuj� indexy slo�en� z v�ce sloupc� - potom op�t nelze vlo�it z�znam, kter� by ji� v t�to kombinaci n�kde v tabulce existoval. 
    \item INDEX - Definic� indexu tohoto typu je v tabulce zaji�t�na optimalizace vyhled�v�n� podle sloupce, ke kter�mu se dan� index v�e. V�t�inou si datab�zov� server vytvo�� a nad�le udr�uje vnit�n� seznam odkaz� na ��dky tabulky, se�azen� podle hodnot sloupce, k n�mu� se v�e. Udr�ov�n� takto se�azen� posloupnosti urychluje vyhled�v�n� (je mo�no pou��t n�kter� interpola�n� numerick� metody), �azen� i jin� z�sahy do tabulky, kter� jsou omezeny podm�nkou na doty�n� z�znamy. 
    \item FULL-TEXT - Vytvo�en�m tohoto indexu se datab�zov� server bude sna�it optimalizovat full-textov� vyhled�v�n� v dan�m sloupci u dan� tabulky.
\end{pitemize}
\end{obecne}

\begin{obecne}{Triggery}
  Datab�zov� trigger je programov� k�d, kter� je automaticky vykon�n jako
  reakce na n�jakou ud�lost v ur�it� datab�zov� tabulce. Triggery mohou omezit
  p��stup k ur�it�m dat�m, prov�d�t logov�n�, nebo kontrolovat zm�ny dat.

  Rozli�ujeme dv� hlavn� t��dy trigger� a to \emph{��dkov� trigger} a
  \emph{dotazov� (statement) trigger}. ��dkov� trigger m��eme definovat pro
  ka�d� ��dek tabulky, zat�mco dotazov� trigger se vykon� pouze jednou pro
  konkr�tn� datab�zov� dotaz. Ka�d� t��da trigger� m��e b�t n�kolika typ�. Jsou
  \emph{before triggers} a \emph{after triggers}, co� zna�� kdy m� b�t trigger
  vykon�n. Tak� se m��eme setkat s \emph{instead of triggers}, kter� je potom
  vykon�n \emph{m�sto} dotazu kter�m byl spu�t�n. 

  Jak� ud�losti mohou trigger spustit se pochopiteln� li�� datab�zov� syst�m od
  syst�mu, ale existuj� t�i typick� ud�losti, kter� to mohou b�t:
  \begin{penumerate}
    \item INSERT~-- nov� z�znam je vlo�en do datab�ze,
    \item UPDATE~-- z�znam je m�n�n,
    \item DELETE~-- z�znam je maz�n.
  \end{penumerate}
  Krom� t�chto typick�ch ud�lost� m��e datab�zov� syst�m umo��ovat nastavovat
  triggery tak� na maz�n�, �i vytv��en� cel�ch tabulek, �i dokonce p�ihl�en�
  nebo odhl�en� u�ivatele.

  Hlavn� vlastnosti a efekty datab�zov�ch trigger� jsou:
  \begin{pitemize}
    \item nep�ij�maj� ��dn� parametry nebo argumenty,
    \item nemohou volat operace pro ��zen� transakc� COMMIT a ROLLBACK,
    \item maj� p��stup k dat�m, kter� budou m�n�ny, je tedy mo�n� vykon�vat akce
    na z�klad� nich,
    \item nemohou vracet z�znamy,
    \item obt�n� se lad�,
    \item mnoho trigger� nebo slo�it� triggery mohou pr�ci s datab�z� velice
    zpomalit a nav�c znep�ehlednit.
  \end{pitemize}
\end{obecne}

\begin{obecne}{K �emu triggery pou��vat?}
  Triggery se v datab�z�ch pou��vaj� z n�kolika d�vod�, kter� mohou souviset s
  konzistenc� dat, jejich �dr�bou, nebo mohou b�t zp�sob, kter�m datab�ze
  komunikuje s okol�m. Pod�vejme se na n�kter� typick� sch�mata:
  \begin{pitemize}
    \item \textbf{Konzistence dat}~-- Trigger m��e prov�st v�po�et a na z�klad�
    toho povolit nebo nepovolit zm�nu dat v datab�zi. Nap��klad trigger m��e zak�zat
    smaz�n� z�kazn�ka z datab�ze v p��pad�, kdy m� u n�s n�jak� dluh a podobn�.  
    \item \textbf{Logov�n�}~-- Trigger m��e evidovat kdo, kdy a jak m�nil data.
    Lze tak dohledat pracovn�ka, kter� zadal �patn� �daje nebo zjistit, v kolik
    hodin do�lo k v�erej�� uz�v�rce.
    \item \textbf{Verzov�n� dat}~-- D�ky trigger�m lze snadno naprogramovat
    aplikaci tak, aby jedna tabulka udr�ovala historii zm�n tabulky jin�. To lze
    s �sp�chem pou��t t�eba jako bezpe�nostn� mechanismus. 
    \item \textbf{Zas�l�n� zpr�v}~-- Trigger m��e spustit n�jak� extern�
    program nebo proces. Nap��klad m��e trigger autorovi poslat e-mail,
    pokud byl k jeho �l�nku p�id�n p��sp�vek.
  \end{pitemize}
\end{obecne}

\begin{obecne}{Ulo�en� procedury}
  Ulo�en� procedura (anglicky stored procedure) je datab�zov� objekt, kter�
  neobsahuje data, ale ��st programu, kter� se nad daty v datab�zi m� vykon�vat.

  Ulo�en� procedura je p�edev��m procedura. Jedn� se o ��st programu, kter� je
  (nebo by aspo� m�l b�t) jasn� funk�n� odd�len� od sv�ho okol�, m� interface
  (seznam parametr�) pro komunikaci s jin�mi moduly programu. M��e m�t vlastn�
  lok�ln� prom�nn� neviditeln� pro ostatn� ��sti programu. 

  Ulo�en� procedura je ulo�en� (rozum�j: ulo�en� v datab�zi). To znamen�, �e se k
  n� lze chovat stejn� jako ke ka�d�mu jin�mu objektu datab�ze (indexu, pohledu,
  triggeru apod.). Lze j� zalo�it, upravovat a smazat pomoc� p��kaz� dotazovac�ho
  jazyka datab�ze (v p��pad� rela�n� datab�ze obvykle pomoc� p��kaz� DDL SQL). 

  Pro psan� ulo�en�ch procedur je obvykle pou��v�n specifick� jazyk konkr�tn�
  datab�ze, kter� je roz���en�m jej�ho dotazovac�ho jazyka (hezk�m p��kladem je
  datab�ze Oracle s procedur�ln�m jazykem PL/SQL, kter� je roz���en�m klasick�ho
  dotazovac�ho jazyka SQL).
\end{obecne}

\begin{obecne}{Pro� ukl�dat procedury?}
  \begin{pitemize}
      \item \textbf{Jednotn� rozhran�}~-- Pou�it� ulo�en�ch procedur vych�z� z
      faktu, �e v�t�ina operac� nad daty v datab�zi prob�h� stejn� bez ohledu na
      to, kdo operaci prov�d�. P��klad: Pokud je t�eba ulo�it do tabulky
      z�kazn�k� nov�ho z�kazn�ka, tak se to z pohledu datab�ze d�je stejn� pro
      z�kazn�ka internetov�ho obchodu, pro z�kazn�ka, kter�ho zad�v� pracovnice
      telefonick�ho centra p�es formul�� programu napsan�ho nap��klad v C++ ,
      nebo pro z�kazn�ky, kte�� jsou vkl�d�ni automaticky na z�klad� textov�ho
      reportu, kter� p�ijde ka�d� den z \uv{kamenn�ch} prodejn�ch m�st a je
      zpracov�v�n pomoc� programu napsan�ho v PowerBuilderu. Je tedy celkem
      dobr� d�vod, aby existovala ulo�en� procedura \uv{Zapi� nov�ho z�kazn�ka},
      kterou by mohly volat v�echny t�i v��e uveden� aplikace - alternativou bez
      ulo�en� procedury by bylo, �e bych podobnou proceduru musel napsat ve
      t�ech verz�ch - jednou v C++, jednou v Power Builderu a jednou v r�mci
      programu pro internetov� obchod (t�eba ASP nebo PHP). 
      \item \textbf{Skryt� datov�ch operac�}~-- Druhou v�hodou pou�it� ulo�en�ch
      procedur je, �e se nemus�m (v programu na \uv{klientsk�} stran�) zab�vat
      t�m, jak jsou data ulo�ena v konkr�tn�ch tabulk�ch. V na�em p��pad� je mi
      jedno, jak si datab�ze uvnit� pamatuje z�kazn�ky - prost� zad�m jako
      parametr procedury jm�no, p��jmen�, ��slo kreditky a co si z�kazn�k koupil
      - a datab�ze (resp. jej� ulo�en� procedura) si to n�jak p�ebere. Ulo�en�
      procedury se v p��pad� datab�zov�ch aplikac� staly z�kladn�m kamenem pro
      realizaci architektury klient/server, kdy je na jedn� stran� (klientsk�
      ��st) realizov�na v b�n�m procedur�ln�m programovac�m jazyku komunikace s
      u�ivatelem (formul��e nebo t�eba webov� str�nky) a na druh� stran�
      (serverov� ��st) je pomoc� ulo�en�ch procedur realizov�na spr�va dat v
      rela�n� datab�zi. Ob� ��sti (klientsk� a serverov�) mezi sebou komunikuj�
      p�es co nejjednodu��� rozhran� - vol�n�m ulo�en�ch procedur. 
  \end{pitemize}
\end{obecne}

TODO: u�ivatel�

\subsection{V�cevrstevn� architektury}

TODO: p�ed�lat, tohle je jen copy \& paste z Wiki a slajd� VUT Brno


\begin{obecne}{Multitier architecture}
Multi-tier architecture (often referred to as n-tier architecture) is a client-server architecture, originally designed by Jonathon Bolster of Hematites Corp, in which an application is executed by more than one distinct software agent. For example, an application that uses middleware to service data requests between a user and a database employs multi-tier architecture. The most widespread use of "multi-tier architecture" refers to three-tier architecture
\end{obecne}


\begin{obecne}{Z�klad kooperativn�ho zpracov�n�}
Faktory ovliv�uj�c� architekturu:
\begin{pitemize}
\item po�adavky na interoperabilitu zdroj� 
\item r�st velikosti zdroj� 
\item r�st po�tu klient� 
\end{pitemize}
Typy slu�eb v datab�zov� technologii:
\begin{pitemize}
\item prezenta�n� slu�by: p��jem vstupu, zobrazov�n� v�sledk� 
\item prezenta�n� logika: ��zen� interakce (hierarchie menu, obrazovek)
\item logika aplikace: operace realizuj�c� algoritmus aplikace
\item logika dat: podpora operac� s daty (integritn� omezen�, ...)
\item datov� slu�by: akce s datab�z� (definice a manipulace, transak�n� zpracov�n�, ...) 
\item slu�by ovl�d�n� soubor�: vlastn� V/V operace
\end{pitemize}
\end{obecne}

\begin{obecne}{Varianty architektury klient-server}
\begin{pitemize}
    \item \textbf{Klient-server se vzd�len�mi daty} \\
Na serveru jsou jen datov� slu�by a ovl�d�n� soubor�, zbytek zaji��uj� klienti. Probl�mem jsou velk� n�roky na p�enosovu kapacitu od klienta k serveru a HW zat�en� klientsk�ch stanic
    \item \textbf{Klient-server se vzd�lenou prezentac�} \\
Na klientsk� stanici jsou jen prezenta�n� slu�by a prezenta�n� logika, zbytek je na serveru. Nev�hodou je pr�v� z�t� na HW serveru.
    \item \textbf{Klient-server s rozd�lenou logikou} \\
��st logiky aplikac� i logiky dat je na serveru a ��st zpracov�v� klient. Jde o vyv�en� �e�en�, kter� m� ale hor�� roz���itelnost.
    \item \textbf{T��vrstv� architektura} \\
Zahrnuje dva servery -- aplika�n� a datab�zov�, spojen� rychlou linkou. Z hlediska z�t�e a roz���itelnosti nejv�hodn�j��.
\end{pitemize}
\end{obecne}

\begin{obecne}{P��nos architektury klient/server a t��vrstv� architektury}
\begin{pitemize}
 \item pru�n�j�� rozd�len� pr�ce \item lze pou��t horizont�ln�(v�ce server�) i vertik�ln�(v�konn�j�� server) �k�lov�n� \item aplikace mohou b�et na levn�j��ch za��zen�ch \item na klientsk�ch stanic�ch lze pou��vat obl�ben� prezenta�n� software \item standardizovan� p��stup umo��uje zp��stupnit dal�� zdroje \item centralizace dat podporuje ��inn�j�� ochranu \item u t��vrstv� architektury centralizace �dr�by aplikace, mo�nost vyu�it� sd�len�ch objekt� (business objects) n�kolika aplikacemi
\end{pitemize}
\end{obecne}

\begin{obecne}{Podpora pro rozd�len� z�t�e v architektu�e klient/server}
\begin{pitemize}
 \item deklarativn� integritn� omezen� \item datab�zov� triggery \item ulo�en� podprogramy
\end{pitemize}
\end{obecne}


\begin{obecne}{Three-tier architecture}
'Three-tier' is a client-server architecture in which the user interface, functional process logic ("business rules"), data storage and data access are developed and maintained as independent modules, most often on separate platforms. The term "three-tier" or "three-layer", as well as the concept of multitier architectures, seems to have originated within Rational Software.

The three-tier model is considered to be a software architecture and a software design pattern.

Apart from the usual advantages of modular software with well defined interfaces, the three-tier architecture is intended to allow any of the three tiers to be upgraded or replaced independently as requirements or technology change. For example, a change of operating system from Microsoft Windows to Unix would only affect the user interface code.

Typically, the user interface runs on a desktop PC or workstation and uses a standard graphical user interface, functional process logic may consist of one or more separate modules running on a workstation or application server, and an RDBMS on a database server or mainframe contains the data storage logic. The middle tier may be multi-tiered itself (in which case the overall architecture is called an "n-tier architecture").

The 3-Tier architecture has the following 3-tiers:
\begin{pitemize}
    \item Presentation Tier
    \item Application Tier/Logic Tier/Business Logic Tier
    \item Data Tier
\end{pitemize}
\end{obecne}


\begin{obecne}{Web Development usage}

In the Web development field, three-tier is often used to refer to Websites, commonly Electronic commerce websites, which are built using three tiers:
\begin{pitemize}
    \item A front end Web server serving static content
    \item A middle dynamic content processing and generation level Application server, for example Java EE platform.
    \item A back end Database, comprising both data sets and the Database management system or RDBMS software that manages and provides access to the data.
\end{pitemize}
\end{obecne}

\begin{obecne}{Other Considerations}

To further confuse issues, the particular data transfer method between the 3 tiers must also be considered. The data exchange may be file-based, client-server, event-based, etc. Protocols involved may include one or more of SNMP, CORBA, Java RMI, Sockets, UDP, or other proprietary combinations/permutations of the above types and others. Typically a single "middle-ware" implementation of a single protocol is chosen as the "standard" within a given system, such as J2EE (which is Java specific) or CORBA (which is language/OS neutral.) The importance of the decision of which protocol is chosen affects such issues as the ability to include legacy applications/libraries, performance, maintainability, etc. When choosing a "middle-ware protocol" (not to be confused with the "middle-of-the-three-tiers") engineers should not be swayed by "public opinion" about a protocol's modern-ness, but should consider the technical benefits and suitability to solve a problem. (for example CGI is very old and "out of date" but is still quite useful and powerful, so is shell scripting, and UDP for that matter)

Ideally the high-level system abstract design is based on business rules and not on the front-end/back-end technologies. The tiers should be populated with functionality in such a way as to minimize dependencies, and isolate functionalities in a coherent manner - knowing that everything is likely to change, and changes should be made in the fewest number of places, and be testable.
\end{obecne}

\subsection{Vazba datab�z� na internetov� technologie}

TODO: v�echno

\subsection{Spr�va datab�zov�ch syst�m�}

TODO: v�echno


\end{document}
