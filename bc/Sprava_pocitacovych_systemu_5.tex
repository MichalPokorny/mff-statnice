\clearpage %&latex
\documentclass[a4paper]{article}

\frenchspacing

\usepackage[cp1250]{inputenc}
\usepackage[czech]{babel}

\usepackage{a4wide}
\usepackage{amsmath, amsthm, amssymb, amsfonts}
\usepackage[mathcal]{eucal}
\usepackage{graphicx}
\usepackage{url}
\usepackage{color}
\usepackage{wrapfig}
\usepackage{capt-of}
\usepackage{float}



% sirka a vyska textu nastavena jako papir, vsechny okraje vynulovany a pridano 20pt na kazdou stranu
% horizontalni rozmery
\setlength{\textwidth}{\paperwidth}
\addtolength{\textwidth}{-40pt}
\addtolength{\hoffset}{-1in}
\addtolength{\hoffset}{20pt}
\setlength{\oddsidemargin}{0in}
\setlength{\marginparsep}{0in}
% vertikalni rozmery
\setlength{\textheight}{\paperheight}
\addtolength{\textheight}{-60pt}
\addtolength{\voffset}{-1in}
\addtolength{\voffset}{20pt}
\setlength{\topmargin}{0in}
\setlength{\headheight}{0in}
\setlength{\headsep}{0in}


%Obrazek na miste
%pouziti
%%\obrazeknahore{adresa}{popisek}{label}
\long\def\obrazeknahore#1#2#3 {

\begin{figure}[t]
    \centering
    \includegraphics[width=0.8\textwidth]{#1}
    
    \caption{#2}
    \label{#3}
    
\end{figure}

}


%==========================================
%PEKELNA MAKRA NA ZAROVNANI OBRAZKU DOPRAVA

\makeatletter


%tohle je makro, ktere mi dovoluje obtekani i u kratkych environmentu
%ABSOLUTNE nechapu, jak to funguje, ale funguje to
%viz http://tex.stackexchange.com/questions/26078/ 
\def\odrovnej{\@@par
\ifnum\@@parshape=\z@ \let\WF@pspars\@empty \fi % reset `parshape'
\global\advance\c@WF@wrappedlines-\prevgraf \prevgraf\z@
\ifnum\c@WF@wrappedlines<\tw@ \WF@finale \fi}

\makeatother



%---
%makro, co da obrazek doprava a ostatni text ho obteka
%(bez toho predchazejiciho makra to ale poradne nebeha)
%pouziti:
%\obrazekvpravo{adresa}{popisek}{label}{procento sirky}
\long\def\obrazekvpravo#1#2#3#4{

\setlength\intextsep{-20pt}

    \begin{wrapfigure}{r}{#4\textwidth}
      \begin{center}
          \vspace{-10pt}
          
        \includegraphics[width=#4\textwidth]{#1}
        \vspace{-10pt}
        
      \end{center}
      
      \caption{#2}
      \label{#3}
      
      
    \end{wrapfigure}

\setlength\intextsep{0pt}

    
}




%---
%makro pro pripady, kdy wrapfigure neco mrsi
%je to docela pekelne
%je nutne mu dat jak text vpravo, tak text vlevo
%a nevim, jestli bude 100% fungovat, ale doufam, ze jo

%pouziti:
%\obrazekvpravominipage{adresa}{popisek}{label}{procento sirky}{1 - procento sirky}{text vlevo}
\long\def\obrazekvpravominipage#1#2#3#4#5#6{

\noindent\begin{minipage}{#5\linewidth}
\vspace{0pt}
#6
\end{minipage}
\hspace{0.5cm}
\noindent\begin{minipage}{#4\linewidth}
\vspace{0pt}
\centering
\includegraphics[width=0.9\textwidth]{#1}
\captionof{figure}{#2}
\label{#3}
\end{minipage}

}

%KONEC PEKELNYCH MAKER
%=====================

% makra pro poznamku u vyrokove a predikatove logiky
\def\vl{ -- ve v�rokov� logice}
\def\pl{ -- v predik�tov� logice}


%Vacsina prostredi je dvojjazicne. V pripade, ze znenie napr pozorovania je pisane po slovensky, malo by byt po slovensky aj oznacenie.

\newenvironment{pozadavky}{\pagebreak[2]\noindent\textbf{Po�adavky}\par\noindent\leftskip 10pt}{\odrovnej\par\bigskip}
\newenvironment{poziadavky}{\pagebreak[2]\noindent\textbf{Po�iadavky}\par\noindent\leftskip 10pt}{\odrovnej\par\bigskip}


\newenvironment{definiceSkull}{\pagebreak[2]\noindent\textbf{$\bigstar$ Definice}\par\noindent\leftskip 10pt}{\odrovnej\par\bigskip}
\newenvironment{definiceNSkull}[1]{\pagebreak[2]\noindent\textbf{$\bigstar$ Definice~}\emph{(#1)}\par\noindent\leftskip 10pt}{\odrovnej\par\bigskip}

\newenvironment{definice}{\pagebreak[2]\noindent\textbf{Definice}\par\noindent\leftskip 10pt}{\odrovnej\par\bigskip}
\newenvironment{definiceN}[1]{\pagebreak[2]\noindent\textbf{Definice~}\emph{(#1)}\par\noindent\leftskip 10pt}{\odrovnej\par\bigskip}
\newenvironment{definicia}{\pagebreak[2]\noindent\textbf{Defin�cia}\par \noindent\leftskip 10pt}{\odrovnej\par\bigskip}
\newenvironment{definiciaN}[1]{\pagebreak[2]\noindent\textbf{Defin�cia~}\emph{(#1)}\par\noindent\leftskip 10pt}{\odrovnej\par\bigskip}

\newenvironment{vetaSkull}{\pagebreak[2]\noindent\textbf{$\bigstar$ V�ta}\par\noindent\leftskip 10pt}{\odrovnej\par\bigskip}
\newenvironment{vetaNSkull}[1]{\pagebreak[2]\noindent\textbf{$\bigstar$ V�ta~}\emph{(#1)}\par\noindent\leftskip 10pt}{\odrovnej\par\bigskip}

\newenvironment{pozorovani}{\pagebreak[2]\noindent\textbf{Pozorov�n�}\par\noindent\leftskip 10pt}{\odrovnej\par\bigskip}
\newenvironment{pozorovanie}{\pagebreak[2]\noindent\textbf{Pozorovanie}\par\noindent\leftskip 10pt}{\odrovnej\par\bigskip}
\newenvironment{poznamka}{\pagebreak[2]\noindent\textbf{Pozn�mka}\par\noindent\leftskip 10pt}{\odrovnej\par\bigskip}
\newenvironment{poznamkaN}[1]{\pagebreak[2]\noindent\textbf{Pozn�mka~}\emph{(#1)}\par\noindent\leftskip 10pt}{\odrovnej\par\bigskip}
\newenvironment{lemma}{\pagebreak[2]\noindent\textbf{Lemma}\par\noindent\leftskip 10pt}{\odrovnej\par\bigskip}
\newenvironment{lemmaN}[1]{\pagebreak[2]\noindent\textbf{Lemma~}\emph{(#1)}\par\noindent\leftskip 10pt}{\odrovnej\par\bigskip}
\newenvironment{veta}{\pagebreak[2]\noindent\textbf{V�ta}\par\noindent\leftskip 10pt}{\odrovnej\par\bigskip}
\newenvironment{vetaN}[1]{\pagebreak[2]\noindent\textbf{V�ta~}\emph{(#1)}\par\noindent\leftskip 10pt}{\odrovnej\par\bigskip}
\newenvironment{vetaSK}{\pagebreak[2]\noindent\textbf{Veta}\par\noindent\leftskip 10pt}{\odrovnej\par\bigskip}
\newenvironment{vetaSKN}[1]{\pagebreak[2]\noindent\textbf{Veta~}\emph{(#1)}\par\noindent\leftskip 10pt}{\odrovnej\par\bigskip}

\newenvironment{dusledek}{\pagebreak[2]\noindent\textbf{D�sledek}\par\noindent\leftskip 10pt}{\odrovnej\par\bigskip}
\newenvironment{dosledok}{\pagebreak[2]\noindent\textbf{D�sledok}\par\noindent\leftskip 10pt}{\odrovnej\par\bigskip}

\newenvironment{dokaz}{\pagebreak[2]\noindent\leftskip 10pt\textbf{D�kaz}\par\noindent\leftskip 10pt}{\odrovnej\par\bigskip}
\newenvironment{dukaz}{\pagebreak[2]\noindent\leftskip 10pt\textbf{D�kaz}\par\noindent\leftskip 10pt}{\odrovnej\par\bigskip}

\newenvironment{ideadukazu}{\pagebreak[2]\noindent\leftskip 10pt\textbf{Idea d�kazu}\par\noindent\leftskip 10pt}{\odrovnej\par\bigskip}


\newenvironment{priklad}{\pagebreak[2]\noindent\textbf{P��klad}\par\noindent\leftskip 10pt}{\odrovnej\par\bigskip}
\newenvironment{prikladN}[1]{\pagebreak[2]\noindent\textbf{P��klad~}\emph{(#1)}\par\noindent\leftskip 10pt}{\odrovnej\par\bigskip}

\newenvironment{prikladSK}{\pagebreak[2]\noindent\textbf{Pr�klad}\par\noindent\leftskip 10pt}{\odrovnej\par\bigskip}
\newenvironment{priklady}{\pagebreak[2]\noindent\textbf{P��klady}\par\noindent\leftskip 10pt}{\odrovnej\par\bigskip}
\newenvironment{prikladySK}{\pagebreak[2]\noindent\textbf{Pr�klady}\par\noindent\leftskip 10pt}{\odrovnej\par\bigskip}

\newenvironment{algoritmusN}[1]{\pagebreak[2]\noindent\textbf{Algoritmus~}\emph{(#1)}\par\noindent\leftskip 10pt}{\odrovnej\par\bigskip}
%obecne prostredie, ktore ma vyuzitie pri specialnych odstavcoch ako (uloha, algoritmus...) aby nevzniklo dalsich x prostredi
\newenvironment{obecne}[1]{\pagebreak[2]\noindent\textbf{#1}\par\noindent\leftskip 10pt}{\odrovnej\par\bigskip}

\newenvironment{report}{\pagebreak[2]\noindent\textbf{Report}\em\par\noindent\leftskip 10pt}{\par\bigskip}

%\newenvironment{reportN}[1]{\pagebreak[2]\noindent\textbf{Report~}\emph{(#1)}\emph\par\noindent\leftskip 10pt}{\odrovnej\par\bigskip}
\newenvironment{reportN}[1]{\pagebreak[2]\noindent\textbf{Report~}\emph{(#1)}\em\par\noindent\leftskip 10pt}{\odrovnej\par\bigskip}

\newenvironment{penumerate}{
\begin{enumerate}
  \setlength{\itemsep}{1pt}
  \setlength{\parskip}{0pt}
  \setlength{\parsep}{0pt}
  %\setlength{\topsep}{200pt}
  \setlength{\partopsep}{200pt}
}{\end{enumerate}}

\def\pismenka{\numberedlistdepth=2} %pouzit, ked clovek chce opismenkovany zoznam...

\newenvironment{pitemize}{
\begin{itemize}
  \setlength{\itemsep}{1pt}
  \setlength{\parskip}{0pt}
  \setlength{\parsep}{0pt}
}{\end{itemize}}

%\definecolor{gris}{gray}{0.95}
\newcommand{\ramcek}[2]{\begin{center}\fcolorbox{white}{gris}{\parbox{#1}{#2}}\end{center}\par}
 \clearpage
\title{\LARGE U�ebn� texty k st�tn� bakal��sk� zkou�ce \\ Spr�va po��ta�ov�ch syst�m� \\ S�t� a internetov� technologie}
\begin{document}
\maketitle
\newpage
\setcounter{section}{4}
\section{S�t� a internetov� technologie}
\begin{e}{Po�adavky}{0}{0}
\begin{pitemize}
\item Architektura ISO/OSI
\item Rodina protokolu TCP/IP (ARP, IPv4, IPv6, ICMP, UDP, TCP) - adresace, routing, fragmentace, spolehlivost, flow control, congestion control, NAT
\item Rozhran� BSD sockets
\item Spolehlivost - spojovan� a nespojovan� protokoly, typy, detekce a oprava chyb
\item Bezpe�nost - IPSec, principy fungov�n� AH, ESP, transport mode, tunnel mode, firewalls
\item Internetov� a intranetov� protokoly a technologie - DNS, SMTP, FTP, HTTP, NFS, HTML, XML, XSLT a jejich pou�it�.
\end{pitemize}
\end{e}
\subsection{Architektura ISO/OSI}

\subsubsection*{�vod}
\begin{e}{Definice}{0}{0}
\textbf{S�ov� model} je ucelen� p�edstava o tom, jak maj� b�t s�t� �e�eny (obsahuje: po�et vrstev, co m� kter� vrstva na starosti; neobsahuje: konkr�tn� p�edstavu jak kter� vrstva pln� sv� �koly - tedy konkr�tn� protokoly). P��kladem je \emph{referen�n� model ISO/OSI} (konkr�tn� protokoly vznikaly samostatn� a dodate�n�).
\textbf{S�ov� architektura} nav�c obsahuje konkr�tn� protokoly - napr. \emph{rodina protokol� TCP/IP}.
\end{e}

Referen�n� model ISO/OSI (International Standards Organization / Open Systems Interconnection) bol pokusom vytvori� univerz�lnu sie�ov� architekt�ru - ale skon�il ako sie�ov� model (bez protokolov). Poch�dza zo \uv{sveta spojov} - organiz�cie ISO, a bol \uv{ofici�lnym rie�en�m}, presadzovan�m \uv{org�nmi �t�tu}; dnes u� prakticky odp�san� - prehral v s�boji s TCP/IP. ISO/OSI bol reakciou na vznik propriet�rnych a uzavret�ch siet�. P�vodne mal model popisova� chovanie otvoren�ch syst�mov vo vn�tri aj medzi sebou, ale bolo od toho upusten� a nakoniec z modelu ostal len sie�ov� model (popis funkcionality vrstiev) a konkr�tne protokoly pre RM ISO/OSI boli vyv�jan� samostatne (a dodato�ne zara�ovan� do r�mca ISO/OSI).

Model vznikal maximalistick�m sp�sobom - obsahoval v�etko �o by mohlo by� v bud�cnosti potrebn�. V�aka rozsiahlosti �tandardu sa implementovali len jeho niektor� podmno�iny - ktor� neboli (v�dy) kompatibiln�. Vznikol GOSIP (Government OSI Profile) ur�uj�ci podmno�inu modelu, ktor� malo ma� implementovan� v�etko �t�tne sie�ov� vybavenie. Naproti tomu v�etk�mu TCP/IP vzniklo naopak - najprv navrhnut�m jednoduch�ho rie�enia, potom postupn�m obohacovan�m o nov� vlastnosti (tie boli zahrnut� a� po preuk�zan� \uv{�ivotaschopnosti}).

\subsubsection*{7 vrstev}
Krit�ri� pri n�vrhu vrstiev boli napr.: rovnomern� vy�a�enos� vrstiev, �o najmen�ie d�tov� toky medzi vrstvami, mo�nos� prevzia� u� existuj�ce �tandardy (X.25), odli�n� funkcie mali patri� do odli�n�ch vrstiev, funkcie na rovnakom stupni abstrakcie mali patri� do rovnakej vrstvy. Niektor� vrstvy z fin�lneho n�vrhu sa pou��vaj� m�lo (rela�n� a prezenta�n�), niektor� zase pr�li� (linkov� - rozpadla sa na 2 podvrstvy LLC+MAC).

\begin{center}
\begin{tabular}{|c|l|}
	\hline
	aplika�n� vrstva & vrstvy orientovan� na podporu aplikac�\\
	prezenta�n� vrstva &\\
	rela�n� vrstva &\\
	\hline
	transportn� vrstva & p�isp�sobovac� vrstva \\
	\hline
	s�ov� vrstva & vrstvy orientovan� na p�enos dat\\
	linkov� vrstva & \\
	fyzick� vrstva & \\
	\hline
\end{tabular}
\end{center}

\textbf{Fyzick� vrstva} sa zaober� prenosom bitov (k�dovanie, modul�cia, synchroniz�cia...) a pon�ka teda slu�by typu po�li a pr�jmi bit (pri�om neinterpretuje v�znam t�chto d�t). Pracuje sa tu s veli�inami ako je \emph{��rka p�sma}, \emph{modula�n� a prenosov� r�chlos�}.

\textbf{Linkov� vrstva} pren�a v�dy cel� bloky d�t (r�mce/frames), pou��va pritom fyzick� vrstvu a prenos v�dy funguje len k priamym susedom. M��e pracova� spo�ahlivo �i nespo�ahlivo, pr�padne poskytova� QoS/best effort. �alej zabezpe�uje riadenie toku - zaistenie toho, aby vysielaj�ci nezahltil pr�jemcu. Del� sa na dve podvrstvy - MAC (pr�stup k zdie�an�mu m�diu - rie�i konflikty pri viacn�sobnom pr�stupe k m�diu) a LLC (ostatn� �lohy).  

\textbf{Sie�ov� vrstva} pren�a pakety (packets) - fakticky ich vklad� do linkov�ch r�mcov. Zaru�uje doru�enie paketov a� ku kone�n�mu adres�tovi (tj. zabezpe�uje smerovanie). M��e pou��va� r�zne algoritmy smerovania - ne/adapt�vne, izolovan�, distribuovan�, centralizovan�... (v architekt�re TCP/IP je to IP vrstva)

\textbf{Transportn� vrstva} zabezpe�uje komunik�ciu medzi koncov�mi ��astn�kmi (end-to-end) a m��e meni� nespo�ahliv� charakter komunik�cie na spo�ahliv�, menej spo�ahliv� na viac spo�ahliv�, nespojovan� prenos na spojovan�... Pr�kladom s� napr. TCP a UDP. �al�ou �lohou je rozli�ovanie jednotliv�ch entit (na rozdiel od napr. sie�ovej vrstvy) v r�mci uzlov - procesy, d�mony, �lohy (rozli�uje sa zv��a nepriamo - napr. v TCP/IP pomocou portov).

\textbf{Rela�n� vrstva} zais�uje vedenie rel�ci� - �ifrovanie, synchroniz�ciu, podporu transakci�. Je to najkritizovanej�ia vrstva v ISO/OSI modele, v TCP/IP �plne ch�ba.

\textbf{Prezenta�n� vrstva} sl��i na konverziu d�t, aby obe strany interpretovali d�ta rovnako (napr. re�lne ��sla, r�zne k�dovanie textov). �alej m� na starosti konverziu d�t do form�tu, ktor� je mo�n� prenies�: napr. lineariz�cia viacrozmern�ch pol�, d�tov�ch �trukt�r; konverzia viacbajtov�ch polo�iek na jednotliv� byty (little vs. big endian). \emph{Pozn�mka}: Z�pis ��sla 1234H v Big endian je [12:34:--:--] (sun, motorola), v Little endian [--:--:34:12] (intel, amd, ethernet).

\textbf{Aplika�n� vrstva} mala p�vodne obsahova� aplik�cie - ale t�ch je ve�a a nebolo mo�n� ich �tandardizova�. Teraz teda obsahuje len \uv{jadro} aplik�ci� - tie, ktor� malo zmysel �tandardizova� (email a pod.). Ostatn� �asti aplik�ci� (GUI) boli vysunut� nad aplika�n� vrstvu.

\subsubsection*{Kritika}
Model ISO/OSI:
\begin{pitemize}
	\item je pr�li� zlo�it�, �a�kop�dny a obtia�ne implementovate�n�
	\item je pr�li� maximalistick�
	\item nere�pektuje po�iadavky a realitu be�nej praxe
	\item po��tal sk�r s roz�ahl�mi sie�ami ako s lok�lnymi
	\item niektor� �innosti (funkcie) zbyto�ne opakuje na ka�dej vrstve
	\item jednozna�ne uprednost�uje spo�ahliv� a spojovan� prenosov� slu�by (ale tie s� spojen� s ve�kou r�iou $\Rightarrow$ spo�ahlivos� si efekt�vnej�ie zabezpe�ia koncov� uzly)
\end{pitemize}

Mo�nos� nespo�ahliv�ho/nespojovan�ho spojenia bolo pridan� do �tandardu a� dodato�ne, napriek tomu bol porazen� architekt�rou TCP/IP. Pou��vaj� sa v�ak niektor� prevzat� prokoly - X.400 (elektronick� po�ta), X.500 (adres�rov� slu�by - od�ah�en�m vznikol �spe�n� protokol LDAP).

\subsection{Cosi}

Soubor \url{Rodina_protokolu_TCP_IP_(AR?P,_IPv4,_IPv6,_ICMP,_UDP,_TCP)_-_adresace,_routing,_fragmentace,_spolehlivost,_?flow_control,_congestion_control,_NAT.O!PS.tex}
 uplne chybel
\subsection{Rozhraní BSD Sockets}

\subsubsection*{Úvod}
\textbf{Berkeley (BSD) sockets} je rozhranie (API) na vyvíjanie aplikácií ktoré používajú medziprocesovú komunikáciu (napr. v rámci siete). De facto je to štandardná abstrakcia pre sieťové sockety. Primárnym jazykom tohto API je C, pre väčšinu ostatných však existujú podobné rozhrania.

BSD sockets je API umožňujúce komunikáciu medzi dvomi hostmi alebo procesmi na jednom počítači, používajúc koncepciu internetových socketov. Toto rozhranie je implicitné pre TCP/IP a je teda jednou zo základných technológií internetu. Programátori môžu využívať rozhrania socketov na troch úrovniach, najzákladnejšou z nich sú RAW sockety (aj keď túto úroveň sa využijú zväčša len na počítačoch implementujúcich technológie týkajúce sa už priamo internetu).

\subsubsection*{Hlavičkové súbory}
Berkeley sockets používajú viaceré hlavičkové súbory, okrem iného:
\begin{pitemize}
\item\textbf{sys/socket.h} Core BSD socket functions and data structures.
\item\textbf{netinet/in.h} AF\_INET and AF\_INET6 address families. Widely used on the Internet, these include IP addresses and TCP and UDP port numbers.
\item\textbf{sys/un.h} AF\_UNIX address family. Used for local communication between programs running on the same computer. Not used on networks.
\item\textbf{arpa/inet.h} Functions for manipulating numeric IP addresses.
\item\textbf{netdb.h} Functions for translating protocol names and host names into numeric addresses. Searches local data as well as DNS.
\end{pitemize}

\subsubsection*{TCP}
TCP poskytuje koncept spojenia. Proces vytvorí TCP socket pomocou volania socket() s parametrom PF\_INET(6) a SOCK\_STREAM.

\begin{obecne}{Server}
Vytvorenie jednoduchého TCP servera vyžaduje nasledujúce kroky:
\begin{pitemize}
\item Vytvorenie TCP socketu (pomocou volania \emph{socket()})
\item Pripojenie socketu na port, kde bude načúvať (\emph{bind()}; parametrami je sockaddr\_in štruktúra, v ktorej sa nastavuje sin\_family (AF\_INET-IPv4,\\AF\_INET6-IPv6) a sin\_port)
\item Pripravenie socketu na načúvanie na porte (\emph{listen()}).
\item Akceptovanie príchodzích pripojení pomocou \emph{accept()}. Táto funkcia blokuje volajúceho do príchodu pripojenia a vracia identifikátor príchodzieho spojenia, ktorý sa môže ďalej použiť. accept() je hneď možné volať na pôvodný identifikátor socketu na čakanie na ďalšie spojenia.
\item Komunikácia s klientom pomocou \emph{send()}, \emph{recv()} alebo \emph{read()} a \emph{write()}
\item Keď už socket nie je potrebný, je možné ho zavrieť pomocou \emph{close()}.
\end{pitemize}
\end{obecne}

\begin{obecne}{Klient}
Vytvorenie TCP klienta vyžaduje nasledujúce kroky:
\begin{pitemize}
\item Vytvorenie TCP socketu (pomocou volania \emph{socket()})
\item Pripojenie k serveru pomocou \emph{connect()}) (znovu sa používa štruktúra sockaddr\_in, vypĺňa sa sin\_family, sin\_port (ako pri serveri) + sin\_addr (adresa servera))
\item Komunikácia so serverom pomocou \emph{send()}, \emph{recv()} alebo \emph{read()} a \emph{write()}
\item Keď už socket nie je potrebný, je možné ho zavrieť pomocou \emph{close()}.
\end{pitemize}
\end{obecne}

\subsubsection*{UDP}
UDP je protokol bez spojenia (conectionless) a bez garancie doručenia správ. UDP balíky môžu (okrem správneho počtu/poradia) doraziť mimo poradia, môžu byť duplikované alebo nedoraziť ani raz. Vďaka minimálnym garanciám má UDP oproti TCP oveľa menšiu réžiu. Keďže tento protokol nevytvára spojenia, dáta sa prenášajú v datagramoch.

Adresovací priestor UDP (porty UDP) je úplne nezávislý na priestore portov TCP.

\begin{obecne}{Server}
Keďže sa nevytvárajú spojenia, po vytvorení socketu (ako pri TCP pomocou socket()+bind()) už aplikácia (server) rovno čaká príchodzie datagramy pomocou funkcie \emph{recvfrom()}. Na konci sa socket zatvára pomocou close().
\end{obecne}

\begin{obecne}{Klient}
U klienta je tiež oproti spojovanej verzii zjednodušenie - stačí vyrobiť socket (pomocou socket()) a potom už iba posielať datagramy pomocou \emph{sendto()}. Na konci sa socket zatvára pomocou close().
\end{obecne}

\subsubsection*{Najdôležitejšie funkcie}

\begin{pitemize}

\item \textbf{int socket(int domain, int type, int protocol)}
	\begin{pitemize}
		\item \emph{domain} (PF\_INET | PF\_INET6)
		\item \emph{type} (SOCK\_STREAM, SOCK\_DGRAM,\\SOCK\_SEQPACKET (spoľahlivé zoradené balíky),\\SOCK\_RAW (raw protokoly nad sieťovou vrstvou))
		\item \emph{protocol} (väčšinou IPPROTO\_IP, ďalšie sú v netinet/in.h)
	\end{pitemize}

	\item \textbf{struct hostent *gethostbyname(const char *name)\\
	struct hostent *gethostbyaddr(const void *addr, int len, int type)}
	\begin{pitemize}
		\item Vracia pointer na hostent štruktúru, ktorá popisuje internetového hosta zadaného pomocou mena alebo adresy (obsahuje buď informácie od name servera, alebo z lokálneho /etc/hosts súboru)...
	\end{pitemize}

	\item \textbf{int connect(int sockfd, const struct sockaddr *serv\_addr, socklen\_t addrlen)}
	\item \textbf{int bind(int sockfd, struct sockaddr *my\_addr, socklen\_t addrlen)}
	\item \textbf{int listen(int sockfd, int backlog)}
	\begin{pitemize}
		\item \emph{backlog} určuje maximálne koľko pripojení môže vo fronte čakať na akceptovanie...
	\end{pitemize}

	\item \textbf{int accept(int sockfd, struct sockaddr *cliaddr, socklen\_t *addrlen)}\\
	do \emph{cliaddr} sa vyplnia informácie o klientovi...
\end{pitemize}

\subsubsection*{Blokujúce a neblokujúce volania}
BSD sockety môžu fungovať v dvoch módoch - blokujúcich a neblokujúcich. V blokujúcom móde funkcie nevrátia riadenie programu, kým nie sú spracované všetky dáta - čo môže spôsobiť rôzne problémy (program \uv{zamrzne}, keď socket načúva; alebo keď socket čaká na dáta, ktoré neprichádzajú). Typicky sa nastavuje neblokujúci mód pomocou \emph{fcntl()} alebo \emph{ioctl()}

\subsection{Spolehlivost - spojovan� a nespojovan� protokoly, typy, detekce a oprava chyb}

\subsubsection*{Spolehlivost}

\textbf{Spolehlivost}:
\begin{pitemize}
	\item m��e b�t zaji�t�na na kter�koliv vrstv� (krom� fyzick�)
	\item TCP/IP �e�� na transportn� (TCP), ISO/OSI o�ek�v� spolehlivost na v�ech (po��naje linkovou)
	\item v�t�i re�ie, zpo�d�n� p�i chyb�ch 
\end{pitemize}

\textbf{Nespolehliv� komunikace}:
\begin{pitemize}
	\item men�� re�ie, lep�� odezva
	\item v�hodn� pro audio/video p�enosy, kde lze tolerovat ztr�ty 
\end{pitemize}

\subsubsection*{Spojovan� a nespojovan� protokoly}

\textbf{Spojovan� komunikace}: stavov�, virtu�ln� okruhy, navazov�n� a ukon�en� spojen�. Viz TCP.

\textbf{Nespojovan� komunikace}: zas�l�n� zpr�v, datagramy (UDP), nestavov�, bez navazov�n� a ukon�ov�n�.  Viz UDP.

\subsubsection*{Detekce a oprava chyb}
\begin{pitemize}
	\item schopnost poznat, �e do�lo k n�jak� chyb� p�i p�enosu
	\item Hammingovy k�dy - p��li� velk� redundance, nepou��van�
	\item potvrzov�n� (ACK) - viz TCP/IP
		\begin{pitemize}
			\item p��jemce si znovu nech� zaslat po�kozen�/nedoru�en� data
			\item podm�nkou existence zp�tn�ho kan�lu (alespo� half-duplex)
			\item jednotliv� vs. kontinu�ln�
			\item kladn� (ACK) a z�porn� (NAK)
			\item samostatn� vs. nesamostatn� (piggybacking)
			\item metoda ok�nka
			\item selektivn� opakov�n� vs. opakov�n� s n�vratem 
		\end{pitemize}
	\item parita - p���n�, pod�ln�
	\item kontroln� sou�ty
	\item cyklick� redundantn� sou�y (CRC)
	\item druhy chyb: pozm�n�n� data, shluky chyb, v�padky dat
	\item p�i chyb� nutno vy��dat si cel� r�mec znovu 
\end{pitemize}

\subsection{Bezpečnost -- IPSec, principy fungování AH, ESP, transport mode, tunnel mode, firewalls}

\subsubsection*{IPSec}

\begin{pitemize}
	\item Není to pouze jeden protokol ale soustava vzájemně provázaných opatření a dílčích protokolů pro zabezpečení komunikace pomocí IP protokolu, funguje na síťové vrstvě -- není závislý na protokolech vyšších vrstev jako je TCP a UDP (např. SSL protokol pracuje na transportní vrstvě)
	\item Podporováno jak v IPv4 (podpora nepovinná) i v IPv6 (podpora povinná)
	\item Zajišťuje důvěrnost (šifruje přenášená data) a integritu (data nejsou při přenosu změněna)
	\item několik desítek RFC dokumentů
	\item autentifikace -- ověření původu dat (odesílatele)
	\item kryptování -- šifrování komunikace (mimo IP hlavičky)
	\item může být implementováno na bráně (security gateway, lokální síť je považována za bezpečnou) nebo na koncových zařízení 

	\item \textbf{SA (Security Association)}
	\begin{pitemize}
		\item point-to-point bezpečnostní spoj (návrh uvažuje i o jiných variantách)
		\item pro každý směr a každý prototokol nutné mít vlastní SA spoj
	\end{pitemize}
\end{pitemize}

IPsec módy:
\begin{pitemize}
	\item \textbf{transport mode}
	\begin{pitemize}
		\item IP hlavička nechráněná (jeden z důvodů je užívání systému NAT), tělo paketu šifrováno (data vyšších protokolů)
		\item použitelné jen na koncových stanicích 
	\end{pitemize}
	\item \textbf{tunnel mode}
	\begin{pitemize}
		\item pakety jsou celé (včetně hlavičky) zašifrovány a vloženy do dalšího paketu, na druhé straně rozbaleny
		\item povinné pro security gateways, volitelné pro koncové stanice
		\item ve vnější IP hlavičce se jako příjemce uvádí security gateway na hranici cílové sítě 
	\end{pitemize}
\end{pitemize}

IPsec protokoly:
\begin{pitemize}
	\item \textbf{AH (Authentication Header)}
	\begin{pitemize}
		\item komunikující strany se dohodnou na klíči
		\item k datům se připojuje hash
		\item chrání také před replay attack
		\item provádí autentizaci a kontrolu změny dat, neprovádí šifrování 
	\end{pitemize}
	\item \textbf{ESP (Encapsulating Security Payload)}
	\begin{pitemize}
		\item provádí autentizaci a také šifruje obsah
		\item pro šifrování používá 3DES, Blowfish aj. (původně DES, již není považováno za bezpečné) 
	\end{pitemize}
\end{pitemize}

Dohoda klíčů:
\begin{pitemize}
	\item před použitím protokolu AH či ESP si musí strany dohodnout klíče
	\item manuální konfigurace
	\item automatická konfigurace -- IKE (Internet Key Exchange) protokol 
\end{pitemize}

\subsubsection*{Firewally}
\begin{pitemize}
	\item sledování a filtrování komunikace na síti
	\begin{pitemize}
		\item blokování -- zabraňuje neoprávněnému přístupu
		\item prostupnost -- propouštění povoleného toku 
	\end{pitemize}
	\item paketové filtry -- např. na routeru
	\item stavový firewall (stateful) -- sleduje vztahy mezi pakety, ohlíží se na historii
	\item na různých vrstvách
	\begin{pitemize}
		\item síťová -- pouze dle zdrojových a cílových adres a protokolu
		\item transportní -- také podle portů
		\item aplikační -- dle obsahu (dat) 
	\end{pitemize}
	\item demilitarizovaná zóna (DMZ):
	\begin{pitemize}
		\item jiné řešení bezpečnosti
		\item přístup ven pouze přes specializovaná zařízení (proxy, brány), nelze přímo -- platí pro oba směry 
	\end{pitemize}
\end{pitemize}

\input{informatika/siete_a_bezpecnost/Internetove_a_intranetove_protokoly_a_technologie_-_DNS,_SMTP,_FTP,_HTTP,_NFS,_HTML,_XML,_XSLT_a_jejich_pouziti.PS.tex}

\end{document}
