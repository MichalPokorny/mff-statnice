%&latex
\documentclass[a4paper]{article}

\frenchspacing

\usepackage[cp1250]{inputenc}
\usepackage[czech]{babel}


\usepackage{a4wide}
\usepackage{amsmath, amsthm, amssymb, amsfonts}
\usepackage[mathcal]{eucal}
\usepackage{graphicx}
\usepackage{url}
\usepackage{color}
\usepackage{wrapfig}
\usepackage{capt-of}
\usepackage{float}

\newenvironment{vetaSkull}{\pagebreak[2]\noindent\textbf{$\bigstar$ V�ta}\par\noindent\leftskip 10pt}{\odrovnej\par\bigskip}
\newenvironment{vetaNSkull}[1]{\pagebreak[2]\noindent\textbf{$\bigstar$ V�ta~}\emph{(#1)}\par\noindent\leftskip 10pt}{\odrovnej\par\bigskip}

% sirka a vyska textu nastavena jako papir, vsechny okraje vynulovany a pridano 20pt na kazdou stranu
% horizontalni rozmery
\setlength{\textwidth}{\paperwidth}
\addtolength{\textwidth}{-40pt}
\addtolength{\hoffset}{-1in}
\addtolength{\hoffset}{20pt}
\setlength{\oddsidemargin}{0in}
\setlength{\marginparsep}{0in}
% vertikalni rozmery
\setlength{\textheight}{\paperheight}
\addtolength{\textheight}{-60pt}
\addtolength{\voffset}{-1in}
\addtolength{\voffset}{20pt}
\setlength{\topmargin}{0in}
\setlength{\headheight}{0in}
\setlength{\headsep}{0in}


%Obrazek na miste
%pouziti
%%\obrazeknahore{adresa}{popisek}{label}
\long\def\obrazeknahore#1#2#3 {

\begin{figure}[t]
    \centering
    \includegraphics[width=0.8\textwidth]{#1}
    
    \caption{#2}
    \label{#3}
    
\end{figure}

}


%==========================================
%PEKELNA MAKRA NA ZAROVNANI OBRAZKU DOPRAVA

\makeatletter


%tohle je makro, ktere mi dovoluje obtekani i u kratkych environmentu
%ABSOLUTNE nechapu, jak to funguje, ale funguje to
%viz http://tex.stackexchange.com/questions/26078/ 
\def\odrovnej{\@@par
\ifnum\@@parshape=\z@ \let\WF@pspars\@empty \fi % reset `parshape'
\global\advance\c@WF@wrappedlines-\prevgraf \prevgraf\z@
\ifnum\c@WF@wrappedlines<\tw@ \WF@finale \fi}

\makeatother



%---
%makro, co da obrazek doprava a ostatni text ho obteka
%(bez toho predchazejiciho makra to ale poradne nebeha)
%pouziti:
%\obrazekvpravo{adresa}{popisek}{label}{procento sirky}
\long\def\obrazekvpravo#1#2#3#4{

\setlength\intextsep{-20pt}

    \begin{wrapfigure}{r}{#4\textwidth}
      \begin{center}
          \vspace{-10pt}
          
        \includegraphics[width=#4\textwidth]{#1}
        \vspace{-10pt}
        
      \end{center}
      
      \caption{#2}
      \label{#3}
      
      
    \end{wrapfigure}

\setlength\intextsep{0pt}

    
}




%---
%makro pro pripady, kdy wrapfigure neco mrsi
%je to docela pekelne
%je nutne mu dat jak text vpravo, tak text vlevo
%a nevim, jestli bude 100% fungovat, ale doufam, ze jo

%pouziti:
%\obrazekvpravominipage{adresa}{popisek}{label}{procento sirky}{1 - procento sirky}{text vlevo}
\long\def\obrazekvpravominipage#1#2#3#4#5#6{

\noindent\begin{minipage}{#5\linewidth}
\vspace{0pt}
#6
\end{minipage}
\hspace{0.5cm}
\noindent\begin{minipage}{#4\linewidth}
\vspace{0pt}
\centering
\includegraphics[width=0.9\textwidth]{#1}
\captionof{figure}{#2}
\label{#3}
\end{minipage}

}

%KONEC PEKELNYCH MAKER
%=====================


%Vacsina prostredi je dvojjazicne. V pripade, ze znenie napr pozorovania je pisane po slovensky, malo by byt po slovensky aj oznacenie.

\newenvironment{pozadavky}{\pagebreak[2]\noindent\textbf{Po�adavky}\par\noindent\leftskip 10pt}{\odrovnej\par\bigskip}
\newenvironment{poziadavky}{\pagebreak[2]\noindent\textbf{Po�iadavky}\par\noindent\leftskip 10pt}{\odrovnej\par\bigskip}

\newenvironment{definice}{\pagebreak[2]\noindent\textbf{Definice}\par\noindent\leftskip 10pt}{\odrovnej\par\bigskip}
\newenvironment{definiceN}[1]{\pagebreak[2]\noindent\textbf{Definice~}\emph{(#1)}\par\noindent\leftskip 10pt}{\odrovnej\par\bigskip}
\newenvironment{definicia}{\pagebreak[2]\noindent\textbf{Defin�cia}\par \noindent\leftskip 10pt}{\odrovnej\par\bigskip}
\newenvironment{definiciaN}[1]{\pagebreak[2]\noindent\textbf{Defin�cia~}\emph{(#1)}\par\noindent\leftskip 10pt}{\odrovnej\par\bigskip}

\newenvironment{pozorovani}{\pagebreak[2]\noindent\textbf{Pozorov�n�}\par\noindent\leftskip 10pt}{\odrovnej\par\bigskip}
\newenvironment{pozorovanie}{\pagebreak[2]\noindent\textbf{Pozorovanie}\par\noindent\leftskip 10pt}{\odrovnej\par\bigskip}
\newenvironment{poznamka}{\pagebreak[2]\noindent\textbf{Pozn�mka}\par\noindent\leftskip 10pt}{\odrovnej\par\bigskip}
\newenvironment{poznamkaN}[1]{\pagebreak[2]\noindent\textbf{Pozn�mka~}\emph{(#1)}\par\noindent\leftskip 10pt}{\odrovnej\par\bigskip}
\newenvironment{lemma}{\pagebreak[2]\noindent\textbf{Lemma}\par\noindent\leftskip 10pt}{\odrovnej\par\bigskip}
\newenvironment{lemmaN}[1]{\pagebreak[2]\noindent\textbf{Lemma~}\emph{(#1)}\par\noindent\leftskip 10pt}{\odrovnej\par\bigskip}
\newenvironment{veta}{\pagebreak[2]\noindent\textbf{V�ta}\par\noindent\leftskip 10pt}{\odrovnej\par\bigskip}
\newenvironment{vetaN}[1]{\pagebreak[2]\noindent\textbf{V�ta~}\emph{(#1)}\par\noindent\leftskip 10pt}{\odrovnej\par\bigskip}
\newenvironment{vetaSK}{\pagebreak[2]\noindent\textbf{Veta}\par\noindent\leftskip 10pt}{\odrovnej\par\bigskip}
\newenvironment{vetaSKN}[1]{\pagebreak[2]\noindent\textbf{Veta~}\emph{(#1)}\par\noindent\leftskip 10pt}{\odrovnej\par\bigskip}

\newenvironment{dusledek}{\pagebreak[2]\noindent\textbf{D�sledek}\par\noindent\leftskip 10pt}{\odrovnej\par\bigskip}
\newenvironment{dosledok}{\pagebreak[2]\noindent\textbf{D�sledok}\par\noindent\leftskip 10pt}{\odrovnej\par\bigskip}

\newenvironment{dokaz}{\pagebreak[2]\noindent\leftskip 10pt\textbf{D�kaz}\par\noindent\leftskip 10pt}{\odrovnej\par\bigskip}
\newenvironment{dukaz}{\pagebreak[2]\noindent\leftskip 10pt\textbf{D�kaz}\par\noindent\leftskip 10pt}{\odrovnej\par\bigskip}

\newenvironment{ideadukazu}{\pagebreak[2]\noindent\leftskip 10pt\textbf{Idea d�kazu}\par\noindent\leftskip 10pt}{\odrovnej\par\bigskip}


\newenvironment{priklad}{\pagebreak[2]\noindent\textbf{P��klad}\par\noindent\leftskip 10pt}{\odrovnej\par\bigskip}
\newenvironment{prikladN}[1]{\pagebreak[2]\noindent\textbf{P��klad~}\emph{(#1)}\par\noindent\leftskip 10pt}{\odrovnej\par\bigskip}

\newenvironment{prikladSK}{\pagebreak[2]\noindent\textbf{Pr�klad}\par\noindent\leftskip 10pt}{\odrovnej\par\bigskip}
\newenvironment{priklady}{\pagebreak[2]\noindent\textbf{P��klady}\par\noindent\leftskip 10pt}{\odrovnej\par\bigskip}
\newenvironment{prikladySK}{\pagebreak[2]\noindent\textbf{Pr�klady}\par\noindent\leftskip 10pt}{\odrovnej\par\bigskip}

\newenvironment{algoritmusN}[1]{\pagebreak[2]\noindent\textbf{Algoritmus~}\emph{(#1)}\par\noindent\leftskip 10pt}{\odrovnej\par\bigskip}
%obecne prostredie, ktore ma vyuzitie pri specialnych odstavcoch ako (uloha, algoritmus...) aby nevzniklo dalsich x prostredi
\newenvironment{obecne}[1]{\pagebreak[2]\noindent\textbf{#1}\par\noindent\leftskip 10pt}{\odrovnej\par\bigskip}

\newenvironment{report}{\pagebreak[2]\noindent\textbf{Report}\em\par\noindent\leftskip 10pt}{\par\bigskip}

%\newenvironment{reportN}[1]{\pagebreak[2]\noindent\textbf{Report~}\emph{(#1)}\emph\par\noindent\leftskip 10pt}{\odrovnej\par\bigskip}
\newenvironment{reportN}[1]{\pagebreak[2]\noindent\textbf{Report~}\emph{(#1)}\em\par\noindent\leftskip 10pt}{\odrovnej\par\bigskip}

\newenvironment{penumerate}{
\begin{enumerate}
  \setlength{\itemsep}{1pt}
  \setlength{\parskip}{0pt}
  \setlength{\parsep}{0pt}
  %\setlength{\topsep}{200pt}
  \setlength{\partopsep}{200pt}
}{\end{enumerate}}

\def\pismenka{\numberedlistdepth=2} %pouzit, ked clovek chce opismenkovany zoznam...

\newenvironment{pitemize}{
\begin{itemize}
  \setlength{\itemsep}{1pt}
  \setlength{\parskip}{0pt}
  \setlength{\parsep}{0pt}
}{\end{itemize}}

%\definecolor{gris}{gray}{0.95}
\newcommand{\ramcek}[2]{\begin{center}\fcolorbox{white}{gris}{\parbox{#1}{#2}}\end{center}\par}


\title{\LARGE U�ebn� texty k st�tn� bakal��sk� zkou�ce \\ Matematika \\ ��sla}

\begin{document}

%\maketitle

%\newpage
\setcounter{section}{0}
\section{��sla}

\begin{pozadavky}
\begin{pitemize}
\item Vlastnosti p�irozen�ch, cel�ch, racion�ln�ch, re�ln�ch a komplexn�ch ��sel
\item Posloupnosti a limity
\item Cauchyovsk� posloupnosti
\end{pitemize}
\end{pozadavky}

\subsection{Re�ln� ��sla}
$\mathbb{R}$ definujeme axiomaticky.

\begin{definice}
Mno�inou v�ech re�ln�ch ��sel (zna��me ji $\mathbb{R}$) budeme rozum�t mno�inu, na n� je definov�no s��t�n� (zna��me $x+y$), n�soben� (zna��me $x\cdot y$) a uspo��d�n� (zna��me $x \le y$), kter� sp�uj� tyto axiomy:

\begin{penumerate}
	\item (Algebraick� operace)
	\begin{penumerate}
		\item $\forall x,y,z \in \mathbb{R} : x+(y+z) = (x+y)+z \hfill\textit{(asociativn� z�kon pro s��t�n�)}$
		\item $\forall x,y \in \mathbb{R} : x+y = y+x \hfill\textit{(komutativn� z�kon pro s��t�n�)}$
		\item v $\mathbb{R}$ existuje nulov� prvek (zna��me ho 0) tak, �e $\forall x \in \mathbb{R}: x+0 = x$
		\item pro ka�d� $x \in \mathbb{R}$ existuje opa�n� prvek (zna��me ho $-x$) tak, �e $x+(-x)=0$
		\item $\forall x,y,z \in \mathbb{R}: x\cdot (y\cdot z)=(x\cdot y)\cdot z \hfill\textit{(asociativn� z�kon pro n�soben�)}$
		\item $\forall x,y \in \mathbb{R} : x\cdot y = y\cdot x \hfill\textit{(komutativn� z�kon pro n�soben�)}$
		\item v $\mathbb{R}$ existuje jednotkov� prvek (zna��me ho 1) tak, �e $\forall x \in \mathbb{R}: x\cdot 1=x$
		\item pro ka�d� $x \in \mathbb{R}, x \neq 0$ existuje inverzn� prvek (zna��me ho $x^{-1}$) tak, �e $x\cdot x^{-1}=1$
		\item $\forall x,y,z \in \mathbb{R}: (x+y)\cdot z = x\cdot z + y\cdot z \hfill\textit{(distributivn� z�kon)}$
	\end{penumerate}
	\item (Uspo��d�n�)
	\begin{penumerate}
		\item $\forall x,y,z \in \mathbb{R}: ((x \le y) \wedge (y \le z)) \Rightarrow (x \le z) \hfill\textit{(tranzitivita)}$
		\item $\forall x,y \in \mathbb{R}: (((x \le y) \wedge (y \le x)) \Rightarrow (x=y) \hfill\textit{(slab� antisymetrie)}$
		\item $\forall x,y \in \mathbb{R}: (x \le y) \vee (y \le x)\hfill\textit{(linearita)}$
		\item $\forall x,y,z \in \mathbb{R}: (x \le y) \Rightarrow (x+z \le y+z)$
		\item $\forall x,y,z \in \mathbb{R}: (x \le y) \wedge (0 \le z) \Rightarrow (x\cdot z \le y\cdot z)$
	\end{penumerate}
	\item (Netrivialita)
	\begin{penumerate}
		\item $0 \neq 1$
	\end{penumerate}
	\item (�plnost)
		\\\begin{definiceN}{Axiom suprema}
		Nech� $M \subset \mathbb{R}$ je nepr�zdn� a shora omezen� (tj. $\exists a \forall x \in M: x \le a$). Pak existuje ��slo $s \in \mathbb{R}$, kter� m� vlastnosti:
		\begin{penumerate}
			\item $\forall x \in M: x \le s\hfill\textit{(s je horn� omezen�)}$
			\item $\forall s' \in \mathbb{R}, s' < s\;\; \exists x \in M: x > s'\hfill\textit{(v�echny men�� nejsou horn� omeznen�)}$
		\end{penumerate}
		
		\end{definiceN}
\end{penumerate}

\end{definice}
\begin{poznamka}
Axiom supr�ma je d�le�it�, odli�uje re�ln� ��sla od racion�ln�ch.


		��slo $s$ z~axiomu suprema je jednozna�n� ur�eno, zna�� se $\sup M$ a ��k� se mu \textbf{supremum mno�iny $M$}. Supremum mno�iny je jej� nejmen�� horn� z�vora (\textbf{horn� z�vora} nebo \textbf{horn� mez} je ka�d� takov� prvek, pro kter� plat� bod (a) definice suprema). Nejv�t�� doln� z�voru mno�iny $M$ naz�v�me \textbf{infimem mno�iny $M$} a zna��me $\inf M$; z~axiomu suprema plyne, �e ka�d� zdola omezen� mno�ina m� infimum.
\end{poznamka}

\begin{poznamka}
Relace \uv{$<$} na re�ln�ch ��slech (a stejn� tak na p�irozen�ch a racion�ln�ch ��slech) se definuje takto: $a<b$, pr�v� kdy� $a\leq b$ a z�rove� $a\neq b$.
\end{poznamka}

\begin{definice}
     $\mathbb{R}$ lze sestrojit i tak, �e vezmeme $\mathbb{Q}$ a $\mathbb{R}$ definujeme z~nich -- tento p��stup jsem form�ln� popsal zde, ale nejsem si jist, jestli bude nutn� u zkou�ek -- nejsp� ne a je tu sp� pro �plnost. Dal��, ekvivalentn� p��stup jsou tzv. Dedekindovy �ezy, skrze kter� byly dokonce re�ln� ��sla p�vodn� vytvo�ena\footnote{v�ce info nap�. tady -- \url{http://en.wikipedia.org/wiki/Dedekind_cut} -- op�t, nep�edpokl�d�m, �e by bylo nutn� zn�t.}.
     
Nech� $R$ je mno�ina v�ech Cauchyovsk�ch posloupnost� ve $\mathbb{Q}$. Operace na t�to mno�in� definujeme n�sledovn�:
\begin{itemize}
	\item $\left\{x_n\right\} + \left\{y_n\right\} = \left\{x_n + y_n\right\}$
	\item $\left\{x_n\right\} \cdot \left\{y_n\right\} = \left\{x_n \cdot y_n\right\}$
	\item $\left\{x_n\right\} \ge \left\{y_n\right\}$, pr�v� kdy� $\exists n_0\;\; \forall n>n_0: x_n\ge y_n$
\end{itemize}
Na $R$ definujme relaci ekvivalence jako $\left\{x_n\right\} \sim \left\{y_n\right\}  \Leftrightarrow \lim_{k\rightarrow \infty}\left|x_k-y_k\right|=0$ (je nutn� je�t� dok�zat, �e jde o ekvivalenci). Potom re�ln� ��sla $\mathbb{R} = R / \sim$. 

\medskip\begin{poznamka}
Na takto definovan�ch re�ln�ch ��slech lze potom racion�ln� ��sla vz�t jako t��dy ekvivalence konstantn�ch posloupnost� ($x_0 = x_n = q \in \mathbb{Q}$), cel� ��sla potom jako t��dy ekvivalence posloupnost�, kde $x_0 = x_n = n \in \mathbb{N}$.
\end{poznamka}

\medskip\begin{poznamka}
O co p�esn� jde lze dob�e ilustrovat na desetinn�ch rozvoj�ch dan�ho re�ln�ho ��sla: nap�. pro ��slo $\pi$ m��e j�t nap�. o posloupnost $3$, $3,1$, $3,14$, $3,141$, $3,1415$\ldots, ale tak� nap��klad $3$, $3,14$, $3,1415$\ldots. Jde o r�zn� posloupnosti, ob� ale reprezentuj� stejn� re�ln� ��slo $\pi$.

Stejn� je nap��klad ��slo $0,\overline{9}$ rovno ��slu $1$ -- pokud vezmeme $\left\{x_n\right\}$ posloupnost $0,9$, $0,99$\ldots a $\left\{y_n\right\}$ konstantn� rovnou 1, jdou rozd�ly k nule, jsou tedy ve stejn� t��d� ekvivalence, tedy jsou si rovny.
\end{poznamka}


\end{definice}

\begin{poznamka}
    Jak jsem ji� psal, $\mathbb{R}$ se od $\mathbb{Q}$ li�� pouze t�m, �e m� ka�d� shora omezen� mno�ina supremum. Z~tohoto axiomu plyne, �e v $\mathbb{R}$ m� \emph{ka�d� Cauchyovsk� posloupnost limitu}\footnote{definice Cauchyovsk�ch posloupnost� bude d�le}, co� u $\mathbb{Q}$ neplat�.
\end{poznamka}


\subsection{P�irozen� ��sla}

\begin{definice}
	�ekneme, �e mno�ina $S \subset \mathbb{R}$ je \emph{induktivn�}, jestli�e plat�
	\begin{pitemize}
		\item $1 \in S$
		\item $x \in S \Rightarrow (x+1) \in S$
	\end{pitemize}
	Mno�inu \emph{p�irozen�ch ��sel $\mathbb{N}$} definujeme jako pr�nik v�ech induktivn�ch podmno�in $\mathbb{R}$, tedy
	$$
		\mathbb{N} := \bigcap \{ S; S \subset \mathbb{R}; \textit{S induktivn�}\}
	$$
\end{definice}



\begin{vetaN}{Induktivnost p�irozen�ch ��sel}
	Mno�ina $\mathbb{N}$ je induktivn�.
	
	\medskip
	\begin{dukaz}
	    1 je tam proto, �e je ve v�ech. Pokud je n�jak� ��slo $x$ z~$\mathbb{N}$ ve v�ech, tak i $x+1$ je ve v�ech.
    \end{dukaz}
\end{vetaN}


\begin{vetaN}{Slab� indukce}
    Pro vlastnost $\varphi$ plat�:
	\begin{penumerate}
		\item Plat�-li $\varphi(1)$ a $\varphi(n)\Rightarrow\varphi(n+1)$, plat� $\varphi$ pro v�echny $\mathbb{N}$ \hfill \emph{(slab� indukce)}
	\end{penumerate}
	
	\medskip
	\begin{ideadukazu}
	    Trivi�ln� z definice.
    \end{ideadukazu}
\end{vetaN}


\begin{vetaN}{Vlastnosti p�irozen�ch ��sel}
	Mno�ina $\mathbb{N}$ m� nasleduj�c� vlastnosti:
	\begin{penumerate}
		\item $n \in \mathbb{N} \Rightarrow n \ge 1$
		\item $n \in \mathbb{N} \backslash \{1\} \Rightarrow \exists m \in \mathbb{N}: n=m+1$
		\item $m,n \in \mathbb{N}, m < n \Rightarrow m + 1 \le n$ %p�ed pravou stranou \exists byt nemusi -- Tuetschek
		\item ka�d� nepr�zdn� podmno�ina $\mathbb{N}$ m� nejmen�� prvek
		\item Plat�-li $\varphi(1)$ a $\left(\forall k \leq n, k \in\mathbb{N} : \varphi(k)\right)\Rightarrow\varphi(n+1)$, plat� $\varphi$ pro v�echny $\mathbb{N}$ \hfill \emph{(siln� indukce)}
	\end{penumerate}
	
	\medskip
	\begin{ideadukazu}
	    5. se dok�e sporem a t�m, �e \emph{kone�n�} mno�ina prvk� m� v�dy nejmen�� prvek; 4. se dok�e p�es silnou indukci; 1., 2. a 3. bych dokazoval p�es 4. (ale je mo�n�, �e by to byl d�kaz kruhem).
    \end{ideadukazu}
\end{vetaN}

\begin{vetaN}{Archim�dova vlastnost re�ln�ch ��sel}
	Pro ka�d� $x \in \mathbb{R}$ existuje $n \in \mathbb{N}$ takov�, �e $x < n$.
	
	
	\medskip
	\begin{ideadukazu}
	    Sporem -- vybereme infimum $x$ mno�iny t�ch, pro kter� neexistuje (je zdola omezen� 1); pro $x-1$ u� mus� existovat, tak�e existuje i pro $x$, co� vede ke sporu.
    \end{ideadukazu}
\end{vetaN}


\begin{definiceN}{Peanove axi�my pre zavedenie prirodzen�ch ��sel}
    Op�t m��eme $\mathbb{N}$ zav�st jinak. Tato a n�sleduj�c� definice jsou ekvivalentn�.
    
Mno�ina $\mathbb{N}$ je takov� mno�ina, pro kterou plat�:
\begin{pitemize}
	\item Existuje ��slo 0 (to neznamen�, �e nula je prirodzen� ��slo, v $\mathbb{N}$ roli tejto nuly hraje jednotka).
	\item Na mno�ine prirodzen�ch ��sel je definovan� un�rna oper�cia \uv{nasledovn�k}, ozna�ovan� S.
	\item Neexistuje �iadne prirodzen� ��slo, ktor�ho nasledovn�kom je 0.
	\item R�zne prirodzen� ��sla maj� r�znych nasledovn�kov: $a \neq b \Rightarrow S(a) \neq S(b)$ (t.j. funkcia nasledovn�ka je prost�).
	\item Ak ��slo 0 sp��a nejak� vlastnos� a s��asne ju sp��a ka�d� nasledovn�k prirodzen�ho ��sla, potom t�to vlastnos� sp��aj� v�etky prirodzen� ��sla (\emph{axi�m matematickej indukcie}).
\end{pitemize}
\end{definiceN}

\begin{definiceN}{Kon�trukcia prirodzen�ch ��sel zalo�en� na te�rii mno��n}
	Ozna�me $0 := \{ \}$ a definujme $S(a) = a \cup \{a\}$ pre v�etky a. Mno�ina prirodzen�ch ��sel je potom definovan� ako prienik v�etk�ch mno��n obsahuj�cich $0$, ktor� s� uzavret� vzh�adom na funkciu nasledovn�ka. Predpokladaj�c platnost axi�mu nekone�nosti, d� sa dok�za�, �e t�to defin�cia sp��a Peanove axi�my. \emph{Axi�m nekone�nosti} vyzer� takto:
$$\exists\mathbb{N}:\emptyset\in\mathbb{N}\wedge(\forall x:x\in \mathbb{N}\Rightarrow x\cup\{x\}\in\mathbb{N})$$
	
	V \uv{klasickom} z�pise ��sel potom ka�d�mu prirodzen�mu ��slu zodpoved� mno�ina prirodzen�ch ��sel men��ch ako ono samo, tak�e
	\begin{pitemize}
		\item $0=\big\{\big\}$
		\item $1 = \big\{0\big\} = \big\{\left\{ \right\}\big\}$
		\item $2 = \big\{0,1\big\} = \big\{0, \left\{0\right\}\big\} = \big\{\left\{ \right\}, \left\{\left\{ \right\}\right\}\big\}$
		\item $3 = \big\{0,1,2\big\} = \big\{0, \left\{0\right\}, \left\{0, \left\{0\right\}\right\}\big\} = \big\{\left\{ \right\}, \left\{\left\{ \right\}\right\}, \left\{\left\{ \right\}, \left\{\left\{ \right\}\right\}\right\}\big\}$
	\end{pitemize}

\end{definiceN}

\begin{poznamka}
$\mathbb{N}$ je uzavret� na s��tanie.
\end{poznamka}

\subsection{Cel� a racion�ln� ��sla}
\begin{definice}
	Krom� symbol� $\mathbb{R}$ a $\mathbb{N}$, kter� jsme ji� zavedli, budeme zna�it symbolem $\mathbb{Z}$ mno�inu \emph{cel�ch ��sel}, tedy
	$$\mathbb{Z} = \mathbb{N} \cup \{0\} \cup \{-n, n \in \mathbb{N}\}$$
	a symbolem $\mathbb{Q}$ mno�inu \emph{racion�ln�ch ��sel}, tedy
	$$\mathbb{Q} = \left\{ \frac{p}{q}, p \in \mathbb{Z}, q \in \mathbb{N} \right\}$$
	
	$\mathbb{Z}$ je uzav�ena na s��t�n�, od��t�n� a n�soben�,
	$\mathbb{Q}$ je uzav�ena na s��t�n�, od��t�n�, n�soben� a d�len� nenulov�m ��slem.
	
	Mno�in� $\mathbb{R} \setminus \mathbb{Q}$ se tak� ��k� \emph{iracion�ln�}.
\end{definice}

\begin{vetaN}{Existence cel� ��sti}
	Pro ka�d� $x \in \mathbb{R}$ existuje pr�v� jedno ��slo $[x] \in \mathbb{Z}$ spl�uj�c�
	$$x-1 < [x] \le x$$
	Toto ��slo naz�v�me \emph{doln� celou ��st� ��sla x}.
	
	
	
	\medskip
	\begin{ideadukazu}
	    Sporem -- pro kladn� vybereme infimum $x$ mno�iny t�ch, pro kter� neexistuje (je zdola omezen� 0); pro $x-1$ u� mus� existovat, tak�e existuje i pro $x$, co� vede ke sporu. Se supremem podobn�.
    \end{ideadukazu}
\end{vetaN}

\begin{vetaN}{Hustota $\mathbb{Q}$ a $\mathbb{R} \backslash \mathbb{Q}$}
	Nech� $a,b \in \mathbb{R}, a < b$. Pak existuj� $q \in \mathbb{Q}$ a $r \in \mathbb{R} \backslash \mathbb{Q}$ takov�, �e
	$$a < q < b,\ a < r < b$$
	
	\medskip
	\begin{ideadukazu}
	    V�bec nev�m, jestli jde o korektn� d�kaz, ale j� bych dokazoval \uv{konstruk�n�} - pokud jsou $a$ nebo $b$ v $\mathbb{R} \backslash \mathbb{Q}$, vzal bych $q$ jako $(a+b)/2$ a $r$ jako tak dlouh� desetinn� rozvoj $q$, aby $a<r<b$. Pokud jsou $a$ i $b$ v $\mathbb{Q}$, vzal bych $q$ jako $a+\frac{b-a}{\sqrt{2}}$, co� by m�lo b�t iracion�ln� ��slo, a pro $r$ ud�lal op�t tot�. Ale nev�m, jestli t�eba existence odmocniny n�hodou nez�vis� na t�to v�t� :)
	\end{ideadukazu} 
\end{vetaN}

\begin{vetaN}{o existenci n-t� odmocniny}
	Nech� $x \in \mathbb{R}, x \ge 0$ a nech� $n \in \mathbb{N}$. Pak existuje pr�v� jedno $y \in \mathbb{R}, y \ge 0$ takov�, �e $y^n = x$.
\end{vetaN}

\subsection{Komplexn� ��sla}
\begin{definice}
Komplexn�m ��slem nazveme ��slo tvaru $a + bi$, kde $a,b \in \mathbb{R}$ naz�v�me \textbf{re�lnou a imagin�rn� ��st�} komplexn�ho ��sla. Tento tvar komplexn�ho ��sla se naz�v� \textbf{algebraick�}. P�smeno $i$ zna�� \textbf{imagin�rn� jednotku}, kter� se form�ln� zav�d� jako ��slo spl�uj�c� rovnici $i^2 +1 = 0$ tj. jako $\sqrt{-1}$ , kter� v re�ln�ch ��slech neexistuje.

Pokud je $b = 0$, je doty�n� ��slo re�ln�m ��slem, tj. re�ln� ��sla tvo�� podmno�inu ��sel komplexn�ch. Pokud je $a = 0$, mluv�me o \textbf{ryze imagin�rn�m ��sle}.
\end{definice}

\subsubsection{Aritmetika}
Aritmetika je na $\mathbb{C}$ definov�na n�sledovn�:
\begin{itemize}
\item$(a + ib) + (c+ id) = (a+c) + i(b+d)$
\item$(a + ib) - (c+ id) = (a-c) + i(b-d)$
\item$(a + ib) \cdot (c+ id) = (ac-bd) + i(ad+bc)$

\item$\frac{a+ib}{c+id}  = \frac{(a+ib)(c-id)}{(c+id)(c-id)} = \frac{(ac+bd) + i(bc-ad)}{c^2+d^2} = \left( \frac{ac+bd}{c^2+d^2} \right) + i \left( \frac{bc-ad}{c^2+d^2} \right)$
\end{itemize}

\obrazekvpravo{matematika/obrazky/01-dia1}{Komplexn� ��slo 2D}{fig:ComplexNum2D}{0.25}

Pojmem \textbf{komplexn� sdru�en� ��slo} komplexn�ho ��sla $z = a + ib$ se naz�v� ��slo: $$\bar{z} = a - ib$$

\textbf{Absolutn� hodnotou} (tak� \textbf{modul}) komplexn�ho ��sla $z = a + bi$ se naz�v�:
$$|z| = \sqrt{a^2 + b^2} = \sqrt{z\cdot \bar{z}}$$

Komplexn� ��sla se zobrazuj� v \textbf{komplexn� (Gaussov�) rovin�} jako body se sou�adnicemi $[x,y]$; $x$ je re�ln� ��st komplexn�ho ��sla, $y$ imagin�rn� ��st. Na ose $x$ le�� re�ln� ��sla, na ose $y$ ryze imagin�rn� ��sla.

N�sleduj�c� vlastnosti plat� pro v�echna komplexn� ��sla $z$ a $w$, nen�-li uvedeno jinak.

\begin{itemize}
\item$\overline{z + w} = \overline{z} + \overline{w}$

\item$\overline{zw} = \overline{z}\ \overline{w}$

\item$\overline{\left({\frac{z}{w}}\right)} = \frac{\overline{z}}{\overline{w}},$ pro $w$ nenulov�

\item $\overline{z} = z,$ pr�v� kdy� je $z$ re�ln� ��slo

\item $\left| \overline{z} \right| = \left| z \right|$

\item ${\left| z \right|}^2 = z\overline{z}$

\item $z^{-1} = \frac{\overline{z}}{{\left| z \right|}^2},$ pro $z$ nenulov�
\end{itemize}


Ka�d� komplexn� ��slo $z$ r�zn� od nuly je mo�n� jednozna�n� vyj�d�it v \textbf{goniometrick�m tvaru}. 


Pokud si v komplexn� rovin� zvol�me pol�rn� sou�adnicov� syst�m, vzd�lenost ��sla $z$ od po��tku je pr�v� jeho absolutn� hodnota $|z|$ a orientovan� �hel $\varphi = |\sphericalangle JOZ|$ (\emph{argument}), kde $J$ je bod $J[1,0]$, $O$ je po��tkem soustavy a $Z$ je obraz komplexn�ho ��sla $z=a + bi$ se sou�adnicemi $Z[a,b]$, plat�:
$$z = \left|z\right| \left(\cos \varphi + i\cdot \sin \varphi\right)$$

Argument $\varphi$ lze vyj�d�it ze vztah�: $\cos \varphi = \frac{a}{|z|}$ a $\sin \varphi = \frac{b}{|z|}$

Pro d�len� komplexn�ch ��sel $z_1 = |z_1|\cdot (\cos\varphi_1 + i\cdot \sin
\varphi_1)$ a $z_2 = |z_2|\cdot (\cos \varphi_2 + i\cdot \sin \varphi_2)$ plat� n�sleduj�c� rovnice:
$$\frac{z_1}{z_2} = \frac{|z_1|}{|z_2|}\cdot \left( \cos(\varphi_1 - \varphi_2) + i\cdot \sin(\varphi_1 - \varphi_2) \right)$$

Pro n�soben� komplexn�ch ��sel $z_1$ a $z_2$ z~p�edchoz�ho p��kladu slou�� vzorec:
$$z_1 \cdot  z_2 = |z_1|\cdot |z_2|\cdot \left( \cos(\varphi_1 + \varphi_2) + i\cdot \sin(\varphi_1 + \varphi_2) \right)$$

Pro \textbf{n-tou mocninu komplexn� ��sla} v goniometrick�m tvaru plat� tzv. \textbf{Moivreova v�ta}:
$$z^n = |z|^n (\cos n \varphi + i\cdot \sin n \varphi)$$

Odmocnina nen� v $\mathbb{C}$ jednozna�n� -- nap�. $i^2 = (-i)^2 = i$, nebo nap�. $i^3 = \left(-\frac{\sqrt{3}}{2}-\frac{i}{2}
\right)^3 = \left(\frac{\sqrt{3}}{2}-\frac{i}{2}
\right)^3 = -i$.

\subsection{Posloupnosti a limity}
\begin{definiceN}{posloupnost}
\emph{Posloupnost� re�ln�ch ��sel} naz�v�me jak�koli zobrazen� mno�iny $\mathbb{N}$ do mno�iny $\mathbb{R}$. Posloupnost obvykle zna��me symbolem $\{a_n\}^{\infty}_{n=1}$ nebo $\{a_n\}_{n \in \mathbb{N}}$. Pro ka�d� konkr�tn� $n \in \mathbb{N}$ naz�v�me re�ln� ��slo $a_n$ \emph{n-t�m �lenem} posloupnosti $\{a_n\}$.
\end{definiceN}

\begin{definiceN}{Omezen� posloupnosti}
\begin{penumerate}
	\item Posloupnost $\{a_n\}$ je \emph{shora omezen�}, je-li $\{a_n; n \in \mathbb{N}\}$ shora omezen�.
	\item Posloupnost $\{a_n\}$ je \emph{zdola omezen�}, je-li $\{a_n; n \in \mathbb{N}\}$ zdola omezen�.
	\item Posloupnost $\{a_n\}$ je \emph{omezen�}, je-li zdola omezen� a shora omezen�.
\end{penumerate}
\end{definiceN}

\begin{definiceN}{Rostouc� a klesaj�c� posloupnosti}
\begin{penumerate}
	\item Posloupnost $\{a_n\}$ je \emph{klesaj�c�}, jestli�e $\forall n \in \mathbb{N}: a_n > a_{n+1}$.
	\item Posloupnost $\{a_n\}$ je \emph{rostouc�}, jestli�e $\forall n \in \mathbb{N}: a_n < a_{n+1}$.
	\item Posloupnost $\{a_n\}$ je \emph{neklesaj�c�}, jestli�e $\forall n \in \mathbb{N}: a_n \le a_{n+1}$.
	\item Posloupnost $\{a_n\}$ je \emph{nerostouc�}, jestli�e $\forall n \in \mathbb{N}: a_n \ge a_{n+1}$.
	\item Posloupnost $\{a_n\}$ je \emph{monot�nn�}, jestli�e je nerostouc� nebo neklesaj�c�.
	\item Posloupnost $\{a_n\}$ je \emph{ryze monot�nn�}, jestli�e je rostouc� nebo klesaj�c�.
\end{penumerate}
\end{definiceN}


\subsubsection{Vlastn� limity}




\begin{definice}
    
Nech� $\{a_n\}$ je posloupnost re�ln�ch ��sel a $A \in \mathbb{R}$. �ekneme, �e A je \emph{vlastn� limitou posloupnosti} $\{a_n\}$, jestli�e
$$\forall \varepsilon > 0\ \exists n_0 \in \mathbb{N}: \forall n \ge n_0, n \in \mathbb{N}: |a_n - A| < \varepsilon$$
zna��me
$$\lim_{n \rightarrow \infty} a_n = A$$
\end{definice}

\obrazekvpravo{matematika/obrazky/02-limita}{Ilustrace k definici vlastn� limity}{fig:limita_posloupnosti}{0.5}




\begin{poznamka}
Na obr�zku \ref{fig:limita_posloupnosti} jsem nakreslil posloupnost k osv�tlen� definice limity. Jde o posloupnost\footnote{konkr�tn� $a_0=1, a_n=\cos a_{n-1}$, ale nen� to d�le�it�}, kter� konverguje k ��slu $0.739085$, ozna�eno $A$. 

\end{poznamka}


\begin{definice}
Jestli�e existuje $A \in \mathbb{R}$ tak, �e $\lim_{n \rightarrow \infty}a_n = A$, pak ��k�me, �e posloupnost $\{a_n\}$ m� \emph{vlastn� limitu} nebo �e \emph{konverguje} (je \emph{konvergentn�}). V opa�n�m p��pad� ��k�me, �e posloupnost \emph{diverguje}.
\end{definice}

\begin{pozorovani}
Ne ka�d� posloupnost je konvergentn�. Nap��klad posloupnost 0,1,0,1,0,... nem� vlastn� limitu a podobn� posloupnost $\{2^n\}$ nem� vlastn� limitu.
\end{pozorovani}

\begin{priklady}
\begin{pitemize}
	\item $\lim_{n \rightarrow \infty} (\sqrt{n+1} - \sqrt{n}) = 0$
	\item $\lim_{n \rightarrow \infty} \sqrt[n]{n} = 1$
\end{pitemize}
\end{priklady}

\begin{poznamka}
N�sleduj�c� v�ty maj� pom�rn� lehk� d�kazy, kter� zde jenom nazna��m; pokud si je budete zkou�et, je dobr� si u nich uv�domit z�kladn� vlastnosti absolutn� hodnoty:
\begin{pitemize}
	\item $\left|a \cdot b\right|= \left|a\right| \cdot\left| b\right|$
	\item $\left|a + b\right| \leq \left|a\right| + \left| b\right|$
	\item $\left|a \right| \leq b\Leftrightarrow -b \leq a \leq b$
\end{pitemize}
\end{poznamka}

\begin{vetaN}{o jednozna�nosti limity posloupnosti}
Ka�d� posloupnost m� nejv��e jednu limitu.


\medskip
\begin{ideadukazu}
    Kdyby m�la dv�, zvol�me $\varepsilon$ men�� ne� vzd�lenost. Od v�t��ho z dvou $n_0$ (jedno pro ka�dou z~limit) by pak musela b�t z�rove� ve dvou disjunktn�ch intervalech -- spor.
\end{ideadukazu}


\end{vetaN}

\begin{vetaN}{o omezenosti konvergentn� posloupnosti}
Ka�d� konvergentn� posloupnost je omezen�.


\medskip
\begin{ideadukazu}
    Pro libovoln� zvolen� $\varepsilon$ je od n�jak�ho $n_0$ posloupnost omezen� $A+\varepsilon$ shora a $A-\varepsilon$ zdola a p�ed $n_0$ je jich kone�n� mnoho, tak�e m��eme vybrat nejv�t��/nejmen��.
\end{ideadukazu}

\end{vetaN}

\begin{definice}
Nech� $\{a_n\}_{n \in \mathbb{N}}$ je posloupnost re�ln�ch ��sel. �ekneme, �e posloupnost $\{b_k\}_{k \in \mathbb{N}}$ je \emph{vybran�} z~posloupnosti, neboli �e posloupnost $\{b_k\}_{k \in \mathbb{N}}$ je \emph{podposloupnost�} posloupnosti $\{a_n\}_{n \in \mathbb{N}}$, jestli�e existuje rostouc� posloupnost p�irozen�ch ��sel $\{n_k\}$ takov�, �e $b_k = a_{n_k}$ pro v�echna $k \in \mathbb{N}$.
\end{definice}

\begin{poznamka}
    Tahle definice je trochu nepr�hledn�, ale jenom ��k�, �e z~jedn� posloupnosti vybereme jinou.
\end{poznamka}

\begin{vetaN}{o limit� vybran� posloupnosti}
Nech� $\{a_n\}_{n \in \mathbb{N}}$ je posloupnost re�ln�ch ��sel a nech� $\lim_{n \rightarrow \infty} a_n = A$. Nech� posloupnost $\{b_k\}_{k \in \mathbb{N}}$ je vybran� z~posloupnosti $\{a_n\}_{n \in \mathbb{N}}$. Pak $\lim_{k \rightarrow \infty} b_k = A$.

\medskip
\begin{ideadukazu}
    Nem��e j�t jinam -- pokud je $n_0$ u $a_n$ v~intervalu, t�m sp� pak bude $b_n$ v intervalu.
\end{ideadukazu}

\end{vetaN}

\begin{vetaN}{o aritmetice limit}
Nech� $\{a_n\}_{n \in \mathbb{N}}$ a $\{b_n\}_{n \in \mathbb{N}}$ jsou dv� posloupnosti re�ln�ch ��sel a nech� $\lim_{n \rightarrow \infty} a_n = A \in \mathbb{R}$ a $\lim_{n \rightarrow \infty} b_n = B \in \mathbb{R}$. Pak plat�:
\begin{enumerate}
	\item $\lim_{n \rightarrow \infty} (a_n+b_n) = A+B$
	\item $\lim_{n \rightarrow \infty} (a_n \cdot  b_n) = A\cdot B$
	\item je-li $\forall n \in \mathbb{N}: b_n \neq 0$ a $B \neq 0$, pak $\lim_{n \rightarrow \infty} \frac{a_n}{b_n}=\frac{A}{B}$
\end{enumerate}


\medskip
\begin{ideadukazu}
    \begin{enumerate}
    \item Pro ka�d� $\varepsilon$ m��eme vz�t $\frac{\varepsilon}{2}$, pro $a$ i $b$ naj�t jejich $n_0$, od v�t��ho z nich je i sou�et ve $\varepsilon$
    
    \item V�cem�n� tot�, vezmeme $\sqrt{\varepsilon}$
    
    \item Dok�eme, �e $\frac{1}{b_n}$ jde k $\frac{1}{B}$. Jednoduchou �pravou m�me $$\left|\frac{1}{b_n}-\frac{1}{B}\right| = \left|\frac{1}{b_n}\right|\cdot\left|\frac{1}{B}\right|\cdot\left|b_n-B\right|.$$ 
    
    K omezen� $\left|\frac{1}{b_n}\right|<\left|\frac{2}{B}\right|$ provedeme dost zb�sil� trik\footnote{z Draho�ov�ch skript; pokud to n�kdo um�te l�p, tak napi�te na f�rum, ale pochybuji}. 
    
    $$\Big|\left|b_n\right| - \left|B\right|\Big| \leq \left|b_n-B\right|<\varepsilon,$$ tak�e $\left|b_n\right|\geq\left|B\right|-\varepsilon$.
    
    Pokud bychom vzali $\varepsilon<|B|$, je $\frac{1}{\left|B\right|-\varepsilon}>0$ a proto $$\left|\frac{1}{b_n}\right|\cdot\left|\frac{1}{B}\right|\cdot\left|b_n-B\right| \leq \left|\frac{1}{\left|B\right|-\varepsilon}\right|\cdot\left|\frac{1}{B}\right|\cdot\left|b_n-B\right|.$$
    
    Pokud bychom vzali $\varepsilon<\frac{\left|B\right|}{2}$, je $$\frac{1}{\left|B\right|-\varepsilon}<\frac{1}{\left|B\right|-\frac{\left|B\right|}{2}}=\frac{2}{\left|B\right|},$$ tak�e cel� rozd�l p�vodn�ch zlomk� $$\left|\frac{1}{b_n}-\frac{1}{B}\right| < \left|\frac{2}{B}\right|\cdot\left|\frac{1}{B}\right|\cdot\left|b_n-B\right|.$$
    
    D�le jasn�, proto�e $\left|\frac{2}{B}\right|\cdot\left|\frac{1}{B}\right|$ je konstantn�.
    
    \end{enumerate}
\end{ideadukazu}



\end{vetaN}

\begin{vetaN}{o limit� a uspo��d�n�}
Nech� $\{a_n\}_{n \in \mathbb{N}}$ a $\{b_n\}_{n \in \mathbb{N}}$ jsou dv� posloupnosti re�ln�ch ��sel a nech� $\lim_{n \rightarrow \infty} a_n = A \in \mathbb{R}$ a $\lim_{n \rightarrow \infty} b_n = B \in \mathbb{R}$. Pak plat�:
\begin{penumerate}
	\item Jestli�e $A < B$, potom $\exists n_0 \in \mathbb{N}\ \forall n > n_0: a_n < b_n$
	\item Jestli�e $\exists n_0 \in \mathbb{N}\ \forall n \ge n_0: a_n \ge b_n$, pak $A \ge B$
\end{penumerate}

\noindent Pozor, ostrost nerovnost� v tomto p��pad� je d�le�it�.

\medskip
\begin{ideadukazu}
    Provede se sporem a \uv{chyt�e} velk�mi epsilony.
\end{ideadukazu}
\end{vetaN}

\begin{vetaN}{o policajtech}
Nech� $\{a_n\}_{n \in \mathbb{N}}$, $\{b_n\}_{n \in \mathbb{N}}$ a $\{c_n\}_{n \in \mathbb{N}}$ jsou posloupnosti re�ln�ch ��sel, spl�uj�c�
\begin{penumerate}
	\item $\exists n_0 \in \mathbb{N}\ \forall n > n_0: a_n \le c_n \le b_n$
	\item $\lim_{n \rightarrow \infty}a_n = \lim b_n = A \in \mathbb{R}$
\end{penumerate}
Pak
$$
\lim c_n = A
$$


\medskip
\begin{ideadukazu}
    Pokud je v epsilonu v jedn� i v druh�, zvol�me v�t�� z $n_0$ a mus� tam b�t i ta prost�edn�.
\end{ideadukazu}

\end{vetaN}

\begin{vetaN}{o limit� sou�inu mizej�c� ($\lim=0$) a omezen� posloupnosti}
Nech� $\{a_n\}_{n \in \mathbb{N}}$ a $\{b_n\}_{n \in \mathbb{N}}$ jsou posloupnosti re�ln�ch ��sel, nech� je $\lim_{n \rightarrow \infty} a_n = 0$ a $\{b_n\}$ omezen�. Pak
$$
\lim_{n \rightarrow \infty}(a_n \cdot b_n) = 0
$$


\medskip
\begin{ideadukazu}
    U mizej�c� vezmeme $\frac{\varepsilon}{K}$, kde $K$ je horn� omezen� $b_n$.
\end{ideadukazu}

\end{vetaN}

\subsubsection{Nevlastn� limity}
\begin{definice}
�ekneme, �e posloupnost  $\{a_n\}$ m� \emph{nevlastn� limitu} $+ \infty$, jestli�e:
$$ \forall K \in \mathbb{R}\ \exists n_0 \in \mathbb{N}: \forall n \ge n_0, n \in \mathbb{N}: a_n \ge K $$

Obdobn� �ekneme, �e posloupnost  $\{a_n\}$ m� \emph{nevlastn� limitu} $- \infty$, jestli�e:
$$ \forall K \in \mathbb{R}\ \exists n_0 \in \mathbb{N}: \forall n \ge n_0, n \in \mathbb{N}: a_n \le K $$

M�-li posloupnost nevlastn� limitu, ��k�me o n�, �e diverguje, stejn� jako v p��pad�, �e ��dnou limitu nem�.
\end{definice}


\obrazekvpravominipage{matematika/obrazky/03-limita_nevlastni}{Ilustrace k definici nevlastn� limity}{fig:limita_nevlastni}{0.5}{0.5} {

\begin{poznamka}
Ilustrace k definici nevlastn� limity je na obr�zku \ref{fig:limita_nevlastni}.
\end{poznamka}


\begin{poznamka}
V�echny mo�n� situace jsou:

\noindent Limita posloupnosti:


\begin{pitemize}
	\item neexistuje
	\item existuje
	\begin{pitemize}
		\item vlastn� (= posloupnost konverguje)
		\item nevlastn�
		\begin{pitemize}
			\item $- \infty$
			\item $+ \infty$
		\end{pitemize}
	\end{pitemize}
\end{pitemize}

\end{poznamka}
}


\begin{definice}
Mno�inu $\mathbb{R}^* := \mathbb{R} \cup \{ + \infty, - \infty \}$ naz�v�me \emph{roz���enou re�lnou osou}.
\end{definice}

\begin{poznamka}
V�ty o jednozna�nosti limity, o limit� vybran� posloupnosti, o limit� a uspo��d�n� a o policajtech plat� v nezm�n�n� podob�, jestli�e p�ipust�me nevlastn� limity. V�ta o omezenosti konvergentn� posloupnosti zrejm� neplat� - nebo� je-li $\lim_{n \rightarrow \infty} a_n=\infty$ (nebo $- \infty$), pak posloupnost $\{a_n\}$ nen� omezen�; je ale omezen� \emph{zdola} u $\infty$ a \emph{shora} u $- \infty$. V�tu o aritmetice limit pro roz��renou osu uvedeme zvl᚝.
\end{poznamka}

\begin{vetaN}{o aritmetice limit pro nevlastn� limity}
Nech� $\{a_n\}_{n \in \mathbb{N}}$ a $\{b_n\}_{n \in \mathbb{N}}$ jsou dv� posloupnosti re�ln�ch ��sel a nech� $\lim_{n \rightarrow \infty} a_n = A \in \mathbb{R}^*$ a $\lim_{n \rightarrow \infty} b_n = B \in \mathbb{R}^*$. Pak plat�:
\begin{penumerate}
	\item $\lim_{n \rightarrow \infty} (a_n+b_n) = A+B \textit{, pokud je v�raz A+B definov�n}$
	\item $\lim_{n \rightarrow \infty} (a_n \cdot  b_n) = A\cdot B\textit{, pokud je v�raz }A\cdot B\textit{ definov�n}$
	\item je-li $\forall n \in \mathbb{N}: b_n \neq 0$ a $B \neq 0$, pak $\lim_{n \rightarrow \infty} \frac{a_n}{b_n}=\frac{A}{B} \textit{, pokud je v�raz } \frac{A}{B} \textit{ definov�n}$
\end{penumerate}


\medskip
\begin{ideadukazu}

\begin{penumerate}
	\item Pokud je $\left\{a_n\right\}$ v $\mathbb{R}$, je zdola omezen� $L$, budeme br�t (pokud $\left\{b_n\right\}$ jde k~$\infty$) $K-L$, podobn� u dal��ch variant.
	\item T�eba $\left\{a_n\right\}$ jde k~$A$ v $\mathbb{R}$ a $A>0$ a $\left\{b_n\right\}$ jde k~$\infty$, pro ostatn� se dok�e obdobn�. Pro $\alpha>0$ ud�l�me $\epsilon = A-\alpha$, z~toho plyne $a>\alpha$; $b_n$ ur�it� jde nad $\frac{L}{\alpha}$ pro libovoln� $L$ a tedy $a_nb_n>L$.
	
	\item Dok�eme, �e kdy� $b_n$ jde k $\infty$ tak $\frac{1}{b_n}$ jde k~$0$ -- pro $\epsilon$ plat� odn�kud $b_n>\frac{1}{\epsilon}$, ale je tot�, co $\frac{1}{b_n}<\varepsilon$.
	
\end{penumerate}
\end{ideadukazu}

\end{vetaN}

\begin{definiceN}{Supremum a infimum na roz���en� re�ln� ose}
\begin{pitemize}
	\item Nech� mno�ina $A \subset \mathbb{R}$ je shora neomezen�. Pak klademe $\sup A := +\infty$
	\item Nech� mno�ina $A \subset \mathbb{R}$ je zdola neomezen�. Pak klademe $\inf A := -\infty$
	\item Nech� $A=\emptyset$. Pak klademe $\sup A := -\infty$ a $inf A := +\infty$
\end{pitemize}
\end{definiceN}

\begin{poznamka}
Pr�zdn� mno�ina je jedin� mno�ina, jej� supremum je men�� ne� jej� infimum.
\end{poznamka}

\begin{vetaN}{o limit� pod�lu kladn� a mizej�c� posloupnosti}
Nech� $\{a_n\}_{n \in \mathbb{N}}$ a $\{b_n\}_{n \in \mathbb{N}}$ jsou posloupnosti re�ln�ch ��sel, nech� je $\lim_{n \rightarrow \infty} a_n = A \in \mathbb{R}^*, A>0$ a nech� $\lim_{n \rightarrow \infty}\{b_n\}=0$. Nech�
$$
	\exists n_0 \in \mathbb{N} \textit{ } \forall n \ge n_0, n \in \mathbb{N}: b_n>0
$$
Pak
$$
	\lim_{n \rightarrow \infty} \frac{a_n}{b_n} = + \infty
$$
\medskip
\begin{ideadukazu}
Jestli�e $K$ je horn� omezen�, pak m��eme vz�t epsilon u mizej�c� $\frac{1}{K}$, sou�in pak jde nad $K$.
	
\end{ideadukazu}

\end{vetaN}

\subsubsection{Monot�nn� posloupnosti}
\begin{vetaN}{o limit� monot�nn� posloupnosti}
Ka�d� monot�nn� posloupnost m� limitu.

\medskip\begin{ideadukazu}
Nap�. rostouc� posloupnost bu� nen� shora omezen� a m� tedy limitu $\infty$, nebo je shora omezen�, proto�e je ale z�rove� rostouc�, tak se k dan�mu horn�mu omezen� postupn� p�ibli�uje.
\end{ideadukazu}

\end{vetaN}

\begin{poznamka}
Je-li posloupnost neklesaj�c� (nerostouc�) a nav�c shora (zdola) omezen�, pak m� vlastn� limitu. Tato limita se nav�c rovn� supremu (infimu) hodnot.
Je-li posloupnost neklesaj�c� (nerostouc�) a nav�c shora (zdola) neomezen�, pak m� limitu $+ \infty$ ($- \infty$).
\end{poznamka}

\begin{definiceN}{Limes superior a limes inferior}
Nech� $\{a_n\}_{n \in \mathbb{N}}$ je posloupnost re�ln�ch ��sel. Je-li $\{a_n\}$ shora omezen�, definujeme posloupnost $\{b_n\}_{n \in \mathbb{N}}$ p�edpisem:
$$
	b_n := \sup\{a_k; k \ge n \}
$$
Je-li $\{a_n\}$ zdola omezen�, definujeme posloupnost $\{c_n\}_{n \in \mathbb{N}}$ p�edpisem:
$$
	c_n := \inf\{a_k; k \ge n \}
$$
V takov�m p��pad� definujeme:
$$
\lim \sup~a_n := \left\{
\begin{array}{ll} \lim_{n \rightarrow \infty} b_n & \textit{jestli�e je } \{a_n\} \textit{ shora omezen�} \\ \infty & \textit{jestli�e je } \{a_n\} \textit{ shora neomezen�} \end{array}
\right.
$$
Tuto hodnotu naz�v�me \emph{limes superior} posloupnosti $\{a_n\}_{n \in \mathbb{N}}$. Obdobn� definujeme \emph{limes inferior} posloupnosti $\{a_n\}_{n \in \mathbb{N}}$ p�edpisem:
$$
\lim \inf~a_n := \left\{
\begin{array}{ll} \lim_{n \rightarrow \infty} c_n & \textit{jestli�e je } \{a_n\} \textit{ zdola omezen�} \\ - \infty & \textit{jestli�e je } \{a_n\} \textit{ zdola neomezen�} \end{array}
\right.
$$
\end{definiceN}

\begin{poznamka}
Limes superior a limes inferior jsou v�dy dob�e definovan� hodnoty a plat�
$$
\lim \sup~a_n \in \mathbb{R}^*, \lim \inf~a_n \in \mathbb{R}^*, 
$$
Na rozd�l od limity, kter� nemus� existovat, tyto dv� hodnoty existuj� pro
libovolnou posloupnost re�ln�ch ��sel.
\end{poznamka}

\begin{poznamka}
    Limes superior a limes inferior jsou n�kdy definov�ny jako $\inf \sup$ a $\sup \inf$ -- jeliko� jde o monot�nn� posloupnosti, jde ale o tot�.
\end{poznamka}

\begin{vetaN}{o vztahu limity, limes superior a limes inferior}
Nech� $\{a_n\}_{n \in \mathbb{N}}$ je posloupnost re�ln�ch ��sel. Potom 
$$
	\lim a_n = A \in \mathbb{R}^{*}
$$
pr�v� tehdy, kdy�
$$
\lim \sup~a_n = \lim \inf~a_n = A \in \mathbb{R}^*
$$

\medskip\begin{dukaz}
Pro vlastn� limity:

$\Rightarrow$ : Nech� $\lim_n a_n = A$, zvolme libovoln� $\varepsilon > 0$. Zvolme $n_0>0$ takov�, �e pro $n\geq n_0$ je $A-\varepsilon < a_n < A + \varepsilon$. Tedy pro $n \geq n_0$ je $A-\varepsilon \leq \inf_{k \geq n}a_n < A + \varepsilon$. Jeliko� posloupnost $\left\{ \left(\inf_{k\geq n}a_k\right)_n\right\}$ nekles� a je shora omezen�, mus� limita existovat a mus� platit $\lim_{n \rightarrow \infty} \leq A+\varepsilon$. 

Proto�e $\varepsilon$ je nekone�n� mal�, plat� $\lim \inf a_n = A$. 

Podobn� pro $\lim \sup.$

$\Leftarrow$ : Nech� $\lim \sup~a_n = \lim \inf~a_n = A$. Potom pro ka�d� $\varepsilon >0$existuje $n_1>0$ takov�, �e pro $n\geq n_1$ je $A-\varepsilon < \sup_{k\geq n}a_k < A+\varepsilon$. Tak� existuje $n_2>0$ takov�, �e pro $n\geq n_2$ je $A-\varepsilon < \inf_{k\geq n}a_k < A+\varepsilon$. Vezmeme $n_0=\max\left(n_1,n_2\right)$; pro $n\geq n_0$ je $\sup_{k\geq n}a_k$ i $\inf_{k\geq n}a_k$ mezi $A-\varepsilon$ a $A\varepsilon$. Ale proto�e je v tomto intervalu supr�mum i infimum, mus� v tomto intervalu b�t i v�echny hodnoty $a_n \forall n \geq n_0$.

\noindent Pro nevlastn� limity:

Dok�eme pro $\infty$, pro $-\infty$ se dok�e obdobn�.

$\Rightarrow$: Zvolme $K$ libovoln�. Potom je-li $ \lim \inf~a_n = \infty,$ je pro n�jak� $n$ $\inf_{k\geq n}a_k > K$, tedy $\forall k\geq n$ je $a_k > K$.

$\Leftarrow$: Je-li $ \lim~a_n = \infty$, pro dostate�n� velk� $n$ plat�, �e $k\geq n \Rightarrow a_k>K$, tedy $\inf_{k\geq n}a_k > K$ a $\sup_{k\geq n}a_k > K$.

\end{dukaz}


\end{vetaN}

\begin{definice}
Nech� $\{a_n\}_{n \in \mathbb{N}}$ je posloupnost re�ln�ch ��sel. Pak $A \in \mathbb{R}^*$ nazveme \emph{hromadnou hodnotou} posloupnosti $\{a_n\}$, jestli�e existuje vybran� posloupnost takov� $\{a_{n_k}\}$, �e $\lim_{k \rightarrow \infty} a_{n_k} = A$. Mno�ina v�ech hromadn�ch hodnot zna��me $H(\{a_n\})$
\end{definice}

\begin{vetaN}{o vztahu limes superior, limes inferior a hromadn�ch hodnot}
Nech� $\{a_n\}_{n \in \mathbb{N}}$ je posloupnost re�ln�ch ��sel. Potom $H\{(a_n)\} \neq \emptyset$,
$$
\lim \sup~a_n = \max H(\{a_n\}) \textit{ a } \lim \inf~a_n = \min H(\{a_n\})
$$
\end{vetaN}

K tomuhle jsem d�kaz nena�el ani nevymyslel :-(

\subsection{Cauchyovsk� posloupnosti}

Tato sekce je vypracovan� podle skript Prof. A. Pultra z~matematick� anal�zy\\
(\texttt{http://kam.mff.cuni.cz/\~{}pultr/})
\bigskip

\begin{definiceN}{Bolzano-Cauchyova podm�nka}
�ekneme, �e posloupnost $\{a_n\}_{n\geq 0}$ je \emph{cauchyovsk�}, nebo-li �e spl�uje \emph{Bolzano-Cauchyho podm�nku}, pokud pro n� plat�:
$$
\forall \varepsilon > 0~~\exists n_0 \in \mathbb{N}: \forall m, n \in \mathbb{N}: m \ge n_0, n \ge n_0: |a_n - a_m| < \varepsilon
$$
\end{definiceN}

\begin{vetaN}{Bolzano-Weierstrassova}
Z ka�d� omezen� posloupnosti lze vybrat konvergentn� podposloupnost. Jsou-li $a_n$ v kompaktn�m intervalu $[a,b]$, je limita vybran� posloupnosti v tomto intervalu.

\medskip\begin{dukaz}
Prvn� ��st dok�eme nalezen�m takov� posloupnosti. Vezmeme $A:=\lim\sup~a_n$ a definujeme pro ka�d� $k\in\mathbb{N}$ mno�inu $M_k:=\{j\in\mathbb{N}|j>n_{k-1},a_j\in\langle A - \frac{1}{2^k}, A + \frac{1}{2^k}\rangle\}$ a $n_k:=\min M_k$. Potom $\{a_{n_k}\}$ je vybran� posloupnost, kter� konverguje. Druh� ��st je p��m�m d�sledkem v�ty o limit� a uspo��d�n�.
\end{dukaz}
\end{vetaN}

\begin{lemma}
M�-li cauchyovsk� posloupnost konvergentn� podposloupnost, je konvergentn�.

\medskip\begin{dukaz}
Nech� $\lim a_{n_k} = x$. Pro $\varepsilon > 0$ zvolme $n_0$, aby pro $m,n\geq n_0$ platilo $|a_m-a_n|<\frac{\varepsilon}{2}$ a $|a_k-x|<\frac{\varepsilon}{2}$. Proto�e $k_n\geq n$, plat� $|a_n-x|=|a_n-a_{k_n}|+|a_{k_n}-x|<\varepsilon$.
\end{dukaz}
\end{lemma}


\begin{vetaN}{Bolzano-Cauchyova}
Posloupnost $\{a_n\}$ m� vlastn� limitu, pr�v� kdy� je cauchyovsk�.

\medskip\begin{dukaz}
\begin{penumerate}
    \item Implikace \uv{$\Rightarrow$} je hned vid�t -- sta�� vz�t k $\varepsilon$ takov� $n_0$, �e $|a_n-x|<\frac{\varepsilon}{2}\ \forall n\geq n_0$. Potom je $|a_n-a_m|=|a_n-x+x-a_m|\leq|a_n-x|+|x-a_m|\leq\varepsilon\ \forall m,n\geq n_0$.\medskip
    \item Pro druhou implikaci sta�� dok�zat, �e je cauchyovsk� posloupnost omezen� a zbytek dostaneme z~p�edchoz�ho lemmatu a Bolzano-Weierstrassovy v�ty. Pro $\varepsilon=1$ existuje $n_0$ takov�, �e $a_{n_0}-1<a_n<a_{n_0}+1$ pro ka�d� $n\geq n_0$ (to plyne p��mo z~podm�nky), tak�e zb�v� jen kone�n� po�et �len� mimo toto rozmez� (pro $n<n_0$), a ty v�dy tvo�� omezen� syst�m.
\end{penumerate}
\end{dukaz}
\end{vetaN}



\end{document}
